<h1>Эффективные алгоритмы факторизации</h1>

<hr>

<p>Здесь приведены реализации нескольких алгоритмов факторизации, каждый из которых по отдельности может работать как быстро, так и очень медленно, но в сумме они дают весьма быстрый метод.</p>
<p>Описания этих методов не приводятся, тем более что они достаточно хорошо описаны в Интернете.</p>

<h2>Метод Полларда p-1</h2>
<p>Вероятностный тест, быстро даёт ответ далеко не для всех чисел.</p>
<p>Возвращает либо найденный делитель, либо 1, если делитель не был найден.</p>
<code>template <class T>
T pollard_p_1 (T n)
{
	// параметры алгоритма, существенно влияют на производительность и качество поиска
	const T b = 13;
	const T q[] = { 2, 3, 5, 7, 11, 13 };

	// несколько попыток алгоритма
	T a = 5 % n;
	for (int j=0; j<10; j++)
	{

		// ищем такое a, которое взаимно просто с n
		while (gcd (a, n) != 1)
		{
			mulmod (a, a, n);
			a += 3;
			a %= n;
		}

		// вычисляем a^M
		for (size_t i = 0; i < sizeof q / sizeof q[0]; i++)
		{
			T qq = q[i];
			T e = (T) floor (log ((double)b) / log ((double)qq));
			T aa = powmod (a, powmod (qq, e, n), n);
			if (aa == 0)
				continue;
			
			// проверяем, не найден ли ответ
			T g = gcd (aa-1, n);
			if (1 < g && g < n)
				return g;
		}

	}

	// если ничего не нашли
	return 1;

}</code>

<h2>Метод Полларда "Ро"</h2>
<p>Вероятностный тест, быстро даёт ответ далеко не для всех чисел.</p>
<p>Возвращает либо найденный делитель, либо 1, если делитель не был найден.</p>
<code>template <class T>
T pollard_rho (T n, unsigned iterations_count = 100000)
{
	T
		b0 = rand() % n,
		b1 = b0,
		g;
	mulmod (b1, b1, n);
	if (++b1 == n)
		b1 = 0;
	g = gcd (abs (b1 - b0), n);
	for (unsigned count=0; count<iterations_count && (g == 1 || g == n); count++)
	{
		mulmod (b0, b0, n);
		if (++b0 == n)
			b0 = 0;
		mulmod (b1, b1, n);
		++b1;
		mulmod (b1, b1, n);
		if (++b1 == n)
			b1 = 0;
		g = gcd (abs (b1 - b0), n);
	}
	return g;
}</code>

<h2>Метод Бента (модификация метода Полларда "Ро")</h2>
<p>Вероятностный тест, быстро даёт ответ далеко не для всех чисел.</p>
<p>Возвращает либо найденный делитель, либо 1, если делитель не был найден.</p>
<code>template <class T>
T pollard_bent (T n, unsigned iterations_count = 19)
{
	T
		b0 = rand() % n,
		b1 = (b0*b0 + 2) % n,
		a = b1;
	for (unsigned iteration=0, series_len=1; iteration<iterations_count; iteration++, series_len*=2)
	{
		T g = gcd (b1-b0, n);
		for (unsigned len=0; len<series_len && (g==1 && g==n); len++)
		{
			b1 = (b1*b1 + 2) % n;
			g = gcd (abs(b1-b0), n);
		}
		b0 = a;
		a = b1;
		if (g != 1 && g != n)
			return g;
	}
	return 1;
}</code>

<h2>Метод Полларда Монте-Карло</h2>
<p>Вероятностный тест, быстро даёт ответ далеко не для всех чисел.</p>
<p>Возвращает либо найденный делитель, либо 1, если делитель не был найден.</p>
<code>template <class T>
T pollard_monte_carlo (T n, unsigned m = 100)
{
	T b = rand() % (m-2) + 2;

	static std::vector<T> primes;
	static T m_max;
	if (primes.empty())
		primes.push_back (3);
	if (m_max < m)
	{
		m_max = m;
		for (T prime=5; prime<=m; ++++prime)
		{
			bool is_prime = true;
			for (std::vector<T>::const_iterator iter=primes.begin(), end=primes.end();
				iter!=end; ++iter)
			{
				T div = *iter;
				if (div*div > prime)
					break;
				if (prime % div == 0)
				{
					is_prime = false;
					break;
				}
			}
			if (is_prime)
				primes.push_back (prime);
		}
	}

	T g = 1;
	for (size_t i=0; i<primes.size() && g==1; i++)
	{
		T cur = primes[i];
		while (cur <= n)
			cur *= primes[i];
		cur /= primes[i];
		b = powmod (b, cur, n);
		g = gcd (abs(b-1), n);
		if (g == n)
			g = 1;
	}

	return g;
}</code>

<h2>Метод Ферма</h2>
<p>Это стопроцентный метод, но он может работать очень медленно, если у числа есть маленькие делители.</p>
<p>Поэтому запускать его стоит только после всех остальных методов.</p>
<code>template <class T, class T2>
T ferma (const T & n, T2 unused)
{
	T2
		x = sq_root (n),
		y = 0,
		r = x*x - y*y - n;
	for (;;)
		if (r == 0)
			return x!=y ? x-y : x+y;
		else
			if (r > 0)
			{
				r -= y+y+1;
				++y;
			}
			else
			{
				r += x+x+1;
				++x;
			}
}</code>

<h2>Тривиальное деление</h2>
<p>Этот элементарный метод пригодится, чтобы сразу обрабатывать числа с очень маленькими делителями.</p>
<code>template <class T, class T2>
T2 prime_div_trivial (const T & n, T2 m)
{
	
	// сначала проверяем тривиальные случаи
	if (n == 2 || n == 3)
		return 1;
	if (n < 2)
		return 0;
	if (even (n))
		return 2;

	// генерируем простые от 3 до m
	T2 pi;
	const vector<T2> & primes = get_primes (m, pi);

	// делим на все простые
	for (std::vector<T2>::const_iterator iter=primes.begin(), end=primes.end();
		iter!=end; ++iter)
	{
		const T2 & div = *iter;
		if (div * div > n)
			break;
		else
			if (n % div == 0)
				return div;
	}
	
	if (n < m*m)
		return 1;
	return 0;

}</code>

<h2>Собираем всё вместе</h2>
<p>Объединяем все методы в одной функции.</p>
<p>Также функция использует тест на простоту, иначе алгоритмы факторизации могут работать очень долго. Например, можно выбрать тест BPSW (<algohref=bpsw>читать статью по BPSW</algohref>).</p>
<code>template <class T, class T2>
void factorize (const T & n, std::map<T,unsigned> & result, T2 unused)
{
	if (n == 1)
		;
	else
		// проверяем, не простое ли число
		if (isprime (n))
			++result[n];
		else
			// если число достаточно маленькое, то его разлагаем простым перебором
			if (n < 1000*1000)
			{
				T div = prime_div_trivial (n, 1000);
				++result[div];
				factorize (n / div, result, unused);
			}
			else
			{
				// число большое, запускаем на нем алгоритмы факторизации
				T div;
				// сначала идут быстрые алгоритмы Полларда
				div = pollard_monte_carlo (n);
				if (div == 1)
					div = pollard_rho (n);
				if (div == 1)
					div = pollard_p_1 (n);
				if (div == 1)
					div = pollard_bent (n);
				// придётся запускать 100%-ый алгоритм Ферма
				if (div == 1)
					div = ferma (n, unused);
				// рекурсивно обрабатываем найденные множители
				factorize (div, result, unused);
				factorize (n / div, result, unused);
			}
}</code>

<hr>

<h2>Приложение</h2>
<p><a href="factorization.zip">Скачать [5 КБ]</a> исходник программы, которая использует все указанные методы факторизации и тест BPSW на простоту.</p>