\h1{Дискретное извлечение корня}

Задача дискретного извлечения корня (по аналогии с \algohref=discrete_log{задачей дискретного логарифма}) звучит следующим образом. По данным $n$ ($n$ --- простое), $a$, $k$ требуется найти все $x$, удовлетворяющие условию:
$$ x^k \equiv a \pmod{n} $$

\h2{Алгоритм решения}

Решать задачу будем сведением её к задаче дискретного логарифма.

Для этого применим понятие \algohref=primitive_root{Первообразного корня по модулю $n$}. Пусть $g$ --- первообразный корень по модулю $n$ (т.к. $n$ --- простое, то он существует). Найти его мы можем, как описано в соответствующей статье, за $O( {\rm Ans} \cdot \log \phi(n) \cdot \log n) = O( {\rm Ans} \cdot \log^2 n)$ плюс время факторизации числа $\phi(n)$.

Отбросим сразу случай, когда $a=0$ --- в этом случае сразу находим ответ $x=0$.

Поскольку в данном случае ($n$ --- простое) любое число от $1$ до $n-1$ представимо в виде степени первообразного корня, то задачу дискретного корня мы можем представить в виде:
$$ {\left( g^y \right)}^k \equiv a \pmod{n} $$
где
$$ x \equiv g^y \pmod{n} $$
Тривиальным преобразованием получаем:
$$ {\left( g^k \right)}^y \equiv a \pmod{n} $$
Здесь искомой величиной является $y$, таким образом, мы пришли к задаче дискретного логарифмирования в чистом виде. Эту задачу можно решить \algohref=discrete_log{алгоритмом baby-step-giant-step Шэнкса} за $O( \sqrt{n} \log n )$, т.е. найти одно из решений $y_0$ этого уравнения (или обнаружить, что это уравнение решений не имеет).

Пусть мы нашли некоторое решение $y_0$ этого уравнения, тогда одним из решений задачи дискретного корня будет $x_0 = g^{y_0} \pmod{n}$.

\h2{Нахождение всех решений, зная одно из них}

Чтобы полностью решить поставленную задачу, надо научиться по одному найденному $x_0 = g^{y_0} \pmod{n}$ находить все остальные решения.

Для этого вспомним такой факт, что первообразный корень всегда имеет порядок $\phi(n)$ (см. \algohref=primitive_root{статью о первообразном корне}), т.е. наименьшей степенью $g$, дающей единицу, является $\phi(n)$. Поэтому добавление в показатель степени слагаемого с $\phi(n)$ ничего не меняет:
$$ x^k \equiv g^{ y_0 \cdot k + l \cdot \phi(n) } \equiv a \pmod{n}\ \ \ \forall\ l \in {\cal Z} $$
Отсюда все решения имеют вид:
$$ x = g^{ y_0 + \frac{ l \cdot \phi(n) }{ k } } \pmod{n}\ \ \ \forall\ l \in {\cal Z} $$
где $l$ выбирается таким образом, чтобы дробь $\frac{ l \cdot \phi(n) }{ k }$ была целой. Чтобы эта дробь была целой, числитель должен быть кратен наименьшему общему кратному $\phi(n)$ и $k$, откуда (вспоминая, что наименьшее общее кратное двух чисел ${\rm lcm}(a,b) = \frac{ a \cdot b }{ {\rm gcd}(a,b) }$), получаем:
$$ x = g^{ y_0 + i \frac{ \phi(n) }{ {\rm gcd}(k,\phi(n)) } } \pmod{n}\ \ \ \forall\ i \in {\cal Z} $$
Это окончательная удобная формула, которая даёт общий вид всех решений задачи дискретного корня.

\h2{Реализация}

Приведём полную реализацию, включающую нахождение первообразного корня, дискретное логарифмирование и нахождение и вывод всех решений.

\code
int gcd (int a, int b) {
	return a ? gcd (b%a, a) : b;
}

int powmod (int a, int b, int p) {
	int res = 1;
	while (b)
		if (b & 1)
			res = int (res * 1ll * a % p),  --b;
		else
			a = int (a * 1ll * a % p),  b >>= 1;
	return res;
}

int generator (int p) {
	vector<int> fact;
	int phi = p-1,  n = phi;
	for (int i=2; i*i<=n; ++i)
		if (n % i == 0) {
			fact.push_back (i);
			while (n % i == 0)
				n /= i;
		}
	if (n > 1)
		fact.push_back (n);

	for (int res=2; res<=p; ++res) {
		bool ok = true;
		for (size_t i=0; i<fact.size() && ok; ++i)
			ok &= powmod (res, phi / fact[i], p) != 1;
		if (ok)  return res;
	}
	return -1;
}

int main() {

	int n, k, a;
	cin >> n >> k >> a;
	if (a == 0) {
		puts ("1\n0");
		return 0;
	}

	int g = generator (n);

	int sq = (int) sqrt (n + .0) + 1;
	vector < pair<int,int> > dec (sq);
	for (int i=1; i<=sq; ++i)
		dec[i-1] = make_pair (powmod (g, int (i * sq * 1ll * k % (n - 1)), n), i);
	sort (dec.begin(), dec.end());
	int any_ans = -1;
	for (int i=0; i<sq; ++i) {
		int my = int (powmod (g, int (i * 1ll * k % (n - 1)), n) * 1ll * a % n);
		vector < pair<int,int> >::iterator it =
			lower_bound (dec.begin(), dec.end(), make_pair (my, 0));
		if (it != dec.end() && it->first == my) {
			any_ans = it->second * sq - i;
			break;
		}
	}
	if (any_ans == -1) {
		puts ("0");
		return 0;
	}

	int delta = (n-1) / gcd (k, n-1);
	vector<int> ans;
	for (int cur=any_ans%delta; cur<n-1; cur+=delta)
		ans.push_back (powmod (g, cur, n));
	sort (ans.begin(), ans.end());
	printf ("%d\n", ans.size());
	for (size_t i=0; i<ans.size(); ++i)
		printf ("%d ", ans[i]);

}
\endcode