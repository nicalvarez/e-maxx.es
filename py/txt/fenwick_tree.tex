<h1>Дерево Фенвика</h1>

<p><b>Дерево Фенвика</b> - это структура данных, дерево на массиве, обладающее следующими свойствами:</p>
<p>1) позволяет вычислять значение некоторой обратимой операции G на любом отрезке [L; R] за время <b>O (log N)</b>;</p>
<p>2) позволяет изменять значение любого элемента за <b>O (log N)</b>;</p>
<p>3) требует <b>O (N) памяти</b>, а точнее, ровно столько же, сколько и массив из N элементов;</p>
<p>4) легко обобщается на случай многомерных массивов.</p>
<p>Наиболее распространённое применение дерева Фенвика - для вычисления суммы на отрезке, т.е. функция G (X1, ..., Xk) = X1 + ... + Xk.</p>
<p>Дерево Фенвика было впервые описано в статье "A new data structure for cumulative frequency tables" (Peter M. Fenwick, 1994).</p>
<h2>Описание</h2>
<p>Для простоты описания мы предполагаем, что операция G, по которой мы строим дерево, - это <b>сумма</b>.</p>
<p>Пусть дан массив A[0..N-1]. Дерево Фенвика - массив <b>T</b>[0..N-1], в каждом элементе которого хранится сумма некоторых элементов массива A:</p>
<formula><b>T<sub>i</sub> = сумма A<sub>j</sub></b> для всех <b>F(i) <= j <= i</b>,</formula>
<p>где F(i) - некоторая функция, которую мы определим несколько позже.</p>
<p>Теперь мы уже можем написать <b>псевдокод</b> для функции вычисления суммы на отрезке [0; R] и для функции изменения ячейки:</p>
<code>int sum (int r)
{
	int result = 0;
	while (r >= 0) {
		result += t[r];
		r = f(r) - 1;
	}
	return result;
}

void inc (int i, int delta)
{
	для всех j, для которых F(j) <= i <= j
	{
		t[j] += delta;
	}
}</code>
<p>Функция sum работает следующим образом. Вместо того чтобы идти по всем элементам массива A, она движется по массиву T, делая "прыжки" через отрезки там, где это возможно. Сначала она прибавляет к ответу значение суммы на отрезке [F(R); R], затем берёт сумму на отрезке [F(F(R)-1); F(R)-1], и так далее, пока не дойдёт до нуля.</p>
<p>Функция inc движется в обратную сторону - в сторону увеличения индексов, обновляя значения суммы T<sub>j</sub> только для тех позиций, для которых это нужно, т.е. для всех j, для которых F(j) <= i <= j.</p>
<p>Очевидно, что от выбора функции F будет зависеть скорость выполнения обеих операций. Сейчас мы рассмотрим функцию, которая позволит достичь логарифмической производительности в обоих случаях.</p>
<p><b>Определим значение F(X)</b> следующим образом. Рассмотрим двоичную запись этого числа и посмотрим на его младший бит. Если он равен нулю, то F(X) = X. Иначе двоичное представление числа X оканчивается на группу из одной или нескольких единиц. Заменим все единицы из этой группы на нули, и присвоим полученное число значению функции F(X).</p>
<p>Этому довольно сложному описанию соответствует очень простая формула:</p>
<formula><b>F(X) = X & (X+1)</b>,</formula>
<p>где & - это операция побитового логического "И".</p>
<p>Нетрудно убедиться, что эта формула соответствует словесному описанию функции, данному выше.</p>
<p> </p>
<p>Нам осталось только научиться быстро находить такие числа j, для которых F(j) <= i <= j.</p>
<p>Однако нетрудно убедиться в том, что все такие числа j получаются из i последовательными заменами самого правого (самого младшего) нуля в двоичном представлении. Например, для i = 10 мы получим, что j = 11, 15, 31, 63 и т.д.</p>
<p>Как ни странно, такой операции (замена самого младшего нуля на единицу) также соответствует очень простая формула:</p>
<formula><b>H(X) = X | (X+1)</b>,</formula>
<p>где | - это операция побитового логического "ИЛИ".</p>
<h2>Реализация дерева Фенвика для суммы для одномерного случая</h2>
<code>vector<int> t;
int n;

void init (int nn)
{
	n = nn;
	t.assign (n, 0);
}

int sum (int r)
{
	int result = 0;
	for (; r >= 0; r = (r & (r+1)) - 1)
		result += t[r];
	return result;
}

void inc (int i, int delta)
{
	for (; i < n; i = (i | (i+1)))
		t[i] += delta;
}

int sum (int l, int r)
{
	return sum (r) - sum (l-1);
}

void init (vector<int> a)
{
	init ((int) a.size());
	for (unsigned i = 0; i < a.size(); i++)
		inc (i, a[i]);
}</code>
<h2>Реализация дерева Фенвика для минимума для одномерного случая</h2>
<p>Следует сразу заметить, что, поскольку дерево Фенвика позволяет найти значение функции в произвольном отрезке [0;R], то мы никак не сможем найти минимум на отрезке [L;R], где L > 0. Далее, все изменения значений должны происходить только в сторону уменьшения (опять же, поскольку никак не получится обратить функцию min). Это значительные ограничения.</p>
<code>vector<int> t;
int n;

const int INF = 1000*1000*1000;

void init (int nn)
{
	n = nn;
	t.assign (n, INF);
}

int getmin (int r)
{
	int result = INF;
	for (; r >= 0; r = (r & (r+1)) - 1)
		result = min (result, t[r]);
	return result;
}

void update (int i, int new_val)
{
	for (; i < n; i = (i | (i+1)))
		t[i] = min (t[i], new_val);
}

void init (vector<int> a)
{
	init ((int) a.size());
	for (unsigned i = 0; i < a.size(); i++)
		update (i, a[i]);
}</code>
<h2>Реализация дерева Фенвика для суммы для двумерного случая</h2>
<p>Как уже отмечалось, дерево Фенвика легко обобщается на многомерный случай.</p>
<code>vector <vector <int> > t;
int n, m;

int sum (int x, int y)
{
	int result = 0;
	for (int i = x; i >= 0; i = (i & (i+1)) - 1)
		for (int j = y; j >= 0; j = (j & (j+1)) - 1)
			result += t[i][j];
	return result;
}

void inc (int x, int y, int delta)
{
	for (int i = x; i < n; i = (i | (i+1)))
		for (int j = y; j < m; j = (j | (j+1)))
			t[i][j] += delta;
}</code>