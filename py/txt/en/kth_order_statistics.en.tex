the <h1>K-th order statistics in O (N)</h1>

<p>Suppose you are given an array A of length N and let an integer K. the Goal is to find in this array the K-th largest number, i.e. K-th order statistics.</p>
<p> </p>
<p>the Main idea is to use the ideas of the quicksort. In fact, the algorithm is simple, more difficult to prove that it works in average O (N), unlike quick sort.</p>
<p> </p>
<p>the Implementation in the form of non-recursive functions:</p>
<code>template <class T>
T order_statistics (std::vector<T> a, unsigned n, unsigned k)
{
using std::swap;
for (unsigned l=1, r=n; ; )
{

if (r <= l+1)
{
// current portion consists of 1 or 2 elements
// can easily find the answer
if (r == l+1 && a[r] < a[l])
swap (a[l], a[r]);
return a[k];
}

// updating a[l], a[l+1], a[r]
unsigned mid = (l + r) >> 1;
swap (a[mid], a[l+1]);
if (a[l] > a[r])
swap (a[l], a[r]);
if (a[l+1] > a[r])
swap (a[l+1], a[r]);
if (a[l] > a[l+1])
swap (a[l], a[l+1]);

// divide
// barrier is a[l+1], i.e., the median among a[l], a[l+1], a[r]
unsigned
i = l+1,
j = r;
const T
cur = a[l+1];
for (;;)
{
while (a[++i] < cur) ;
while (a[--j] > cur) ;
if (i > j)
break;
swap (a[i], a[j]);
}

// insert a barrier
a[l+1] = a[j];
a[j] = cur;

// continue to work in that part,
// which should contain the search item
if (j >= k)
r = j-1;
if (j <= k)
l = i;

}
}</code>
<p> </p>
<p>note that in standard library for C++ this algorithm is already implemented - it's called nth_element.</p>