the <h1>Bit</h1>

<p><b>Bit</b> is a data structure, the tree in the array with the following properties:</p>
<p>1) allows to calculate the value of some reversible operation G on any interval [L; R] during <b>O (log N)</b>;</p>
<p>2) allows you to change the value of any element for <b>O (log N)</b>;</p>
<p>3) it requires <b>O (N) memory</b>, and more precisely, exactly the same as an array of N elements;</p>
<p>4) is easily generalized to the case of multidimensional arrays.</p>
<p>the Most common use of Fenwick tree is to compute the sum on an interval, i.e. the function G (X1, ..., Xk) = X1 + ... + Xk.</p>
<p>Fenwick's Tree was first described in the paper "A new data structure for cumulative frequency tables" (Peter M. Fenwick, 1994).</p>
<h2>Description</h2>
<p>For simplicity, we assume that the operation G, we construct a tree, is a <b>sum</b>.</p>
<p>Suppose you are given an array A[0..N-1]. Fenwick tree is an array of <b>T</b>[0..N-1], each element of which stores the sum of elements of array A:</p>
<formula><b>T<sub>i</sub> = sum<sub>j</sub></b> for all <b>F(i) <= j <= i</b></formula>
<p>where F(i) is some function, which we will define later.</p>
<p>Now we can write <b>pseudo-code</b> for the function computing the sum on the interval [0; R] and for a function of changing cell:</p>
<code>int sum (int r)
{
int result = 0;
while (r >= 0) {
result += t[r];
r = f(r) - 1;
}
return result;
}

void inc (int i, int delta)
{
for all j for which F(j) <= i <= j
{
t[j] += delta;
}
}</code>
<p>the sum Function works as follows. Instead of going through all of the elements of the array A, it moves through the array T, making "jumps" through the segments where possible. First, it adds to the reply value of the sum on the interval [F(R); R], then takes the sum on the interval [F(F(R)-1); F(R)-1], and so on, until you reach zero.</p>
<p>the Function inc is moving in the opposite direction - in the direction of increasing indices, updating the values of the sum T<sub>j</sub> only for those positions for which it is needed, i.e., for all j, for which F(j) <= i <= j.</p>
<p>Obviously, the function F will depend on the speed of both operations. We now consider a function that will allow you to achieve logarithmic performance in both cases.</p>
<p><b>we Determine the value of F(X)</b> as follows. Consider the binary notation of this number and look at his little bit. If it is zero, then F(X) = X. Otherwise, the binary representation of a number X ends in the group of one or more units. Replace all units from this group to zeros, and assign the resulting number value to the function F(X).</p>
<p>This rather complex description corresponds to a very simple formula:</p>
<formula><b>F(X) = X & (X+1)</b></formula>
<p>where & is a bitwise logical "And".</p>
<p>it is Easy to verify that this formula corresponds to a verbal description of the function given above.</p>
<p> </p>
<p>We just have to learn how to quickly find all such numbers j for which F(j) <= i <= j.</p>
<p>However, it is easy to verify that all such numbers j i are obtained from successive substitutions of the rightmost (youngest) of zero in the binary representation. For example, for i = 10 we get that j = 11, 15, 31, 63, etc.</p>
<p>oddly enough, such a transaction (replacement of the lowest zero per unit) also corresponds to a very simple formula:</p>
<formula><b>H(X) = X | (X+1)</b></formula>
<p>where | is the bitwise logical "OR".</p>
the <h2>implementation of a binary indexed tree for the amount for the one-dimensional case</h2>
<code>vector<int> t;
int n;

void init (int nn)
{
n = nn;
t.assign (n, 0);
}

int sum (int r)
{
int result = 0;
for (; r >= 0; r = (r & (r+1)) - 1)
result += t[r];
return result;
}

void inc (int i, int delta)
{
for (; i < n; i = (i | (i+1)))
t[i] += delta;
}

int sum (int l, int r)
{
return sum (r) sum (l-1);
}

void init (vector<int> a)
{
init ((int) a.size());
for (unsigned i = 0; i < a.size(); i++)
inc (i, a[i]);
}</code>
the <h2>implementation of a binary indexed tree for the minimum one-dimensional case</h2>
<p>it Should be noted that, as Fenwick tree allows you to find the function value at an arbitrary interval [0;R], then we won't be able to find a minimum of on the interval [L;R], where L > 0. Further, all changes should happen only in the direction of decreasing (again, because there's no way of reversing the min function). This is a significant limitation.</p>
<code>vector<int> t;
int n;

const int INF = 1000*1000*1000;

void init (int nn)
{
n = nn;
t.assign (n, INF);
}

int getmin (int r)
{
int result = INF;
for (; r >= 0; r = (r & (r+1)) - 1)
result = min (result, t[r]);
return result;
}

void update (int i, int new_val)
{
for (; i < n; i = (i | (i+1)))
t[i] = min (t[i], new_val);
}

void init (vector<int> a)
{
init ((int) a.size());
for (unsigned i = 0; i < a.size(); i++)
update (i, a[i]);
}</code>
the <h2>implementation of a binary indexed tree for the sum for the two-dimensional case</h2>
<p>As already noted, the Bit easily generalized to the multidimensional case.</p>
<code>vector <vector <int> > t;
int n, m;

int sum (int x, int y)
{
int result = 0;
for (int i = x; i >= 0; i = (i & (i+1)) - 1)
for (int j = y; j >= 0; j = (j & (j+1)) - 1)
result += t[i][j];
return result;
}

void inc (int x, int y, int delta)
{
for (int i = x; i < n; i = (i | (i+1)))
for (int j = y; j < m; j = (j | (j+1)))
t[i][j] += delta;
}</code>