\h1{Centers of gravity of polygons and polyhedra}

\bf{Center of gravity} (or \bf{center of mass}) of some body is any point with the property that if you hang body at this point, it will maintain its position.

Below are considered two-dimensional and three-dimensional problems associated with finding the centers of mass of --- mostly from the perspective of computational geometry.

The following solutions can be divided into two main \bf{fact}. First --- that the center of mass of a system of material points is equal to the average of their coordinates, taken with the factors proportional to their masses. The second fact --- that if we know the centres of mass of two disjoint pieces, then the center of mass of their combination will lie on the line segment connecting these two centres, and he will divide it in the same relation as the mass of the second figure refers to the weight first.


\h2{the two-Dimensional case: polygons}

Actually, speaking of the center of mass of a two-dimensional figure, can be keep in mind one of the three following \bf{tasks}:

\ul{
\li the Center of mass of the points system --- i.e. all the mass is concentrated only at the vertices of the polygon.
\li the Center of mass frame --- i.e., the mass of the polygon is concentrated on the perimeter.
\li the Center of mass of a solid figure --- i.e., the mass of the polygon are distributed over the entire area.
}

Each of these tasks has its own solution, and will be considered separately below.


\h3{Center of mass point systems}

This is the simplest of the three tasks, the solution of --- famous physical formula of the center of mass of a system of material points:

$$ \vec{r_c} = \frac{ \sum\limits_i \vec{r_i} ~ m_i }{ \sum\limits_i m_i }, $$

where $m_i$ --- mass points, $\vec{r_i}$ --- their radius-vectors (representing their position relative to the origin) and $\vec{r_c}$ --- the desired radius-vector of center of mass.

In particular, if all points have the same mass, the coordinates of the center of mass is \bf{arithmetic mean} of point coordinates. For \bf{triangle} this point is called the \bf{centroid} and coincides with the intersection point of medians:

$$ \vec{r_c} = \frac{ \vec{r_1} + \vec{r_2} + \vec{r_3} }{ 3 }. $$

For \bf{proof} these formulas it is enough to remember that equilibrium is achieved at a point $r_c$, in which the sum of the moments of all forces is equal to zero. In this case, it becomes the condition that the sum of the radius vectors of all points relative to the point $r_c$, multiplied by the respective mass points is zero:

$$ \sum\limits_i \left( \vec{r_i} - \vec{r_c} \right) m_i = \vec{0}, $$

and, here expressing $\vec{r_c}$, we obtain the required formula.


\h3{Center of mass frame}

We assume for simplicity that the frame is uniform, i.e. its density is everywhere the same.

But then each side of the polygon can be replaced with a single point --- the middle of this section (because the center of mass of the homogeneous segment is the middle of this section), with weight equal to the length of this segment.

Now we have the problem of a system of material points, and applying to it the decision from the previous paragraph, we find:

$$ \vec{r_c} = \frac{ \sum\limits_i \vec{r_i^\prime} ~ l_i }{ P }, $$

where $\vec{r_i^\prime}$ --- point to middle of the $i$-th side of the polygon, $l_i$ --- length of $i$-th side of $P$ --- perimeter, i.e. the sum of the lengths of the sides.

For \bf{triangle} we can show the following statement: this point is \bf{the point of intersection of the bisectors} of the triangle formed by the midpoints of the sides of the original triangle. (to show this, we need to use the formula given above, and then notice that the bisectors divide the sides of the resulting triangle in the same ratios as the centers of mass of these parties).


\h3{Center of mass of the solid shapes}

We believe that the mass is spread uniformly on the figure, i.e. the density in each point of the figure is equal to the same number.

\h4{triangles}

It is argued that for a triangle the answer is still the same \bf{centroid}, i.e. the point formed by the arithmetic mean of the coordinates of the vertices:

$$ \vec{r_c} = \frac{ \vec{r_1} + \vec{r_2} + \vec{r_3} }{ 3 }. $$

\h4{Case triangle: proof}

We give here an elementary proof that does not use the theory of integrals. 

The first of its kind, purely geometric, proof led Archimedes, but it was very complex, with a large number of geometric constructions. The proof given here is taken from art Apostol, Mnatsakanian "Finding Centroids the Easy Way".

The proof is to show that the center of mass of a triangle lies on one of the medians; repeating this process twice more, we will show that the center of mass lies at the point of intersection of medians, and the centroid.

We divide this triangle $T$ into four by connecting the midpoints of sides, as shown in the figure below:

\img{centroids_1.jpg}

Four of the resulting triangle similar to the triangle $T$ with ratio $1/2$.

The triangles # 1 and # 2 together form a parallelogram, the center of mass of which $c_{12}$ is the intersection point of its diagonals (since this figure is symmetrical about both diagonals, and, therefore, its center of mass must lie on each of the two diagonals). Point of $c_{12}$ is in the middle of the common side of the triangles # 1 and # 2, and lies on the median of a triangle $T$:

\img{centroids_2.jpg}

Now suppose that the vector $\vec{r}$ - vector drawn from the vertex $A$ to the center of mass of the $c_1$ of the triangle No. 1 and let the vector $\vec{m}$ is vector drawn from $A$ to a point $c_{12}$ (which, recall, is the midpoint of the side on which it lies):

\img{centroids_3.jpg}

Our aim is to show that the vector $\vec{r}$ and $\vec{m}$ are collinear.

Denote by $c_3$ and $c_4$ point, which are the centers of mass of triangles # 3 and # 4. Then, obviously, the center of mass of the combination of these two triangles will be the point of $c_{34}$, which is the midpoint of the segment $c_3 c_4$. Moreover, the vector from point $c_{12}$ to a point $c_{34}$ coincides with the vector $\vec{r}$.

The desired center of mass $c$ of a triangle $T$ lies in the middle of the segment connecting the points $c_{12}$ and $c_{34}$ (since we broke the triangle $T$ into two parts of equal areas: No. 1-No. 2 and No. 3-No. 4):

\img{centroids_4.jpg}

Thus, the vector from vertex $A$ to the centroid of $c$ is $ $\vec{m} + \vec{r}/2$. On the other hand, since the triangle No. 1 is similar to triangle $T$ with the factor $1/2$, then this vector is $ $2 \vec{r}$. This yields the equation:

$$ \vec{m} + \vec{r}/2 = 2 \vec{r}, $$

whence we find:

$$ \vec{r} = \frac{2}{3} \vec{m}. $$

Thus, we have proved that the vector $\vec{r}$ and $\vec{m}$ are collinear, which means that the desired centroid $c$ lies on the median emanating from a vertex $A$.

Moreover, along the way, we have proven that the centroid divides each median in the ratio $2:1$, counting from the top.



\h4{Case polygon}

Let us proceed to the General case --- i.e., to the case \bf{mnogosolnca}. For him, such reasoning no longer applies, so we reduce the problem to triangular: namely, we divide the polygon into triangles (i.e. triangulorum it), find the center of mass of each triangle and then find the center of mass of the resulting centers of mass of triangles.

The final formula is derived as follows:

$$ \vec{r_c} = \frac{ \sum\limits_i \vec{r_i^\circ} ~ S_i }{ S }, $$

where $\vec{r_i^\circ}$ --- the centroid of $i$-th triangle in the triangulation of the given polygon, $S_i$ --- the area of the $i$-th triangle of the triangulation of $S$ --- the area of the polygon.

Triangulation of convex polygon --- trivial task: for example, we can take the triangles $(r_1,r_{i-1},r_i)$, where $i = 3 \ldots n$.

\h4{Case polygon: alternative method}

On the other hand, the application of the above formula is not very convenient for \bf{non-convex polygons}, since their triangulation --- in itself no easy task. But for such polygons, you can come up with a simpler approach. Namely, the analogy with how to find the area of an arbitrary polygon: selects an arbitrary point $z$, and then summed up the iconic area of the triangles formed by this point and the points of the polygon: $S = |\sum_{i=1}^n S_{z,p_i,modification replaces{i+1}}|$. A similar technique can be applied to find the center of mass: only now we will summarize the centers of mass of triangles $(z,p_i,modification replaces{i+1})$, with coefficients taken proportional to their squares, i.e., the final formula for the center of mass is:

$$ \vec{r_c} = \frac{ \sum\limits_i {\vec r}_{z,p_i,modification replaces{i+1}}^\circ ~ S_{z,p_i,modification replaces{i+1}} }{ S }, $$

where $z$ --- arbitrary point, $p_i$ --- point polygon ${\vec r}_{z,p_i,modification replaces{i+1}}^\circ$ --- the centroid of the triangle $(z,p_i,modification replaces{i+1})$, $S_{z,p_i,modification replaces{i+1}}$ --- significant area of this triangle, $S$ --- significant area of the whole polygon (i.e. $S = \sum_{i=1}^{n} S_{z,p_i,modification replaces{i+1}}$).


\h2{the three-Dimensional case: polyhedra}

Similarly to the two-dimensional case, in 3D it is possible to speak of four possible formulations of the problem:

\ul{
\li the Center of mass of the points system --- vertices of the polyhedron.
\li the Center of mass frame --- the edges of the polyhedron.
\li the Center of mass of the surface --- i.e., the mass divided by the surface area of the polyhedron.
\li the Center of mass of a solid polyhedron --- i.e., the mass is distributed throughout the polyhedron.
}


\h3{Center of mass point systems}

As in the two-dimensional case, we can apply a physical formula and get the same result:

$$ \vec{r_c} = \frac{ \sum\limits_i \vec{r_i} ~ m_i }{ \sum\limits_i m_i }, $$

which in the case of equal masses becomes the arithmetic mean of the coordinates of all points.


\h3{Center of mass of a skeleton of a polyhedron}

Similarly to the two-dimensional case, we simply replace each edge of a polyhedron as a point particle, located in the middle of this edge, and with a mass equal to the length of this edge. If you receive a task about material points, we easily find its solution as a weighted sum of the coordinates of these points.


\h3{Center of mass of the surface of a polyhedron}

Each face of polyhedron faces --- two-dimensional figure, the center of mass which we are able to look. Finding the centres of mass and by replacing every edge of its center of mass, we get the problem with material points, which is easy to solve.


\h3{Center of mass of the solid polyhedron}

\h4{Case tetrahedron}

As in the two-dimensional case, solve simple first task --- task for the tetrahedron.

It is argued that the center of mass of the tetrahedron coincides with the point of intersection of its medians (a median of a tetrahedron is called a line segment drawn from its vertices to the center of mass of the opposite face; thus, the median of the tetrahedron passes through the origin and through the point of intersection of medians of triangle faces).

Why is it so? Here is correct reasoning similar to the two-dimensional case: if we we cut the tetrahedron to the tetrahedron with a plane passing through the vertex of the tetrahedron and some kind of a median opposite faces, both of the resulting tetrahedron will have the same volume (because a triangular face is broken by a median into two triangles of equal area, and the height of the two tetrahedra is not changed). Repeating these arguments several times, we obtain that the center of mass lies at the point of intersection of the medians of the tetrahedron.

This point --- the point of intersection of medians of a tetrahedron --- it's called \bf{centroid}. It can be shown that it actually has coordinates equal to the average of the coordinates of the vertices of the tetrahedron:

$$ \vec{r_c} = \frac{ \vec{r_1} + \vec{r_2} + \vec{r_3} + \vec{r_4} }{ 4 }. $$

(this can be deduced from the fact that the centroid divides the medians in the ratio $1:3$)

Thus, between the cases of the tetrahedron and triangle there is no fundamental difference point equal to the average of the vertices is the center of mass of the two formulations of the problem: and when the mass is located only at the vertices, and when mass distributed over area/volume. Actually, this result is generalized to arbitrary dimension: the center of mass of an arbitrary \bf{simplex} (simplex) is the arithmetic mean of the coordinates of its vertices.

\h4{Case arbitrary polyhedron}

Let us proceed to the General case --- case of an arbitrary polyhedron.

Again, as in the two-dimensional case, we make the reduction of this task to have been solved: splitting a polyhedron to tetrahedrons (i.e. produce tetraacetate), find the center of mass of each of them, and get the final answer to the problem in the form of a weighted sum of the found centers of mass.

