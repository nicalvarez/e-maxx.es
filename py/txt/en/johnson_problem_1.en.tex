\h1{ Task Johnson with one machine }

It is the task of drawing up an optimal schedule of processing $n$ of parts on a single machine, if the $i$-th component is on it at time $t_i$, and $t$ seconds to wait before processing this part penalty is paid $f_i(t)$.

Thus, the task is to find such reordering of parts that the next value of (the fine) is minimal. If we denote by $\pi$ a permutation of the parts ($\pi_1$ --- the number of the first workpiece, $\pi_2$ --- second, etc.), the amount of the fine, $f(\pi)$ is equal to:

$$ F(\pi) = f_{\pi_1}(0) + f_{\pi_2}(t_{\pi_1}) + f_{\pi_3}(t_{\pi_1} + t_{\pi_2}) + \ldots + f_{\pi_n}\left(\sum_{i=1}^{n-1} t_{\pi_i}\right). $$

Sometimes this problem is called a single-processor to serve many applications.


\h2{ the solution of the problem in some particular cases }


\h3{ First special case: linear functions are fine }

Learn how to solve this problem in the case when all the $f_i(t)$ linear, i.e. are of the form:

$$ f_i(t) = c_i \cdot t, $$

where $c_i$ is-non - negative integers. Note that these linear functions free term equal to zero, because otherwise the answer immediately, you can add this free member, and to solve the problem with zero-free member.

Fix some schedule --- permutation $\pi$. Fix some number $i=1 \ldots n-1$, and let the permutation $\pi^\prime$ is $ the permutation $\pi$, which traded $i$-th and $i+1$-th elements. Let's see how much it has changed fine:

$$ F(\pi^\prime) - F(\pi) = $$

it is easy to understand that the changes occurred only with $i$-th and $i+1$-th components:

$$ = c_{\pi^\prime_i} \cdot \sum_{k=1}^{i-1} t_{\pi^\prime_k} + c_{\pi^\prime_{i+1}} \cdot \sum_{k=1}^{i} t_{\pi^\prime_k} - c_{\pi_i} \cdot \sum_{k=1}^{i-1} t_{\pi_k} - c_{\pi_{i+1}} \cdot \sum_{k=1}^{i} t_{\pi_k} = $$
$$ = c_{\pi_{i+1}} \cdot \sum_{k=1}^{i-1} t_{\pi^\prime_k} + c_{\pi_i} \cdot \sum_{k=1}^{i} t_{\pi_k^\prime} - c_{\pi_i} \cdot \sum_{k=1}^{i-1} t_{\pi_k} - c_{\pi_{i+1}} \cdot \sum_{k=1}^{i} t_{\pi_k} = $$
$$ = c_{\pi_i} \cdot t_{\pi_{i+1}} - c_{\pi_{i+1}} \cdot t_{\pi_i}. $$

It is clear that if the schedule is $\pi$ is optimal, any change leads to increase of penalty (or keep the previous value), so the optimal plan, you can record the condition:

$$ \forall i=1 \ldots n-1 ~~~:~~ c_{\pi_i} \cdot t_{\pi_{i+1}} - c_{\pi_{i+1}} \cdot t_{\pi_i} \ge 0. $$

Converting, we get:

$$ \forall i=1 \ldots n-1 ~~~:~~ \frac{ c_{\pi_i} }{ t_{\pi_i} } \ge \frac{ c_{\pi_{i+1}} }{ t_{\pi_{i+1}} }. $$

Thus, \bf{optimal schedule} can be obtained by simply \bf{sort} all items against $c_i$ to $t_i$ in the reverse order.

It should be noted that we obtained this algorithm is the so-called \bf{permutation technique}: we tried to swap two adjacent elements of the schedule, figured out how they became a penalty, and hence derived the algorithm for finding optimal schedules.


\h3{ Second special case: exponential functions are fine }

Suppose now that the penalty functions are of the form:

$$ f_i(t) = c_i \cdot e^{\alpha \cdot t}, $$

where all the numbers $c_i$ is non-negative, the constant $\alpha$ is positive.

Then, applying similarly the permutation here; it is easy to obtain that details should be sorted in descending order of quantities:

$$ v_i = \frac{ 1 - e^{ \alpha \cdot t_i } }{ c_i }. $$


\h3{ Third special case: the same monotone functions are fine }

In this case, it is believed that all $f_i(t)$ coincides with some function $\phi(t)$, which is increasing.

It is clear that in this case, the optimum positioning of the parts in the order of increasing processing time $t_i$.


\h2{ Theorem Livshits-Kladova }

Theorem Livshits-Kladova establishes that the permutation technique is applicable only for the above three special cases, and only them, i.e.:

\ul{
\li Linear case: $f_i(t) = c_i \cdot t + d_i$, where $c_i$ --- non-negative constants,
\li the Exponential case: $f_i(t) = c_i \cdot e^{\alpha \cdot t} + d_i$, where $c_i$ and $\alpha$ is a positive constant,
\li case is Identical: $f_i(t) = \phi(t)$, where $\phi$ --- increasing function.
}

This theorem is proved under the assumption that the penalty functions are smooth enough (there are third derivatives).

In all three cases is applicable permutation technique, whereby the desired optimal schedule can be found by a simple sort, therefore, within time $O (n \log n)$.

