\h1{Game Tag: the existence of a solution}

Recall that the game is a field of $4$ by $4$ are $15$ chips numbered from $1$ to $15$, and one field is left blank. Requires, moving at each step any chip on free position, to come eventually to the next position:
$$ \matrix{
1&2&3&4\cr
5&6&7&8\cr
9&10&11&12\cr
13&14&15&\bigcirc\cr
} $$

Tag ("15 puzzle") invented in 1880 the Chapman Noyes (Noyes Chapman).

\h2{Existence of solutions}

Here we consider the following problem: for a given position on the Board to say whether there is a sequence of moves leading to the solution or not.

Let us have some position on the Board:

$$ \matrix{
a_1 & a_2 & a_3 & a_4 \cr 
a_5 & a_6 & a_7 & a_8 \cr 
a_9 & a_{10} & a_{11} & a_{12} \cr 
a_{13} & a_{14} & a_{15} & a_{16} \cr 
} $$
where one of the elements is zero and denotes the empty cage, $a_z = 0$.

Consider a permutation:
$$ a_1 a_2 \ldots a_{z-1} a_{z+1} \ldots a_{15} a_{16} $$
(i.e., a permutation of numbers corresponding to positions on the Board, without the zero element)

Denote by $N$ the number of inversions in this permutation (i.e. the number of elements $a_i$ and $a_j$ that $i < j$ but $a_i > a_j$).

Next, let $K$ --- the line number in which the element is empty (i.e., in our notation, $K = (z-1)\ {\rm div}\ 4 + 1)$.

Then, \bf{solution exists only when $N+K$ is even}.

\h2{the Implementation}

Let us illustrate the above algorithm by using code:
\code
int a[16];
for (int i=0; i<16; ++i)
cin >> a[i];

int inv = 0;
for (int i=0; i<16; ++i)
if (a[i])
for (int j=0; j<i; ++j)
if (a[j] > a[i])
inv++;
for (int i=0; i<16; ++i)
if (a[i] == 0)
inv += 1 + i / 4;

puts ((inv & 1) ? "No Solution" : "Solution Exists");
\endcode

\h2{Proof}
Johnson (Johnson) in 1879 proved that if $N+K$ is odd, then there is no solution, and story (Story) in the same year proved that all positions for which $N+K$ is even, have solution.

However, both these proofs were quite complicated.

In 1999, Archer (Archer) proposed a much more simple proof (download his article \href=http://www.cs.cmu.edu/afs/cs/academic/class/15859-f01/www/notes/15-puzzle.the pdf{here}).
