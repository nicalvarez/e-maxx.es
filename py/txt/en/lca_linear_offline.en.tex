\h1{Least common ancestor. Finding $O(1)$ in the offline (the algorithm of Tarjan)}

Given a tree $G$ with $n$ vertices and given $m$ queries of the form $(a_i, b_i)$. For each request $(a_i, b_i)$ is required to find the least common ancestor of vertices $a_i$ and $b_i$, i.e. this node $c_i$, which is farthest from the root of the tree, and thus is an ancestor of both vertices $a_i$ and $b_i$.

We consider the problem in offline mode, i.e. considering that all requests are known in advance. Described below is an algorithm to answer $m$ of queries over the total time $O(n+m)$, i.e., for sufficiently large $m$ at $O(1)$ on the query.

\h2{the Algorithm of Tarjan}

The basis for the algorithm is the data structure \algohref=dsu{"DSU"}, which was invented by Tarjan (Tarjan).

The algorithm is essentially a depth-first-search from the root of the tree, which gradually are the answers to the queries. Namely, the answer to the query $(v,u)$ is found when the depth is at the vertex $u$, and vertex $v$ has already been visited, or Vice versa.

So, let the depth-first-search is the vertex $v$ (and already have been executed transitions in her sons), and found that for some of request $(v,u)$ the vertex $u$ has already been visited by the DFS. Learn then to find $\rm LCA$ these two vertices.

Note that ${\rm LCA}(v,u)$ is either the vertex $v$ or one of its ancestors. Then, we need to find the lowest top among the ancestors of $v$ (including herself), for which the vertex $u$ is a descendant. Note that for a fixed $v$ on such grounds (i.e. what is the lowest ancestor $v$ is an ancestor of some vertex) of vertices of the tree split into a set of disjoint classes. For each ancestor $p \not= v$ vertices $v$ the class itself contains this vertex and all subtrees with roots in those of her sons that are left from the path to $v$ (i.e., who were treated earlier than was achieved for $v$).

We need to learn to effectively support all of these classes, for which we apply the data structure "DSU". Each class will fit in this structure a lot, and for a representative of this set, we Dene the value $\rm ANCESTOR$ --- the vertex $p$, which forms this class.

Let's discuss the implementation of DFS. Let us assume that a vertex $v$. Place it in a separate class in the structure of disjoint sets, ${\rm ANCESTOR}[v] = v$. As usual in a depth-first, iterate over all outgoing edges $(v, to)$. For each such $to$ we first need to call the DFS from this vertex and add the vertex with all its subtree in the class of vertex $v$. It is implemented by the operation $\rm Union$ data structures "DSU", with the subsequent installation of ${\rm ANCESTOR} = v$ for the representative set (because after the merge, the representative of the class could change). Finally, after processing all edges we iterate over all requests of the form $(v,u)$, and if $u$ has been marked as visited by DFS, then the answer to this query will be the peak ${\rm LCA}(v,u) = {\rm ANCESTOR}[{\rm FindSet}(u)]$. It is easy to see that for each request this condition (which is one vertex of the query is current, and the other was previously visited) is executed exactly once.

Rate \bf{asymptotics}. It is composed of several parts. First, it is the asymptotic behavior of depth-first-search, which in this case is $O(n)$. Second, are the operations of Union of sets, which in the reasonable amount for all $n$ spend $O(n)$ operations. Thirdly, for each request the verification of conditions (twice in the query) and determining the outcome (once per query), each, again, all for a reasonable $n$ is $O(1)$. The complexity is obtained to $O(n+m)$, which means that for sufficiently large $m$ ($n = O(m)$) the answer for $O(1)$ per query.

\h2{the Implementation}

We present the complete implementation of this algorithm, including a slightly modified (with the support of $\rm ANCESTOR$) the implementation of a system of intersecting sets of (randomized version).

\code
const int MAXN = maximum number of vertices in the graph;
vector<int> g[MAXN], q[MAXN]; // count all requests
dsu int[MAXN], ancestor[MAXN];
bool u[MAXN];

dsu_get int (int v) {
return v == dsu[v] ? v : dsu[v] = dsu_get (dsu[v]);
}

void dsu_unite (int a, int b, int new_ancestor) {
a = dsu_get (a), b = dsu_get (b);
if (rand() & 1) swap (a, b);
dsu[a] = b, ancestor[b] = new_ancestor;
}

void dfs (int v) {
dsu[v] = v, ancestor[v] = v;
u[v] = true;
for (size_t i=0; i<g[v].size(); ++i)
if (!u[g[v][i]]) {
dfs (g[v][i]);
dsu_unite (v, g[v][i], v);
}
for (size_t i=0; i<q[v].size(); ++i)
if (u[q[v][i]]) {
printf ("%d %d -> %d\n", v+1, q[v][i]+1,
ancestor[ dsu_get(q[v][i]) ]+1);
}

int main() {
... reading graph ...

// read query
for (;;) {
int a, b = ...; // another request
--a, --b;
q[a].push_back (b);
q[b].push_back (a);
}

// depth-first-search and response to requests
dfs (0);
}
\endcode