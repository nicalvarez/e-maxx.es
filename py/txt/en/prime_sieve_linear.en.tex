\h1{ Sieve of Eratosthenes with linear time }

Given a number $n$. You want to find \bf{simple} in the interval $[2; n]$.

The classic way of solving this problem --- \bf{\algohref=eratosthenes_sieve{sieve of Eratosthenes}}. This algorithm is very simple, but works in time $O (n \log \log n)$.

Although at present there are known many algorithms working in sublinear time (i.e. $o(n)$), the following algorithm is interesting for its \bf{simplicity} --- it is practically not more complicated than the classic sieve of Eratosthenes.

In addition, the algorithm presented here as a "side effect" actually computes \bf{factorization of all numbers} in the interval $[2; n]$, which can be useful in many practical applications.

The disadvantage of the above algorithm is that it uses \bf{memory} than the classic sieve of Eratosthenes: you want to make an array of $n$ numbers, while the classic sieve of Eratosthenes need only $n$ bits of memory (that turns to $32$ times less).

Thus, the described algorithm has a sense only be applied to numbers of order $10^7$, not more.

The authorship of the algorithm, apparently, belongs to Grice and Misra (Gries, Misra, 1978 --- see list of references at the end). (And, strictly speaking, to call this algorithm "sieve of Eratosthenes" is incorrect: too difference between these two algorithms.)


\h2{ algorithm Description }

Our aim is to calculate for each number $i$ in the interval $[2; n]$ it \bf{minimal Prime divisor} $lp[i]$.

In addition, we will need to store the list of found Prime numbers --- let's call it array - $pr[]$.

Initially all values $lp[i]$ will fill with zeros, which means that we assume all numbers are simple. In the course of the algorithm this array will gradually fill.

We will now iterate over the current number $i$ from $2$ to $n$. We can have two cases:

\ul{

\li $lp[i] = 0$ --- this means that the number $i$ is a Prime number, because he never showed other factors.

Therefore, we must set the $lp[i] = i$, add $i$ to the list $pr[]$.

\li $lp[i] \ne 0$ --- this means that the current number $i$ --- compound, and the minimum divisor is a simple $lp[i]$.

}

In both cases, then begins the process \bf{hyphenation values} in the array $lp[]$: we will take the number, \bf{multiples} of $i$, and update their value to $lp[]$. However, our goal is to learn how to do it so that in the end of each number the value to $lp[]$ would be set to no more than once.

It is claimed that you can act accordingly. Consider numbers of the form:

$$ x_j = i \cdot integer, $$

where the sequence $integer$ --- this is all simple, not exceeding $lp[i]$ (just for this we need to store a list of all Prime numbers).

All numbers of this type put the new value, $lp[x_j]$ --- obviously, it will equal $integer$.

Why this algorithm is correct and why it runs in linear time --- see, now see its implementation.


\h2{ the Implementation }

The sieve is performed before specified in the constant number $N$.

\code
const int N = 10000000;
int lp[N+1];
vector<int> pr;

for (int i=2; i<=N; ++i) {
if (lp[i] == 0) {
lp[i] = i;
pr.push_back (i);
}
for (int j=0; j<(int)pr.size() && pr[j]<=lp[i] && i*pr[j]<=N; ++j)
lp[i * pr[j]] = pr[j];
}
\endcode

This implementation can speed up a bit by getting rid of the vector $pr$ (replacing it with a simple array with counter), as well as getting rid of the duplicate multiplication in the inner loop, $for$ (for which the result works you just have to remember a particular variable).


\h2{ Proof of correctness }

Let us prove \bf{correctness} of the algorithm, i.e. that it correctly puts all the values in $lp[]$, and each of them will be set exactly once. Here will follow that the algorithm works in linear time --- since all the other steps of the algorithm, obviously, work for $O (n)$.

To this end, note that any number $i$ \bf{unique view} of this type:

$$ i = lp[i] \cdot x, $$

where $lp[i]$ --- (as before) the minimum Prime divisor of $i$ and the number $x$ has no divisors smaller than $lp[i]$, i.e.:

$$ lp[i] \le lp[x]. $$

Now compare this with the fact that makes our algorithm --- it's actually for every $x$ goes through all of the simple which it can multiply, i.e. simple to $lp[x]$, inclusive, to obtain the numbers in the above presentation.

Therefore, the algorithm indeed will be held for each composite number exactly once, putting it the correct value, $lp[]$.

This means the correctness of the algorithm and that it runs in linear time.


\h2{ Time and memory required }

Although the asymptotic behavior of $O (n)$ asymptotics is better than $O (n \log \log n)$ of the classical sieve of Eratosthenes, the difference between them is small. In practice this means only two-fold difference in speed, and optimized versions of the sieve of Eratosthenes and does not lose listed here the algorithm.

Considering the cost of memory required by this algorithm --- the array $lp[]$ of length $n$ and an array of all simple $pr[]$ of length roughly $n / \ln n$ --- this algorithm seems inferior to the classical sieve on all counts.

However, it saves that array $lp[]$, computed by this algorithm allows to look for a factorization of any number in the interval $[2; n]$ for the order of the size of this factorization. Moreover, the price of one extra array, you can do that in this factorization does not require a division operation.

The knowledge of the factorization of all numbers --- very useful information for some tasks, and this algorithm is one of the few that allow you to look for it in linear time.


\h2{ Literature }

\ul{

\li David Gries, and Jayadev Misra. \bf{A Linear Sieve Algorithm for Finding Prime Numbers} [1978]

}