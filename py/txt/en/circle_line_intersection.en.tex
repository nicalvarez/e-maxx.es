<h1>the Intersection of the circle and straight</h1>

<p>Given a circle (coordinates of its center and radius) and the straight line (its equation). You want to find the point they intersect (one, two, or none).</p>
the <h2>Solution</h2>
<p>Instead of formal solutions of a system of two equations let us approach the task of <b>geometric side</b> (and, in this way we get a more exact solution from the point of view of numerical stability).</p>
<p>let us Assume without losing generality that the center of the circle is at the origin (if it isn't, move it there ^ it's suitable constant C in the equation of a line). I.e. have a circle with center in (0,0) of radius r and a straight line with equation Ax + By + C = 0.</p>
<p>First find <b>nearest to the center point</b> straight - point with some coordinates <b>(x<sub>0</sub>,y<sub>0</sub>)</b>. First, this point must be at a distance from the origin:</p>
<formula> |C|
----------
sqrt(A<sup>2</sup>+B<sup>2</sup>)</formula>
<p>secondly, since the vector (A,B) is perpendicular to a line, the coordinates of this point should be proportional to the coordinates of this vector. Given that the distance from the origin to the desired point we know, we just need to normalize the vector (A,B) to this length, and we get:</p>
<formula> A C
x<sub>0</sub> = - -----
A<sup>2</sup>+B<sup>2</sup>

B C
y<sub>0</sub> = - -----
A<sup>2</sup>+B<sup>2</sup></formula>
<p>(not evident here only the characters \'minus\', but these formulas are easy to check by substitution in the equation of a line - must be zero)</p>
<p>Knowing near to the circle center point, we can determine how many points will contain the answer, and even answer if these points 0 or 1.</p>
<p>Indeed, if the distance from (x<sub>0</sub>, y<sub>0</sub>) to the origin (but we already expressed in the formula - see above) is greater than the radius, then <b>the answer is zero points</b>. If this distance is equal to the radius of the <b>answer will be one point</b> - (x<sub>0</sub>,y<sub>0</sub>). But in the case of the remaining two points and their coordinates we have to find.</p>
<p>now, we know that the point (x<sub>0</sub>, y<sub>0</sub>) lies inside the circle. Desired point (ax,ay) and (bx,by), in addition to should belong to a straight line, must lie at the same distance d from the point (x<sub>0</sub>, y<sub>0</sub>), and this distance is easily found:</p>
<formula> C<sup>2</sup>
d = sqrt ( r<sup>2</sup> - ----- )
A<sup>2</sup>+B<sup>2</sup></formula>
<p>note that the vector (-B,A) are collinear straight line, and therefore the desired point (ax,ay) and (bx,by) can be obtained by adding to the point (x<sub>0</sub>,y<sub>0</sub>) of vector (-B,A), normalized to the length d (we get one required point), and subtracting the same vector (will get a second desired point).</p>
<p>the Final solution is:</p>
<formula> d<sup>2</sup>
mult = sqrt ( ----- )
A<sup>2</sup>+B<sup>2</sup>

ax = x<sub>0</sub> + B mult
ay = y<sub>0</sub> - A mult
bx = x<sub>0</sub> - B mult
by = y<sub>0</sub> + A mult</formula>
<p>If we solved this problem purely algebraically, we most likely would have received the solution in another form, which gives a large error. Therefore, the "geometric" method, described here, in addition to clarity, and is more accurate.</p>
the <h2>Implementation</h2>
<p>As stated at the beginning of the description, it is assumed that the circle is at the origin.</p>
<p>Therefore, the input parameters are the radius of the circle and the coefficients a,B,C the equation of the line.</p>
<code>double r, a, b, c; // input

double x0 = -a*c/(a*a+b*b), y0 = -b*c/(a*a+b*b);
if (c*c > r*r*(a*a+b*b)+EPS)
puts ("no points");
else if (abs (c*c - r*r*(a*a+b*b)) < EPS) {
puts ("1 point");
cout << x0 << \' \' << y0 << \'\n\';
}
else {
double d = r*r - c*c/(a*a+b*b);
double mult = sqrt (d / (a*a+b*b));
double ax,ay,bx,by;
ax = x0 + b * mult;
bx = x0 - b * mult;
ay = y0 - a * mult;
by = y0 + a * mult;
puts ("2 points");
cout << ax << \' \' << ay << \'\n\' << bx << \' \' << by << \'\n\';
}</code>