<h1>the BPSW test for primality of numbers</h1>

<hr>

the <h2>Introduction</h2>
<p>the BPSW Algorithm is the test number is Prime. This algorithm is named for the surnames of its inventors: Robert Bailey (Ballie), Carl Pomerance (Pomerance), John Selfridge (Selfridge), Samuel Wagstaff (Wagstaff). The algorithm was developed in 1980. To date, the algorithm does not find a counterexample, nor a proof was found.</p>
<p>the Algorithm BPSW has been verified for all numbers up to 10<sup>15</sup>. In addition, a counterexample was trying to find using the PRIMO program (see <a href="#6">[6]</a>), based on the test for primality using elliptic curves. The program after working for three years, has not found a single counterexample, based on what Martin suggested that there are no BPSW-pseudopostega less than 10<sup>10000</sup> (pseudopotto number - the integral number, on which the algorithm gives a result of "simple"). At the same time, Karl Pomeranz in 1984 presented a heuristic proof that there are infinitely many BPSW-pseudoplastic numbers.</p>
<p>the BPSW Complexity of the algorithm is O (log<sup>3</sup>(N)) bit operations. If we compare the BPSW algorithm with other tests, for example, a test of Miller-Rabin, the algorithm BPSW usually be 3-7 times slower.</p>
<p>the Algorithm is often used in practice. Apparently, many commercial mathematical packages, fully or partially, rely on the BPSW algorithm for checking Prime numbers:</p>

<hr>

the <h2>Brief description</h2>
<p>the Algorithm has a number of different implementations that differ from each other only in details. In our case, the algorithm looks like this:</p>
<p>1. To perform a test of Miller-Rabin base 2.</p>
<p>2. To perform a strong test of the Lucas-Selfridge, using Lucas sequences with the parameters of Selfridges.</p>
<p>3. Return "simple" only in the case when both tests returned "simple".</p>
<p>+0. In addition, at the beginning of the algorithm it is possible to add validation on trivial divisors of, say, 1000. This will increase the speed of the composite numbers, however, somewhat slowing down the algorithm.</p>
<p>so, the BPSW algorithm is based on the following:</p>
<p>1. (fact) test Miller-Rabin test and Lucas-Selfridge if they are wrong, in one direction only: some composite numbers with these algorithms is recognized as simple. In the opposite direction these algorithms are not mistaken.</p>
<p>2. (assumption) test Miller-Rabin test and Lucas-Selfridge if they are wrong, never make a mistake on one number at a time.</p>
<p>actually, the second assumption seems wrong - heuristic the proof is a refutation of Pomerania below. However, in practice a single pseudopostega still did not find, therefore can be considered the second assumption is true.</p>

<hr>

the <h2>Implementation of the algorithms in this article</h2>
<p>All the algorithms in this paper are implemented in C++. All programs are tested only on the compiler Microsoft C++ 8.0 SP1 (2005), should also compile on g++.</p>
<p>the Algorithms are implemented using templates (templates), which allows to use them as built-in numeric types, or your own classes that implement long arithmetics. [ so long long arithmetic in an article not included - TODO ]</p>
<p>In the article are only the most essential functions, the texts of the same helper functions can be downloaded in the Appendix to the article. Here are shown only the headers of these functions together with the comments:</p>
<code>//! Module 64-bit numbers
long long <b>abs</b> (long long n);
unsigned long long abs (unsigned long long n);

//! Returns true if n is even
template <class T>
bool <b>even</b> (const T & n);

//! Divides the number by 2
template <class T>
void <b>bisect</b> (T & n);

//! Multiplies the number by 2
template <class T>
void <b>redouble</b> (T & n);

//! Returns true if n is a perfect square of a Prime number
template <class T>
bool <b>perfect_square</b> (const T & n);

//! Calculates the root of a number, rounding it down
template <class T>
T <b>sq_root</b> (const T & n);

//! Returns the number of bits in the number
template <class T>
unsigned <b>bits_in_number</b> (T n);

//! Returns the value of the k-th bit number (bits are numbered starting from zero)
template <class T>
bool <b>test_bit</b> (const T & n, unsigned k);

//! Multiplies a *= b (mod n)
template <class T>
void <b>mulmod</b> (T & a, T b, const T & n);

//! Computes a^k (mod n)
template <class T, class T2>
T <b>powmod</b> (T a, T2 k, const T & n);

//! Translates the integer n in the form q*2^p
template <class T>
void <b>transform_num</b> (T n, T & p, T & q);

//! Euclidean Algorithm
template <class T, class T2>
T <b>gcd</b> (const T & a, const T2 & b);

//! Computes jacobi(a,b) is the Jacobi symbol
template <class T>
T <b>jacobi</b> (T a, T b)

//! Calculates pi(b) the first Prime numbers. Returns a vector with a simple pi - pi(b)
template <class T, class T2>
const std::vector<T> & <b>get_primes</b> (const T & b, T2 & pi);

//! A trivial check n for primality, we move all divisors of m up to.
//! Result: 1 if precisely n simple p - divisor it is found, 0 if unknown
template <class T, class T2>
T2 <b>prime_div_trivial</b> (const T & n, T2 m);</code>

<hr>

the <h2>a Test of Miller-Rabin</h2>
<p>I will not focus on the test of Miller-Rabin, as he is described in many sources, including and in Russian (for example. see <a href="#5">[5]</a>).</p>
<p>my only comment is that its speed is O (log<sup>3</sup>(N)) bit operations and will lead a ready-made implementation of this algorithm:</p>
<code>template <class T, class T2>
bool miller_rabin (n, T, T2 b)
{

// first check the trivial cases
if (n == 2)
return true;
if (n < 2 || even (n))
return false;

 // check that n and b are relatively Prime (otherwise it will fail)
// if they are not relatively Prime, then either n is not simple, either you have to increase b
if (b < 2)
b = 2;
for (T g; (g = gcd (n, b)) != 1; b++)
if (n > g)
return false;

// degradable n-1 = q*2^p
T n_1 = n;
--n_1;
T p, q;
transform_num (n_1, p, q);

// computed b^q mod n, if it is equal to 1 or n-1, then n is Prime (or pseudoproline)
T rem = powmod (T(b), q, n);
if (rem == 1 || rem == n_1)
return true;

// now calculated b^2q, b^4q, ... , b^((n-1)/2)
// if any of them is equal to n-1, then n is Prime (or pseudoproline)
for (T i=1; i<p; i++)
{
mulmod (rem, rem, n);
if (rem == n_1)
return true;
}

return false;

}</code>

<hr>

the <h2>a test of the Strong Lucas-Selfridge</h2>
<p>a Strong test of the Lucas-Selfridge consists of two parts: algorithm Selfridges to calculate a parameter, and a strong algorithm of Lucas, performed with this parameter.</p>
the <h3>Algorithm Selfridges</h3>
<p>Among the sequence 5, -7, 9, -11, 13, ... find the first number D for which J (D, N) = -1 and gcd (D, N) = 1, where J(x,y) is the Jacobi symbol.</p>
<p><b>Parameters Selfridges</b> will be P = 1 and Q = (1 - D) / 4.</p>
<p>note that the parameter Map does not exist for numbers that are exact squares. Indeed, if the number is an exact square, the Robin D you reach sqrt(N), which will prove that gcd (D, N) > 1, i.e., it will be found that N is composite.</p>
<p>in addition, the parameters of Selfridge will be calculated incorrectly for even numbers and for the unit; however, verification of these cases is not difficult.</p>
<p>Thus, <b>before the start of the algorithm</b> you should check that N is odd, 2 large, and is not an exact square, otherwise (if both same conditions) you need to immediately exit the algorithm with the result "composite".</p>
<p>Finally, note that if D for some number N is too large, the algorithm from the computational point of view won't be valid. Although in practice this has not been noticed (had been quite enough 4-byte number), however the probability of this event should not be ruled out. However, for example, on the interval [1; 10<sup>6</sup>] max(D) = 47, and on the interval [10<sup>19</sup>; 10<sup>19</sup>+10<sup>6</sup>] max(D) = 67. In addition, Bailey and Wagstaff in 1980 analytically proved this observation (see Ribenboim, 1995/96, p. 142).</p>
the <h3>is a Strong Lucas algorithm</h3>
<p><b>algorithm Parameters</b> Lucas are number <b>D, P and Q</b> such that D = P<sup>2</sup> - 4*Q ? 0, and P > 0.</p>
<p>(it is easy to see that the parameters calculated by the algorithm of Selfridges, satisfy these conditions)</p>
<p><b>Lucas Sequence</b> is a sequence U<sub>k</sub> and V<sub>k</sub>, defined as follows:</p>
<formula>U<sub>0</sub> = 0
U<sub>1</sub> = 1
<b>U<sub>k</sub> = U<sub>k-1</sub> Q U<sub>k-2</sub></b>
V<sub>0</sub> = 2
V<sub>1</sub> = P
<b>V<sub>k</sub> = P V<sub>k-1</sub> - Q V<sub>k-2</sub></b></formula>
<p>Next, let M = N - J (D, N).</p>
<p>If N is Prime, and gcd (N, Q) = 1, we have:</p>
<formula><b>U<sub>M</sub> = 0 (mod N)</b></formula>
<p>In particular, when the parameters D, P, Q computed by the algorithm of Selfridges, we have:</p>
<formula>U<sub>N+1</sub> = 0 (mod N)</b></formula>
<p>Return is, generally, wrong. However, pseudoplastic numbers with this algorithm is not very much, on what, actually, is based and the algorithm of Lucas.</p>
<p>so, <b>Lucas algorithm consists in calculating the U<sub>M</sub> and compared with zero</b>.</p>
<p>Next, you need to find some way of speeding up calculations U<sub>K</sub>, otherwise, clearly, no practical meaning in this algorithm.</p>
<p>Have:</p>
<formula>U<sub>k</sub> = (a<sup>k</sup> - b<sup>k</sup>) / (a - b),
V<sub>k</sub> = a<sup>k</sup> + b<sup>k</sup>,</formula>
<p>where a and b are the roots of the quadratic equation x<sup>2</sup> - P x + Q = 0.</p>
<p>Now the following equality can be proved elementarily:</p>
<formula>U<sub>2k</sub> = U<sub>k</sub> V<sub>k</sub> (mod N)
V<sub>2k</sub> = V<sub>k</sub><sup>2</sup> - Q 2<sup>k</sup> (mod N)</formula>
<p>Now, if M = E 2<sup>T</sup>, where E is an odd number, it is easy to obtain:</p>
<formula><b>U<sub>M</sub> = U<sub>E</sub> V<sub>E</sub> V<sub>2E</sub> V<sub>4</sub> ... V<sub>2<sup>T-2</sup>E</sub> V<sub>2<sup>T-1</sup>E</sub> = 0 (mod N)</b></formula>
<p>and at least one of the multipliers equal to zero modulo N.</p>
<p>it is Clear that <b>it is enough to calculate U<sub>E</sub> and V<sub>E</sub></b>, and all subsequent multipliers V<sub>2E</sub> V<sub>4</sub> ... V<sub>2<sup>T-2</sup>E</sub> V<sub>2<sup>T-1</sup>E</sub> can <b>be obtained of them</b>.</p>
<p>Thus, we have to learn how to quickly calculate U<sub>E</sub> and V<sub>E</sub> odd E.</p>
<p>First, consider the following formula for addition of members of Lucas sequences:</p>
<formula>U<sub>i+j</sub> = (U<sub>i</sub> V<sub>j</sub> + U<sub>j</sub> V<sub>i</sub>) / 2 (mod N)
V<sub>i+j</sub> = (V<sub>i</sub> V<sub>j</sub> + D U<sub>i</sub> U<sub>j</sub>) / 2 (mod N)</formula>
<p>Should pay attention that the division is performed in the field (mod N).</p>
<p>these Formulas are proved very simply, and here their proof is omitted.</p>
<p>Now, having the formulas for addition and doubling of members of Lucas sequences, and intuitive way of speeding up calculations U<sub>E</sub> and V<sub>E</sub>.</p>
<p>Indeed, consider the binary entry number E. Put first, the result is U<sub>E</sub> and V<sub>E</sub> is equal to, respectively, U<sub>1</sub> and V<sub>1</sub>. Iterate over all bits of E from more Junior to more senior, missing only the first bit (start of sequence). For each i-th bit we compute U<sub>2<sup> i</sup></sub> and V<sub>2<sup> i</sup></sub> from previous members by using the formula of doubling. In addition, if the current i-th bit equal to one, the answer will be to add the current U<sub>2<sup> i</sup></sub> and V<sub>2<sup> i</sup></sub> using the formulas of addition. At the end of the algorithm, running in O (log(E)), we <b>get the desired U<sub>E</sub> and V<sub>E</sub></b>.</p>
<p>If U<sub>E</sub> or V<sub>E</sub> was equal to zero (mod N), then the number N is Prime (or pseudoproline). If they are both different from zero, we calculated V<sub>2E</sub>, V<sub>4</sub> ... V<sub>2<sup>T-2</sup>E</sub>, V<sub>2<sup>T-1</sup>E</sub>. If at least one of them is comparable to zero modulo N, the number N is Prime (or pseudoproline). Otherwise the number N is composite.</p>
the <h3>discussion of the algorithm Selfridges</h3>
<p>Now that we have considered the algorithm of Lucas, it is possible to elaborate on the parameters D,P,Q, one way of obtaining which is the algorithm of Selfridges.</p>
<p>Recall basic requirements:</p>
<formula><b>P > 0</b>,
<b>D = P<sup>2</sup> - 4*Q ? 0</b>.</formula>
<p>Now let's continue exploring these options.</p>
<p><b>D should be an exact square (mod N)</b>.</p>
<p>Indeed, otherwise we get:</p>
<p>D = b<sup>2</sup>, hence J(D,N) = 1, P = b + 2, Q = b + 1, hence U<sub>n-1</sub> = (Q<sup>n-1</sup> - 1) / (Q - 1).</p>
<p>that is, if D is a perfect square, the algorithm of Lucas is becoming almost a normal probability test.</p>
<p>One of the best ways to avoid such <b>to require that J(D,N) = -1</b>.</p>
<p>for Example, you can choose the first number from the sequence D 5, -7, 9, -11, 13, ..., for which J(D,N) = -1. Also let P = 1. Then Q = (1 - D) / 4. This method was proposed by Selfridge.</p>
<p>However, there are other ways to choose D. you Can select it from the sequence 5, 9, 13, 17, 21, ... Also let P be the least odd, prevoshodnaya sqrt(D). Then Q = (P<sup>2</sup> - D) / 4.</p>
<p>it is Clear that the choice of a particular method of calculating the parameters of Lucas and depends on its result - pseudopodia may differ at various ways of selecting. As shown, the algorithm proposed by Selfridge, was very successful: all pseudopodia Lucas-Selfridge are not pseudoproline Miller-Rabin, at least no counterexample was found.</p>
the <h3>Implementation of the algorithm a strong Lucas-Selfridge</h3>
<p>Now you only have to implement the algorithm:</p>
<code>template <class T, class T2>
lucas_selfridge bool (const T & n, T2 unused)
{

// first check the trivial cases
if (n == 2)
return true;
if (n < 2 || even (n))
return false;

// check that n is not an exact square, otherwise the algorithm will give an error
if (perfect_square (n))
return false;

// the Selfridge algorithm: find the first number d such that:
// jacobi(d,n)=-1 and it belongs to the range { 5,-7,9,-11,13,... }
T2 dd;
for (T2 d_abs = 5, d_sign = 1; ; d_sign = -d_sign, ++++ d_abs)
{
dd = d_abs * d_sign;
T g = gcd (n, d_abs);
if (1 < g && g < n)
// find divisor - d_abs
return false;
if (jacobi (T(dd), n) == -1)
break;
}

// parameters of Selfridge
T2
p = 1,
q = (p*p - dd) / 4;

// degradable n+1 = d*2^s
T n_1 = n;
n_1++;
T s, d;
transform_num (n_1, s, d);

// the algorithm of Lucas
T
u = 1,
v = p,
u2m = 1,
v2m = p,
qm = q,
qm2 = q*2,
qkd = q;
for (unsigned bit = 1, bits = bits_in_number(d); bit < bits; bit++)
{
mulmod (u2m, v2m, n);
mulmod (v2m, v2m, n);
while (v2m < qm2)
v2m += n;
v2m -= qm2;
mulmod (qm, qm, n);
qm2 = qm;
redouble (qm2);
if (test_bit (d, bit))
{
T t1, t2;
t1 = u2m;
mulmod (t1, v, n);
t2 = v2m;
mulmod (t2, u, n);

T t3, t4;
t3 = v2m;
mulmod (t3, v, n);
t4 = u2m;
mulmod (t4, u, n);
mulmod (t4, (T)dd, n);

u = t1 + t2;
if (!even (u))
u += n;
bisect (u);
u %= n;

v = t3 + t4;
if (!even (v))
v += n;
bisect (v);
v %= n;
mulmod (qkd, qm, n);
}
}

// exactly easy (or pseudo-simple)
if (u == 0 || v == 0)
return true;

// deviceset the remaining members
Qkd2 T = qkd;
redouble (qkd2);
for (T2 r = 1; r < s; ++r)
{
mulmod (v, v, n);
v= qkd2;
if (v < 0) v += n;
if (v < 0) v += n;
if (v >= n) v -= n;
if (v >= n) v -= n;
if (v == 0)
return true;
if (r < s-1)
{
mulmod (qkd, qkd, n);
qkd2 = qkd;
redouble (qkd2);
}
}

return false;

}</code>

<hr>

the <h2>Code BPSW</h2>
<p>it Now remains just to combine the results of all 3 tests: test a small trivial divisors, test Miller-Rabin test is a strong Lucas-Selfridge.</p>
<code>template <class T>
bool baillie_pomerance_selfridge_wagstaff (T n)
{

// first check for trivial divisors - for example, to 29
int div = prime_div_trivial (n, 29);
if (div == 1)
return true;
if (div > 1)
return false;

// test of Miller-Rabin base 2
if (!miller_rabin (n, 2))
return false;

// test the strong Lucas-Selfridge
return lucas_selfridge (n, 0);

}</code>
<p><a href=BPSW_main.zip>from Here</a> you can download the program (source + exe) that contains a full implementation of BPSW test. [77 KB]</p>

<hr>

the <h2>brief implementation</h2>
<p>the code Length can be reduced considerably at the expense of universality, abandoning the templates and various helper functions.</p>
<code>const int trivial_limit = 50;
int p[1000];

int gcd (int a, int b) {
return a ? gcd (b%a, a) : b;
}

int powmod (int a, int b, int m) {
int res = 1;
while (b)
if (b & 1)
res = (res * 1ll * a) % m, --b;
else
a = (a * 1ll * a) % m, b >>= 1;
return res;
}

bool miller_rabin (int n) {
int b = 2;
for (int g; (g = gcd (n, b)) != 1; b++)
if (n > g)
return false;
int p=0, q=n-1;
while ((q & 1) == 0)
p++, q >>= 1;
int rem = powmod (b, q, n);
if (rem == 1 || rem == n-1)
return true;
for (int i=1; i<p; ++i) {
rem = (rem * 1ll * rem) % n;
if (rem == n-1) return true;
}
return false;
}

int jacobi (int a, int b)
{
if (a == 0) return 0;
if (a == 1) return 1;
if (a < 0)
if ((b & 2) == 0)
return jacobi (-a, b);
else
return - jacobi (-a, b);
int a1=a, e=0;
while ((a1 & 1) == 0)
a1 >>= 1, ++e;
int s;
if ((e & 1) == 0 || (b & 7) == 1 || (b & 7) == 7)
s = 1;
else
s = -1;
if ((b & 3) == 3 && (a1 & 3) == 3)
s = -s;
if (a1 == 1)
return s;
return s * jacobi (b % a1, a1);
}

bool bpsw (int n) {
if ((int)sqrt(n+0.0) * (int)sqrt(n+0.0) == n) return false;
int dd=5;
for (;;) {
int g = gcd (n, abs(dd));
if (1<g && g<n) return false;
if (jacobi (dd, n) == -1) break;
dd = dd<0 ? -dd+2 : -dd-2;
}
int p=1, q=(p*p-dd)/4;
int d=n+1, s=0;
while ((d & 1) == 0)
s++, d>>=1;
long long u=1, v=p, u2m=1, v2m=p, qm=q, qm2=q*2, qkd=q;
for (int mask=2; mask<=d; mask<<=1) {
u2m = (u2m * v2m) % n;
v2m = (v2m * v2m) % n;
while (v2m < qm2) v2m += n;
v2m -= qm2;
qm = (qm * qm) % n;
qm2 = qm * 2;
if (d & mask) {
long long t1 = (u2m * v) % n, t2 = (v2m * u) % n,
t3 = (v2m * v) % n, t4 = (((u2m * u) % n) * dd) % n;
u = t1 + t2;
if (u & 1) u += n;
u = (u >> 1) % n;
v = t3 + t4;
if (v & 1) v += n;
v = (v >> 1) % n;
qkd = (qkd * qm) % n;
}
}
if (u==0 || v==0) return true;
long long qkd2 = qkd*2;
for (int r=1; r<s; ++r) {
v = (v * v) % n - qkd2;
if (v < 0) v += n;
if (v < 0) v += n;
if (v >= n) v -= n;
if (v >= n) v -= n;
if (v == 0) return true;
if (r < s-1) {
qkd = (qkd * 1ll * qkd) % n;
qkd2 = qkd * 2;
}
}
return false;
}

bool prime (int n) { // this function need to call to check on simplicity
for (int i=0; i<trivial_limit && p[i]<n; ++i)
if (n % p[i] == 0)
return false;
if (p[trivial_limit-1]*p[trivial_limit-1] >= n)
return true;
if (!miller_rabin (n))
return false;
bpsw return (n);
}

void prime_init() { // to call before first call to prime() !
for (int i=2, j=0; j<trivial_limit; ++i) {
bool pr = true;
for (int k=2; k*k<=i; ++k)
if (i % k == 0)
pr = false;
if (pr)
p[j++] = i;
}
}</code>

<hr>

the <h2>Heuristic proof is a refutation of Pomerania</h2>
<p>Pomeranz in 1984 proposed the following heuristic proof.</p>
<p>Claim: <b>the Number of BPSW-pseudopath from 1 to X is greater than X<sup>1</sup> for any a > 0</b>.</p>
<p>Proof.</p>
<p>Let k > 4 be an arbitrary but fixed number. Let T be some large number.</p>
<p>Let P<sub>k</sub>(T) be the set of simple p in the interval [T; T<sup>k</sup>],:</p>
<p>(1) p = 3 (mod 8), J(5,p) = -1</p>
<p>(2) the number (p-1)/2 is not an exact square</p>
<p>(3) the number (p-1)/2 is made up exclusively of simple q < T
<p>(4) the number (p-1)/2 is made up exclusively of simple q such that q = 1 (mod 4)</p>
<p>(5) the number (p+1)/4 is not an exact square</p>
<p>(6) the number (p+1)/4 is composed exclusively of simple d < T</p>
<p>(7) the number (p+1)/4 is composed exclusively of simple d such that q = 3 (mod 4)</p>
<p>it is Clear that about 1/8 of all the primes in the interval [T; T<sup>k</sup>] satises (1). You can also show that the conditions (2) and (5) retain some part numbers. Heuristically, condition (3) and (6) also allows us to leave some part numbers from the interval (T; T<sup>k</sup>). Finally, event (4) has probability (c (log T)<sup>-1/2</sup>) and event (7). Thus, the cardinality of P<sub>k</sub>(T) priblizitelno equal when T -> oo</p>
<p><img src=BPSW_formula1.jpg></p>
<p>where c is some positive constant depending on the choice of k.</p>
<p>Now we <b>can build a number n</b>, is an exact square, composed of a simple l from P<sub>k</sub>(T), where l is odd and less than T<sup>2</sup> / log(T<sup>k</sup>). The number of ways to choose such a number n is approximately</p>
<p><img src=BPSW_formula2.jpg></p>
<p>for a large T and a fixed k. In addition, each such number n is less than e<sup>T<sup>2</sup></sup>.</p>
<p>we denote by Q<sub>1</sub> product of simple q < T for which q = 1 (mod 4) and Q<sub>3</sub> is the product of simple q < T for which q = 3 (mod 4). Then gcd (Q<sub>1</sub>, Q<sub>3</sub>) = 1 and Q<sub>1</sub> Q<sub>3</sub> ? e<sup>T</sup>. Thus, the number of ways to choose n <b>additional terms</b></p>
<formula>n = 1 (mod Q<sub>1</sub>), n = -1 (mod Q<sub>3</sub>)</formula>
<p>should be, heuristically at least</p>
the <formula>e<sup>T<sup> 2</sup> (1 - 3 / k)</sup> / e<sup> 2T</sup> > <b>e<sup>T<sup> 2</sup> (1 - 4 / k)</sup></b></formula>
<p>for large T.</p>
<p>But <b>every n is a counterexample to the BPSW test</b>. Indeed, n is a Carmichael number (i.e. the number on which the test of Miller-Rabin will make a mistake at any base), so it will be automatically pseudoproblem base 2. Since n = 3 (mod 8) and each p | n is equal to 3 (mod 8), it is clear that n will also be strong pseudoproblem base 2. Since J(5,n) = -1, then every Prime p | n satises J(5,p) = -1, and since p+1 | n+1 for any Prime p | n, it follows that n - pseudopotto for any Lucas test Lucas with discriminant 5.</p>
<p>Thus, we have shown that for any fixed k and all large T, there will be at least e<sup>T<sup> 2</sup> (1 - 4 / k)</sup> of counterexamples to the test BPSW among numbers less than e<sup>T<sup> 2</sup></sup>. Now, if we put x = e<sup>T<sup> 2</sup></sup>, there will be at least x<sup>1 - 4 / k</sup> of counterexamples smaller than x. Since k is a random number, our evidence indicates that <b>number of counterexamples smaller than x, there is a number greater than x<sup>1</sup> for any a > 0</b>.</p>

<hr>

the <h2>Practical test BPSW test</h2>
<p>In this section, we discuss results obtained by me in testing my implementation BPSW test. All tests were performed on the embedded type of 64-bit long long integer. Long arithmetic has not been tested.</p>
<p>Tests were conducted on a Celeron 1.3 GHz.</p>
<p>All times are given in <b>microseconds</b> (10<sup> -6</sup> h).</p>

the <h3>Average time of work on the segment numbers, depending on the limit of the trivial brute-force</h3>
<p>this refers to the parameter passed to the function prime_div_trivial () in the code above is $ 29.</p>
<p><a href="BPSW_test_1.zip">Download</a> test program (source and exe). [83 KB]</p>
<p>If you run the test <b>all odd numbers</b> of cut, the results will be such:</p>
<table class=table2 cellspacing=0>
<tr><th>origin<br>cut</th><th>end<br>cut</th><th>limit>; <br> - Robin ></th><th width=13%>0</th><th width=13%>10<sup>2</sup></th><th width=13%>10<sup>3</sup></th><th width=13%>10<sup>4</sup></th><th width=13%>10<sup>5</sup></th></tr>
<tr><td>1</td><td>10<sup>5</sup></td><td></td><td>8.1</td><td>4.5</td><td>0.7</td><td>0.7</td><td>0.9</td></tr>
<tr><td>10<sup>6</sup></td><td>10<sup>6</sup>+10<sup>5</sup></td><td></td><td>12.8</td><td>6.8</td><td>7.0</td><td>1.6</td><td>1.6</td>
<tr><td>10<sup>9</sup></td><td>10<sup>9</sup>+10<sup>5</sup></td><td></td><td>28.4</td><td>12.6</td><td>12.1</td><td>17.0</td><td>17.1</td></tr>
<tr><td>10<sup>12</sup></td><td>10<sup>12</sup>+10<sup>5</sup></td><td></td><td>41.5</td><td>16.5</td><td>15.3</td><td>19.4</td><td>54.4</td></tr>
<tr><td>10<sup>15</sup></td><td>10<sup>15</sup>+10<sup>5</sup></td><td></td><td>66.7</td><td>24.4</td><td>21.1</td><td>24.8</td><td>58.9</td></tr>
</table>
<p>If you run the test <b>only on the Prime numbers</b> of cut, the speed of work is as follows:</p>
<table class=table2 cellspacing=0>
<tr><th>origin<br>cut</th><th>end<br>cut</th><th>limit>; <br> - Robin ></th><th width=13%>0</th><th width=13%>10<sup>2</sup></th><th width=13%>10<sup>3</sup></th><th width=13%>10<sup>4</sup></th><th width=13%>10<sup>5</sup></th></tr>
<tr><td>1</td><td>10<sup>5</sup></td><td></td><td>42.9</td><td>40.8</td><td>3.1</td><td>4.2</td><td>4.2</td></tr>
<tr><td>10<sup>6</sup></td><td>10<sup>6</sup>+10<sup>5</sup></td><td></td><td>75.0</td><td>76.4</td><td>88.8</td><td>13.9</td><td>15.2</td>
<tr><td>10<sup>9</sup></td><td>10<sup>9</sup>+10<sup>5</sup></td><td></td><td>186.5</td><td>188.5</td><td>201.0</td><td>294.3</td><td>283.9</td></tr>
<tr><td>10<sup>12</sup></td><td>10<sup>12</sup>+10<sup>5</sup></td><td></td><td>288.3</td><td>288.3</td><td>302.2</td><td>387.9</td><td>1069.5</td></tr>
<tr><td>10<sup>15</sup></td><td>10<sup>15</sup>+10<sup>5</sup></td><td></td><td>485.6</td><td>489.1</td><td>496.3</td><td>585.4</td><td>1267.4</td></tr>
</table>
<p>Thus, optimally choose to <b>limit trivial brute-force is equal to 100 or 1000</b>.</p>
<p>For all these tests I chose a limit of 1000.</p>

the <h3>Average battery life on an interval of numbers</h3>
<p>Now that we have chosen the limit of the trivial brute-force, can more accurately test the speed at different intervals.</p>
<p><a href=BPSW_test2.zip>Download</a> test program (source and exe). [83 KB]</p>
<table class=table1 cellspacing=0>
<tr><th width=100>start<br>cut</th><th width=100>end<br>cut</th><th width=200>time<br>odd numbers</th><th width=200>time<br>on Prime numbers</th></tr>
<tr><td>1</td><td>10<sup>5</sup></td><td>1.2</td><td>4.2</td></tr>
<tr><td>10<sup>6</sup></td><td>10<sup>6</sup>+10<sup>5</sup></td><td>13.8</td><td>88.8</td></tr>
<tr><td>10<sup>7</sup></td><td>10<sup>7</sup>+10<sup>5</sup></td><td>16.8</td><td>115.5</td></tr>
<tr><td>10<sup>8</sup></td><td>10<sup>8</sup>+10<sup>5</sup></td><td>21.2</td><td>164.8</td></tr>
<tr><td>10<sup>9</sup></td><td>10<sup>9</sup>+10<sup>5</sup></td><td>24.0</td><td>201.0</td></tr>
<tr><td>10<sup>10</sup></td><td>10<sup>10</sup>+10<sup>5</sup></td><td>25.2</td><td>225.5</td></tr>
<tr><td>10<sup>11</sup></td><td>10<sup>11</sup>+10<sup>5</sup></td><td>28.4</td><td>266.5</td></tr>
<tr><td>10<sup>12</sup></td><td>10<sup>12</sup>+10<sup>5</sup></td><td>30.4</td><td>302.2</td></tr>
<tr><td>10<sup>13</sup></td><td>10<sup>13</sup>+10<sup>5</sup></td><td>33.0</td><td>352.2</td></tr>
<tr><td>10<sup>14</sup></td><td>10<sup>14</sup>+10<sup>5</sup></td><td>37.5</td><td>424.3</td></tr>
<tr><td>10<sup>15</sup></td><td>10<sup>15</sup>+10<sup>5</sup></td><td>42.3</td><td>499.8</td></tr>
<tr><td>10<sup>16</sup></td><td>10<sup>15</sup>+10<sup>5</sup></td><td>46.5</td><td>553.6</td></tr>
<tr><td>10<sup>17</sup></td><td>10<sup>15</sup>+10<sup>5</sup></td><td>48.9</td><td>621.1</td></tr>
</table>
<p>Or, graphically, the approximate time the BPSW test is run on one number:</p>
<p><img src=BPSW_graph1.gif></p>
<p>so we've got that, in practice, for small numbers (up to 10<sup>17</sup>), <b>the algorithm works in O (log N)</b>. This is because the built-in type for int64 division operation is performed in O(1), i.e. the complexity of the division zavisisit from the number of bits in the number.</p>
<p>If you use the BPSW test for long arithmetic, it is expected that it will run in time O (log<sup>3</sup>(N)). [ TODO ]</p>

<hr>

the <h2>Application. All programs</h2>
<p><a href=BPSW_all.zip>Download</a> each of the programs from this article. [242 KB]</p>

<hr>

the <h2>Literature</h2>
<p>Used by me literature fully available online:</p>
<ol>
the <li>Robert Baillie; Samuel S. Wagstaff<br><b>Lucas pseudoprimes</b><br>Math. Comp. 35 (1980) 1391-1417<br><a href="http://mpqs.free.fr/LucasPseudoprimes.pdf">mpqs.free.fr/LucasPseudoprimes.pdf</a><br> </li>
the <li>Daniel J. Bernstein<br><b>Distinguishing prime numbers from composite numbers: the state of the art in 2004</b><br>Math. Comp. (2004)<br><a href="http://cr.yp.to/primetests/prime2004-20041223.pdf">cr.yp.to/primetests/prime2004-20041223.pdf</a><br> </li>
the <li>Richard P. Brent<br><b>Primality Testing and Integer Factorisation</b><br>The Role of Mathematics in Science (1990)<br><a href="http://wwwmaths.anu.edu.au/~brent/pd/rpb120.pdf">wwwmaths.anu.edu.au/~brent/pd/rpb120.pdf</a><br> </li>
the <li>H. Cohen; H. W. Lenstra<br><b>Primality Testing and Jacobi Sums</b><br>Amsterdam (1984)<br><a href="https://www.openaccess.leidenuniv.nl/bitstream/1887/2136/1/346_065.pdf">www.openaccess.leidenuniv.nl/bitstream/1887/2136/1/346_065.pdf</a><br> </li>
the <li><a name=5></a>Thomas H. Cormen; Charles E. Leiserson; Ronald L. Rivest<br><b>Introduction to Algorithms</b><br>[ links ]<br>The MIT Press (2001)<br> </li>
the <li><a name=6></a>M. Martin<br><b>PRIMO - Primality Proving</b><br><a href="http://www.ellipsa.net/">www.ellipsa.net</a><br> </li>
the <li>F. Morain<br><b>Elliptic curves and primality proving</b><br>Math. Comp. 61(203) (1993)<br><a href="http://citeseer.ist.psu.edu/rd/43190198%2C72628%2C1%2C0.25%2CDownload/ftp%3AqSqqSqftp.inria.frqSqINRIAqSqpublicationqSqpubli-ps-gzqSqRRqSqRR-1256.ps.gz">citeseer.ist.psu.edu/rd/43190198%2C72628%2C1%2C0.25%2CDownload/ftp%3AqSqqSqftp.inria.frqSqINRIAqSqpublicationqSqpubli-ps-gzqSqRRqSqRR-1256.ps.gz</a><br> </li>
the <li>Carl Pomerance<br><b>Are there counter-examples to the Baillie-PSW primality test?</b><br>Math. Comp. (1984)<br><a href="http://www.pseudoprime.com/dopo.pdf">www.pseudoprime.com/dopo.pdf</a><br> </li>
the <li>Eric W. Weisstein<br><b>Baillie-PSW primality test</b><br>MathWorld (2005)<br><a href="http://mathworld.wolfram.com/Baillie-PSWPrimalityTest.html">mathworld.wolfram.com/Baillie-PSWPrimalityTest.html</a><br> </li>
the <li>Eric W. Weisstein<br><b>Strong Lucas pseudoprime</b><br>MathWorld (2005)<br><a href="http://mathworld.wolfram.com/StrongLucasPseudoprime.html">mathworld.wolfram.com/StrongLucasPseudoprime.html</a><br> </li>
the <li>Paulo Ribenboim<br><b>The Book of Prime Number Records</b><br>Springer-Verlag (1989)<br>[ links ]<br> </li>
</ol>
<p>a List of other recommended books, which I cannot find on the Internet:</p>
<ol start=12>
the <li>Zhaiyu Mo; James P. Jones<br><b>A new primality test using Lucas sequences</b><br>Preprint (1997)<br> </li>
the <li>Hans Riesel<br><b>Prime numbers and computer methods for factorization</b><br>Boston: Birkhauser (1994)<br> </li>
</ol>