\h1{ Function Euler }


\h2{ Definition }

\bf{Function Euler} $\phi (n)$ (sometimes denoted $\varphi(n)$ or ${\it phi}(n)$) is the number of integers from $1$ to $n$, coprime with $n$. In other words, the number of integers in the interval $[1; n]$, \algohref=euclid_algorithm{greatest common divisor} with $n$ equal to one.

The first few values of this function (\href=http://oeis.org/A000010{encyclopedia A000010 in OEIS}):

$$ \phi (1)=1, $$
$$ \phi (2)=1, $$
$$ \phi (3)=2, $$
$$ \phi (4)=2, $$
$$ \phi (5)=4. $$


\h2{ Properties }

The following three simple properties of the Euler function --- sufficient to learn how to calculate it for any numbers:

\ul{

\li If $p$ is a Prime number, then $\phi (p)=p-1$.

(This is obvious, since any number except $p$, mutually just with him.)

\li If $p$ is a Prime number, $a$ --- natural number, then $\phi (p^a)=p^a-p^{a-1}$.

(Because with the number $p^a$ are not relatively Prime numbers of the form $pk$ $(k \in \mathcal{N})$, which $p^a / p = p^{a-1}$ units.)

\li If $a$ and $b$ are relatively Prime, then $\phi(ab) = \phi(a) \phi(b)$ ("multiplicative" of the Euler function).

(This fact follows from \algohref=chinese_theorem{Chinese theorem on residues}. Let us consider an arbitrary number $z \le ab$. Denote by $x$ and $y$ remains from dividing $z$ on $a$ and $b$ respectively. Then $z$ are mutually simple with $ab$ if and only when $z$ are mutually simple with $a$ and $b$ separately, or, equivalently, $x$ are mutually simple with $a$ and $y$ are mutually simple with $b$. Using Chinese theorem on residues, we obtain that any pair of numbers $x$ and $y$ $(x \le a, ~ y \le b)$ one-to-one corresponds to the number of $z$ $(z \le ab)$, which completes the proof.)

}

From here you can obtain the Euler function for any $\it n$ through \bf{factorization} (decomposition of $n$ into Prime factors):

if

$$ n = p_1^{a_1} \cdot p_2^{a_2} \cdot \ldots \cdot p_k^{a_k} $$

(where all the $p_i$ --- simple)

$$ \phi(n) = \phi(p_1^{a_1}) \cdot \phi(p_2^{a_2}) \cdot \ldots \cdot \phi(p_k^{a_k}) = $$
$$ = (p_1^{a_1} - p_1^{a_1-1}) \cdot (p_2^{a_2} - p_2^{a_2-1}) \cdot \ldots \cdot (p_k^{a_k} - p_k^{a_k-1}) = $$
$$ = n \cdot \left( 1-{1\over p_1} \right) \cdot \left( 1-{1\over p_2} \right) \cdot \ldots \cdot \left( 1-{1\over p_k} \right). $$


\h2{ the Implementation }

Simple code that computes the Euler function, the number of elementary factorize method for $O (\sqrt n)$:

\code
int phi (int n) {
int result = n;
for (int i=2; i*i<=n; ++i)
if (n % i == 0) {
while (n % i == 0)
n /= i;
result -= result / i;
}
if (n > 1)
result -= result / n;
return result;
}
\endcode

A key role for the computation of the Euler function is being \bf{factorization} of $n$. It can be implemented over time, a significantly smaller $O(\sqrt{n})$: see \algohref=factorization{Efficient algorithms for factorization}.


\h2{ the Application of the Euler function }

The most famous and important property of the Euler function is expressed in \bf{Euler's theorem}:
$$ a^{\phi(m)} \equiv 1 \pmod m, $$
where $\it$ a $ and $\it m$ are coprime.

In the particular case when $\it m$ is simple, Euler's theorem turns into the so-called \bf{small Fermat's theorem}:
$$ a^{m-1} \equiv 1 \pmod m $$

The Euler theorem often occurs in practical applications, see, for example, \algohref=reverse_element{Return item in the module}.


\h2{ Problem in online judges }

The task list in which you want to calculate the Euler function,or use Euler's theorem or the value of the Euler function to restore the original number:

\ul{

\li \href=http://uva.onlinejudge.org/index.php?option=onlinejudge&page=show_problem&problem=1120{UVA #10179 \bf{"Irreducible Basic Fractions"} ~~~~ [difficulty: easy]}

\li \href=http://uva.onlinejudge.org/index.php?option=onlinejudge&page=show_problem&problem=1240{UVA #10299 \bf{"Relatives"} ~~~~ [difficulty: easy]}

\li \href=http://uva.onlinejudge.org/index.php?option=com_onlinejudge&Itemid=8&page=show_problem&problem=2302{UVA #11327 \bf{"Enumerating Rational Numbers"} ~~~~ [difficulty: medium]}

\li \href=http://acm.timus.ru/problem.aspx?space=1&num=1673{TIMUS #1673 \bf{"Permit to test"} ~~~~ [difficulty: high]}

}