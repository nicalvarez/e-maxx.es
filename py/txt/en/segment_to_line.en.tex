\h1{ finding the equation of the line to cut }

Task --- given the coordinates of the endpoint of the line segment we draw a straight line passing through it.

We believe that the cut is non-degenerate, i.e. has a length greater than zero (otherwise, of course, passes through infinitely many different straight).


\h2{ the two-Dimensional case }

Let the given line segment $PQ$, i.e. the known coordinates of its ends, $P_x$, $P_y$ and $Q_x$, $Q_y$.

You want to build \bf{the equation of a line in the plane}, passing through this cut, i.e. to find coefficients $A$, $B$, $C$ in the equation of a line:

$$ A x + B y + C = 0. $$

Note that the desired triples $(A,B,C)$ that passes through the given segment, \bf{infinitely many}: you can multiply all three ratios to an arbitrary nonzero number and get the same line. Therefore, our task is to find one of these triples.

It is easy to verify (by substitution of these expressions and the coordinates of points $P$ and $Q$ in the equation of a line) that fits the following set of coefficients:

$$ A = P_y - Q_y, $$
$$ B = Q_x - P_x, $$
$$ C = - A P_x - P_y B. $$


\h3{ case Integer }

An important advantage of this method of construction direct is that if all coordinates are integers, then the resulting coefficients will also be \bf{integer}. In some cases, it allows geometric operations, generally without resorting to real numbers.

However, there is a small drawback: for the same line can have different three ratios. To avoid this, but not to leave integer coefficients, you can apply the following method, often referred to as \bf{valuation}. Find \algohref=euclid_algorithm{greatest common divisor} number $|A|$, $|B|$, $|C|$, divide all three factors, and then perform the normalization: if $A<0$ or $A=0, B<0$, then multiply all three of the coefficient $-1$. In the end, we arrive at the conclusion that for the same direct will have the same three factors that will allow you to easily check out straight for equality.


\h3{ real-valued case }

When working with real numbers should always be mindful of the errors.

The coefficients $A$ and $B$ we get the order of the source coordinates, the coefficient $C$ --- for the order of the square from them. It may already be enough large numbers and, for example, when \algohref=lines_intersection{straight} they will be even greater, which can lead to large rounding errors when the source coordinate is $10^3$.

Therefore, when working with real numbers, it is desirable to produce so-called \bf{normalization} is a line: namely, to make the coefficients such that $A^2 + B^2 = 1$. For this we need to calculate the number of $Z$:

$$ Z = \sqrt{ A^2 + B^2 }, $$

and to share all three of the coefficient $A$, $B$, $C$ on it.

Thus, the order of the coefficients $A$ and $B$ does not depend on the order of the input coordinate, and the coefficient $C$ will be of the same order as the input coordinates. In practice, this leads to a significant improvement of the accuracy of calculations.

Finally, it is worth mentioning the \bf{comparison} direct --- in fact, after this normalization for the same line can have only two triples of coefficients: accurate to multiplication by $-1$. Accordingly, if we perform an additional normalization considering the sign (if $A<-\varepsilon$ or $|A|<\varepsilon, B<-\varepsilon$, then multiply by $-1$), then the resulting coefficients will be unique.


\h2{ three-Dimensional and multidimensional case }

In the three-dimensional case \bf{simple equation} describing the direct (it can be set as the intersection of two planes, i.e. a system of two equations, but it is inconvenient way).

Therefore, in three-dimensional and multidimensional cases we have to use \bf{a parametric approach to defining the direct}, i.e. in the form of a point $p$ and vector $v$:

$$ p + v t, ~~~ t \in \cal{R}. $$

E. T. direct --- all the points you can get from point $p$ by adding vectors $v$ with an arbitrary coefficient.

\bf{Construction} of a line in parametric form the coordinates of the endpoints --- trivial, we just take one end of the segment at the point $p$, and the vector from the first to the second end of the --- for the vector $v$.