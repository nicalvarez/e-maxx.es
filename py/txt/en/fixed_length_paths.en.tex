\h1{ Shortest paths of a fixed length, the number of paths of fixed length }

The following describes these two tasks, built on the same idea: the problem to the construction of matrix (with the usual multiplication operation, and retrofit).


\h2{ the Number of paths of fixed length }

Suppose you are given a directed unweighted graph $G$ with $n$ vertices and given an integer $k$. Required for each pair of vertices $i$ and $j$ to find the number of paths between these vertices, consisting of exactly $k$ edges. The way we consider arbitrary, not necessarily simple (i.e., vertices may be repeated any number of times).

We assume that the graph is set to \bf{a connectivity matrix}, i.e. a matrix $g[][]$ of size $n \times n$ where each element $g[i][j]$ is equal to one if among these vertices there is an edge, and zero if no edges. The algorithm described below works in the case of multiple edges: if some vertices $i$ and $j$ has $m$ edges, then the adjacency matrix should record that the number of $m$. Also, the algorithm correctly takes into account loops in the graph, if any.

It is obvious that in this form \bf{adjacency matrix} of a graph is \bf{the answer to the problem when $k=1$} --- it contains the number of paths of length $1$ between each pair of vertices.

The solution will build \bf{iteratively} let the answer for some $k$ was found, we show how to build it for $k+1$. Let $d_k$ found the answer matrix for $k$, and using $d_{k+1}$ --- the matrix of answers that you want to build. Then the following formula is obvious:

$$ d_{k+1}[i][j] = \sum_{p = 1}^{n} d_k[i][p] \cdot g[p][j]. $$

It is easy to notice that the formula above written --- not that other, as the product of two matrices $d_k$ and $g$ in the usual sense:

$$ d_{k+1} = d_k \cdot g. $$

Thus, \bf{solution} this problem can be represented as follows:

$$ d_k = \underbrace{ g \cdot \ldots \cdot g}_{k\ {\rm times}} = g^k. $$

It remains to notice that the construction of the matrix can be produced efficiently using the algorithm \algohref=binary_pow{\bf{Binary exponentiation}}.

So, the solution has complexity $O (n^3 \log k)$ and it consists in the construction of a binary in $k$-th degree of the adjacency matrix of the graph.


\h2{ Shortest paths of fixed length }

Suppose you are given a directed weighted graph $G$ with $n$ vertices and given an integer $k$. Required for each pair of vertices $i$ and $j$ to find the length of the shortest path between these vertices, consisting of exactly $k$ edges.

We assume that the graph is set to \bf{a connectivity matrix}, i.e. a matrix $g[][]$ of size $n \times n$ where each element $g[i][j]$ contains the length of an edge from vertex $i$ to vertex $j$. If some vertices of the edge, then the corresponding element of the matrix we believe to infinity $\infty$.

It is obvious that in this form \bf{adjacency matrix} of a graph is \bf{the answer to the problem when $k=1$} --- it contains the lengths of shortest paths between every pair of vertices, or $\infty$ if the paths are of length $1$ does not exist.

The solution will build \bf{iteratively} let the answer for some $k$ was found, we show how to build it for $k+1$. Let $d_k$ found the answer matrix for $k$, and using $d_{k+1}$ --- the matrix of answers that you want to build. Then the following formula is obvious:

$$ d_{k+1}[i][j] = \min_{p = 1 \ldots n} ( d_k[i][p] + g[p][j] ). $$

Carefully looking at this formula, it is easy to draw an analogy with matrix multiplication: in fact, the matrix $d_k$ is multiplied by a matrix $g$, only the operation of multiplication instead of a sum over all $p$ takes the minimum over all $p$:

$$ d_{k+1} = d_k \odot g $$

where the operation $\odot$ of the multiplication of two matrices is defined as follows:

$$ A \odot B = C \ \ \Longleftrightarrow\ \ C_{ij} = \min_{p=1 \ldots n} (A_{ip} + B_{pj}). $$

Thus, \bf{solution} this problem can be represented by using the operation of multiplication as follows:

$$ d_k = \underbrace{ g \odot \ldots \odot g}_{k\ {\rm times}} = g^{\odot k}. $$

It remains to notice that the exponentiation with the operation of multiplication can be produced efficiently using the algorithm \algohref=binary_pow{\bf{Binary exponentiation}}, since the only required property for him --- the associative multiplication operation --- obviously, there is.

So, the solution has complexity $O (n^3 \log k)$ and it consists in the construction of a binary in $k$-th degree of the adjacency matrix of the graph with the changed operation of matrix multiplication.


\h2{ the Generalization to the case where the required path length, not more than a given length }

The solutions above solve the problem, when you want to consider ways a certain, fixed length. However, these same solutions can be adapted to solve problems when required to consider paths that contain \bf{more} than the specified number of edges.

You can do this by slightly modifying the input graph. For example, if we are only interested in paths ending in a given vertex of $t$, we can count in \bf{add loop} $(t,t)$ zero weight.

If we continue to be interested in the answers for all pairs of vertices, it is easy to add loops to all vertices will spoil the answer. Instead, \bf{double} each vertex: for each vertex $v$ to vertex $v'$, to hold the edge $(v,v')$ and add a loop $(v',v')$.

Deciding on a modified graph the problem of finding paths of fixed length, the answers to the original problem will be obtained as responses between vertices $i$ and $j'$ (i.e., additional vertices is the top-end in which we can "whirl" the desired number of times).


