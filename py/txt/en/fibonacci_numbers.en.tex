\h1{ Fibonacci Numbers }


\h2{ Definition }

The Fibonacci sequence is defined as follows:

$$ F_0 = 0, $$
$$ F_1 = 1, $$
$$ F_n = F_{n-1} + F_{n-2}. $$

The first few of its members:

$$ 0, ~ 1, ~ 1, ~ 2, ~ 3, ~ 5, ~ 8, ~ 13, ~ 21, ~ 34, ~ , ~ 55, ~ 89, ~ \ldots $$


\h2{ the History }

These numbers were introduced in 1202 Leonardo Fibonacci (Leonardo Fibonacci) (also known as Leonardo of Pisa (Leonardo Pisano)). However, due to mathematics of the 19th century, Luke (Lucas) the name "Fibonacci numbers" was commonly used.

However, Indian mathematicians mentioned the numbers of this sequence before: Gopala (Gopala) to 1135 H., Hemchandra (Hemachandra) --- 1150


\h2{ Fibonacci Numbers in nature }

Fibonacci himself mentioned those numbers in connection with this challenge: "a Man planted a couple of rabbits in a pen, surrounded on all sides by a wall. How many pairs of rabbits a year could produce this pair, if you know that each month, starting from the second, each pair of rabbits produces one pair?". The solution to this problem and will be the number of sequences, now called in his honor. However, Fibonacci described the situation --- more a mind game than a real nature.

Indian mathematicians Gopala and Hemchandra mentioned the number of this sequence in connection with a number of rhythmic patterns, resulting from the alternation of long and short syllables in verse or strong and weak beats. The number of such patterns which are in General $n$ shares, equal to $F_n$.

Fibonacci numbers also appear in the work of Kepler, 1611, who was thinking about numbers found in nature (the work "On hexagonal snowflakes").

An interesting example of a plant --- the yarrow, which has the number of stems (and flowers) there is always a Fibonacci number. The reason is simple: initially with a single stem, the stem then divides into two, then from the main stem branches off another, then the first two stems branch again, then all the stems but the last two, branched, and so on. Thus, each stem after his appearance "skips" one fork, and then begins to divide at each level of branching that results in the Fibonacci numbers.

Generally speaking, many flowers (e.g., lilies) the number of petals is a different Fibonacci number.

In botany it is known the phenomenon of "phyllotaxis". As an example, the arrangement of sunflower seeds: if you look on top of their location, it is possible to see simultaneously two series of spirals (as if superimposed on each other): one is spun clockwise, others-against. It turns out that the number of these spirals is roughly coincide with two consecutive Fibonacci numbers: 55 and 34 or 89 and 144. Similar facts are true for some other colors, as well as for pine cones, broccoli, pineapples, etc.

For many plants (by some estimates, 90% of them) true and interesting fact. Consider any leaf, and will descend from him down until, until we reach the leaf set on the stalk in the same way (i.e., aiming accurately in the same direction). Along the way, we assume all the leaves are coming across to us (i.e., along the height between the starting sheet and ending), but arranged differently. Numbering them, we will gradually make the turns around the stem (because the leaves are on the stalk in a spiral). Depending on, to make the turns clockwise or counterclockwise, to vary the number of turns. But it turns out that the number of turns performed by us clockwise the number of turns of the committed anti-clockwise, and the number of encountered leaves form a 3 consecutive Fibonacci numbers.

However, it should be noted that there are plants for which the above calculations will give the number of different sequences, so one cannot say that the phenomenon of phyllotaxis is the law --- it is rather entertaining trend.


\h2{ Properties }

Fibonacci numbers have many interesting mathematical properties.

Here are some of them:

\ul{

\li the Ratio of Cassini: 

$$ F_{n+1} F_{n-1} - F_n^2 = (-1)^n. $$

\li the Rule of "addition": 

$$ F_{n+k} = F_k F_{n+1} + F_{k-1} F_n. $$

\li the previous equality when $k = n$ it follows: 

$$ F_{2n} = F_n (F_{n+1} + F_{n-1}). $$

\li the previous equality by induction we can obtain that 

$F_{nk}$ is always a multiple of $F_n$.

\li Right and return to the previous statement:

if $F_m$ is a multiple of $F_n$, then $m$ is a multiple of $n$.

\li NODE equality: 

$$ {\rm gcd} (F_m, F_n) = F_{{\rm gcd} (m, n)}. $$

\li In relation to the Euclidean algorithm Fibonacci numbers have the remarkable property that they are the worst inputs for this algorithm (see "Theorem Lama" in \algohref=euclid_algorithm{Euclid's Algorithm}).

}


\h2{ Fibonacci numeral system }

\bf{Theorem Zeckendorf} asserts that every natural number $n$ can be represented uniquely as a sum of Fibonacci numbers:

$$N = F_{k_1} + F_{k_2} + \ldots + F_{k_r}$$

where $k_1 \ge k_2+2$, $k_2 \ge k_3+2$, $\ldots$, $k_r \ge 2$ (i.e. in the account, you cannot use two adjacent Fibonacci numbers).

It follows that any number can be uniquely written in \bf{the Fibonacci number system}, for example:

$$ 9 = 8+1 = F_6 + F_1 = (10001)_F, $$
$$ 6 = 5+1 = F_5 + F_1 = (1001)_F, $$
$$ 19 = 13+5+1 = F_7 + F_5 + F_1 = (101001)_F, $$

and any number can't go two units in a row.

It is easy to obtain and a rule of adding one to the number in the Fibonacci number system: if the lower digit is 0 then replace it by 1, and if equal to 1 (i.e. at the end of 01), 01 is replaced by 10. Then "correcting" the entry, correcting consistently everywhere 011 to 100. As a result, linear time will be the record of the new number.

Translation of Fibonacci number system is a simple "greedy" algorithm: you can just iterate over the Fibonacci numbers from largest to smallest and, if there is some $F_k \le n$, then $F_k$ is included in the record of the number $n$, we take $F_k$ from $n$ and continue the search.


\h2{ the Formula for the nth Fibonacci number }


\h3{ Formula using radicals }

There is a wonderful formula, named after the French mathematician Binet (Binet), although it was known to him de Moivre (Moivre):

$$ F_n = \frac{ \left( \frac{1+\sqrt{5}}{2} \right)^n - \left( \frac{1-\sqrt{5}}{2} \right)^n }{ \sqrt{5} }. $$

This formula is easy to prove by induction, but display it by using the concepts forming functions or by solving functional equations.

Immediately you notice that the second term always have magnitudes less than 1, and moreover, decreases very quickly (exponentially). It follows that the value of the first addend gives "almost" the value of $F_n$. This can be written simply as:

$$F_n = \left[ \frac{ \left( \frac{1+\sqrt{5}}{2} \right)^n }{ \sqrt{5} } \right]$$

where the square brackets denote rounding to the nearest integer.

However, for practical use in calculations these formulae are not suitable because they require very high precision work with fractional numbers.


\h3{ Matrix formula for Fibonacci numbers }

It is easy to prove the following matrix equality:

$$ \pmatrix{
F_{n-2} & F_{n-1} \cr
} \cdot \pmatrix{
0 & 1 \cr
1 & 1 \cr
} = \pmatrix{
F_{n-1} & F_{n} \cr
}. $$

But then, labeling

$$ P \equiv
\pmatrix{
0 & 1 \cr
1 & 1 \cr
}, $$

received:

$$ \pmatrix{
F_0 & F_1 \cr
} \cdot P^n = \pmatrix{
F_{n} & F_{n+1} \cr
}. $$

Thus, for finding the $n$-th Fibonacci number it is necessary to build the matrix $P$ in degree $n$.

Remembering that the construction of the matrices in $n$-th degree can be done in $O (\log n)$ (see \algohref=binary_pow{Binary exponentiation}), it follows that $n$-th Fibonacci number can be easily computed in $O (\log n)$ using only integer arithmetic.


\h2{ the Periodicity of Fibonacci sequence modulo }

Consider the Fibonacci sequence $F_i$ for some module $p$. Let us prove that it is periodic, and the period begins with $F_1=1$ (i.e. preperiod only contains $F_0$).

Prove it by contradiction. Consider $p^2+1$ pairs of Fibonacci numbers modulo $p$:

$$(F_1,F_2),\ (F_2,F_3),\ \ldots,\ (F_{p^2+1},F_{p^2+2}).$$

Since modulo $p$ can only be $p^2$ distinct pairs, the sequence will be at least two identical pairs. This means that the sequence is periodic.

Now we choose among all such identical pairs of two identical pairs with the smallest rooms. Let it pair with some numbers $(F_a,F_{a+1})$ and $(F_b,F_{b+1})$. Prove that $a=1$. Indeed, otherwise, for them there are previous pairs $(F_{a-1},F_a)$ and $(F_{b-1},F_b)$, which, by a property of Fibonacci numbers will also be equal to each other. However, this contradicts that we chose matching pairs with the smallest rooms that we wanted to prove.


\h2{ Literature }

\ul{
\li \book{Ronald Graham, Donald Knuth, Oren Patashnik}{Concrete mathematics}{1998}{graham.djvu}
}
