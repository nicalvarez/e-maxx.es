the <h2>Compute the determinant of a matrix by Gauss method</h2>
<p>given A square matrix A of size NxN. Is required to compute its determinant.</p>
the <h2>the Algorithm</h2>
<p>Use the ideas <algohref=linear_systems_gauss>Gaussian elimination for solving systems of linear equations</algohref>.</p>
<p>let's perform the same steps as in solving systems of linear equations, excluding only division in the current row a[i][i] (more precisely, the space itself can be performed, but implying that the number is taken out of the sign of the determinant). Then all the operations that we will produce with the matrix, will not modify the value of the determinant of the matrix, except, perhaps, a sign (we only exchanged places two lines that changes its sign to the opposite, or add one row to another without changing the value of the determinant).</p>
<p>But a matrix, to which we arrive after performing Gaussian elimination, is diagonal, and its determinant equals the product of the elements standing on a diagonal. The sign, as already mentioned, will be determined by the number of exchanges of rows (if odd, then the sign of the determinant should be changed to the opposite). Thus, we can use Gaussian elimination to calculate the determinant of a matrix in O (N<sup>3</sup>).</p>
<p>it Remains only to note that if at some point we find in the current column of the nonzero element, then the algorithm should stop and return 0.</p>
the <h2>Implementation</h2>
<code>const double EPS = 1E-9;
int n;
vector < vector<double> > a (n, vector<double> (n));
... read n and a ...

double det = 1;
for (int i=0; i<n; ++i) {
int k = i;
for (int j=i+1; j<n; j++)
if (abs (a[j][i]) > abs (a[k][i]))
k = j;
if (abs (a[k][i]) < EPS) {
det = 0;
break;
}
swap (a[i], a[k]);
if (i != k)
det = -det;
det *= a[i][i];
for (int j=i+1; j<n; j++)
a[i][j] /= a[i][i];
for (int j=0; j<n; j++)
if (j != i && abs (a[j][i]) > EPS)
for (int k=i+1; k<n; k++)
a[j][k] = a[i][k] * a[j][i];
}

cout << det;</code>