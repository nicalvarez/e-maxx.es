\h1{ Линейные диофантовы уравнения с двумя переменными }

Диофантово уравнение с двумя неизвестными имеет вид:

$$ a \cdot x + b \cdot y = c, $$

где $a, b, c$ --- заданные целые числа, $x$ и $y$ --- неизвестные целые числа.

Ниже рассматриваются несколько классических задач на эти уравнения: нахождение любого решения, получение всех решений, нахождение количества решений и сами решения в определённом отрезке, нахождение решения с наименьшей суммой неизвестных.



\h2{ Вырожденный случай }

Один вырожденный случай мы сразу исключим из рассмотрения: когда $a = b = 0$. В этом случае, понятно, уравнение имеет либо бесконечно много произвольных решений, либо же не имеет решений вовсе (в зависимости от того, $c = 0$ или нет).



\h2{ Нахождение одного решения }

Найти одно из решений диофантова уравнения с двумя неизвестными можно с помощью \algohref=extended_euclid_algorithm{Расширенного алгоритма Евклида}. Предположим сначала, что числа $a$ и $b$ неотрицательны.

Расширенный алгоритм Евклида по заданным неотрицательным числам $a$ и $b$ находит их наибольший общий делитель $g$, а также такие коэффициенты $x_g$ и $y_g$, что:

$$ a \cdot x_g + b \cdot y_g = g. $$

Утверждается, что если $c$ делится на $g = {\rm gcd} (a,b)$, то диофантово уравнение $a \cdot x + b \cdot y = c$ имеет решение; в противном случае диофантово уравнение решений не имеет. Доказательство следует из очевидного факта, что линейная комбинация двух чисел по-прежнему должна делиться на их общий делитель.

Предположим, что $c$ делится на $g$, тогда, очевидно, выполняется:

$$ a \cdot x_g \cdot (c/g) + b \cdot y_g \cdot (c/g) = c, $$

т.е. одним из решений диофантова уравнения являются числа:

$$ \cases{
x_0 = x_g \cdot (c / g), \cr
y_0 = y_g \cdot (c / g).
} $$

Мы описали решение в случае, когда числа $a$ и $b$ неотрицательны. Если же одно из них или они оба отрицательны, то можно поступить таким образом: взять их по модулю и применить к ним алгоритм Евклида, как было описано выше, а затем изменить знак найденных $x_0$ и $y_0$ в соответствии с настоящим знаком чисел $a$ и $b$ соответственно.

Реализация (напомним, здесь мы считаем, что входные данные $a=b=0$ недопустимы):

\code
int gcd (int a, int b, int & x, int & y) {
	if (a == 0) {
		x = 0; y = 1;
		return b;
	}
	int x1, y1;
	int d = gcd (b%a, a, x1, y1);
	x = y1 - (b / a) * x1;
	y = x1;
	return d;
}

bool find_any_solution (int a, int b, int c, int & x0, int & y0, int & g) {
	g = gcd (abs(a), abs(b), x0, y0);
	if (c % g != 0)
		return false;
	x0 *= c / g;
	y0 *= c / g;
	if (a < 0)   x0 *= -1;
	if (b < 0)   y0 *= -1;
	return true;
}
\endcode



\h2{ Получение всех решений }

Покажем, как получить все остальные решения (а их бесконечное множество) диофантова уравнения, зная одно из решений $(x_0,y_0)$.

Итак, пусть $g = {\rm gcd}(a,b)$, а числа $x_0, y_0$ удовлетворяют условию:

$$ a \cdot x_0 + b \cdot y_0 = c. $$

Тогда заметим, что, прибавив к $x_0$ число $b/g$ и одновременно отняв $a/g$ от $y_0$, мы не нарушим равенства:

$$ a \cdot (x_0 + b/g) + b \cdot (y_0 - a/g) = a \cdot x_0 + b \cdot y_0 + a \cdot b/g - b \cdot a/g = c. $$

Очевидно, что этот процесс можно повторять сколько угодно, т.е. все числа вида:

$$ \cases{
x = x_0 + k \cdot b/g, \cr
y = y_0 - k \cdot a/g,
} ~~~~k \in Z $$

являются решениями диофантова уравнения.

Более того, только числа такого вида и являются решениями, т.е. мы описали множество всех решений диофантова уравнения (оно получилось бесконечным, если не наложено дополнительных условий).



\h2{ Нахождение количества решений и сами решения в заданном отрезке }

Пусть даны два отрезка $[min_x;max_x]$ и $[min_y;max_y]$, и требуется найти количество решений $(x,y)$ диофантова уравнения, лежащих в данных отрезках соответственно.

Заметим, что если одно из чисел $a, b$ равно нулю, то задача имеет не больше одного решения, поэтому эти случаи мы в данном разделе исключаем из рассмотрения.

Сначала найдём решение с минимальным подходящим $x$, т.е. $x \ge min_x$. Для этого сначала найдём любое решение диофантова уравнения (см. пункт 1). Затем получим из него решение с наименьшим $x \ge min_x$ --- для этого воспользуемся процедурой, описанной в предыдущем пункте, и будем уменьшать/увеличивать $x$, пока оно не окажется $\ge min_x$, и при этом минимальным. Это можно сделать за $O(1)$, посчитав, с каким коэффициентом нужно применить это преобразование, чтобы получить минимальное число, большее либо равное $min_x$. Обозначим найденный $x$ через $lx1$.

Аналогичным образом можно найти и решение с максимальным подходящим $x=rx1$, т.е. $x \le max_x$.

Далее перейдём к удовлетворению ограничений на $y$, т.е. к рассмотрению отрезка $[min_y;max_y]$. Способом, описанным выше, найдём решение с минимальным $y \ge min_y$, а также решение с максимальным $y \le max_y$. Обозначим $x$-коэффициенты этих решений через $lx2$ и $rx2$ соответственно.

Пересечём отрезки $[lx1;rx1]$ и $[lx2;rx2]$; обозначим получившийся отрезок через $[lx;rx]$. Утверждается, что любое решение, у которого $x$-коэффициент лежит в $[lx;rx]$ --- любое такое решение является подходящим. (Это верно в силу построения этого отрезка: сначала мы отдельно удовлетворили ограничения на $x$ и $y$, получив два отрезка, а затем пересекли их, получив область, в которой удовлетворяются оба условия.)

Таким образом, количество решений будет равняться длине этого отрезка, делённой на $|b|$ (поскольку $x$-коэффициент может изменяться только на $\pm b$), и плюс один.

Приведём реализацию (она получилась достаточно сложной, поскольку требуется аккуратно рассматривать случаи положительных и отрицательных коэффициентов $a$ и $b$):

\code
void shift_solution (int & x, int & y, int a, int b, int cnt) {
	x += cnt * b;
	y -= cnt * a;
}

int find_all_solutions (int a, int b, int c, int minx, int maxx, int miny, int maxy) {
	int x, y, g;
	if (! find_any_solution (a, b, c, x, y, g))
		return 0;
	a /= g;  b /= g;

	int sign_a = a>0 ? +1 : -1;
	int sign_b = b>0 ? +1 : -1;

	shift_solution (x, y, a, b, (minx - x) / b);
	if (x < minx)
		shift_solution (x, y, a, b, sign_b);
	if (x > maxx)
		return 0;
	int lx1 = x;

	shift_solution (x, y, a, b, (maxx - x) / b);
	if (x > maxx)
		shift_solution (x, y, a, b, -sign_b);
	int rx1 = x;

	shift_solution (x, y, a, b, - (miny - y) / a);
	if (y < miny)
		shift_solution (x, y, a, b, -sign_a);
	if (y > maxy)
		return 0;
	int lx2 = x;

	shift_solution (x, y, a, b, - (maxy - y) / a);
	if (y > maxy)
		shift_solution (x, y, a, b, sign_a);
	int rx2 = x;

	if (lx2 > rx2)
		swap (lx2, rx2);
	int lx = max (lx1, lx2);
	int rx = min (rx1, rx2);

	return (rx - lx) / abs(b) + 1;
}
\endcode

Также нетрудно добавить к этой реализации вывод всех найденных решений: для этого достаточно перебрать $x$ в отрезке $[lx;rx]$ с шагом $|b|$, найдя для каждого из них соответствующий $y$ непосредственно из уравнения $ax+by=c$.


\h2{ Нахождение решения в заданном отрезке с наименьшей суммой x+y }

Здесь на $x$ и на $y$ также должны быть наложены какие-либо ограничения, иначе ответом практически всегда будет минус бесконечность.

Идея решения такая же, как и в предыдущем пункте: сначала находим любое решение диофантова уравнения, а затем, применяя описанную в предыдущем пункте процедуру, придём к наилучшему решению.

Действительно, мы имеем право выполнить следующее преобразование (см. предыдущий пункт):

$$ \cases{
x^\prime = x + k \cdot (b/g), \cr
y^\prime = y - k \cdot (a/g),
} ~~~~k \in Z.$$

Заметим, что при этом сумма $x+y$ меняется следующим образом:

$$x^\prime + y^\prime = x + y + k \cdot (b/g - a/g) = x + y + k \cdot (b - a) / g.$$

Т.е. если $a < b$, то нужно выбрать как можно меньшее значение $k$, если $a > b$, то нужно выбрать как можно большее значение $k$.

Если $a = b$, то мы никак не сможем улучшить решение, --- все решения будут обладать одной и той же суммой.



\h2{ Задачи в online judges }

Список задач, которые можно сдать на тему диофантовых уравнений с двумя неизвестными:

\ul{

\li \href=http://acm.sgu.ru/problem.php?contest=0&problem=106{SGU #106 \bf{"The Equation"} ~~~~ [сложность: средняя]}

}
