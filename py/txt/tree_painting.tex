<h1>Покраска рёбер дерева</h1>

<p>Это достаточно часто встречающаяся задача. Дано дерево G. Поступают запросы двух видов: первый вид - покрасить некоторое ребро, второй вид - запрос количества покрашенных рёбер между двумя вершинами.</p>
<p>Здесь будет описано достаточно простое решение (с использованием <algohref=segment_tree>дерева отрезков</algohref>), которое будет отвечать на запросы за O (log N), с препроцессингом (предварительной обработкой дерева) за O (M).</p>
<h2>Решение</h2>
<p>Для начала нам придётся реализовать <algohref=lca>LCA</algohref>, чтобы каждый запрос второго вида (i,j) сводить к двум запросам (a,b), где a - предок b.</p>
<p>Теперь опишем <b>препроцессинг</b> собственно для нашей задачи. Запустим поиск в глубину из корня дерева, этот поиск в глубину составит некоторый список посещения вершин (каждая вершина добавляется в список, когда поиск заходит в неё, и каждый раз после того, как поиск в глубину возвращается из сына текущей вершины) - кстати говоря, этот же список используется алгоритмом LCA. В этом списке будет присутствовать каждое ребро (в том смысле, что если i и j - концы ребра, то в списке обязательно найдётся место, где i и j идут подряд друг за другом), причём присутствовать ровно 2 раза: в прямом направлении (из i в j, где вершина i ближе к корню, чем вершина j) и в обратном (из j в i).</p>
<p><b>Построим</b> два дерева отрезков (для суммы, с единичной модификацией) по этому списку: T1 и T2. Дерево T1 будет учитывать каждое ребро только в прямом направлении, а дерево T2 - наоборот, только в обратном.</p>
<p>Пусть поступил очередной <b>запрос</b> вида (i,j), где i - предок j, и требуется определить, сколько рёбер покрашено на пути между i и j. Найдём i и j в списке обхода в глубину (нам обязательно нужны позиции, где они встречаются впервые), пусть это некоторые позиции p и q (это можно сделать за O (1), если вычислить эти позиции заранее во время препроцессинга). Тогда <b>ответом будет сумма T1[p..q-1] - сумма T2[p..q-1]</b>.</p>
<p>Почему? Рассмотрим отрезок [p;q] в списке обхода в глубину. Он содержит рёбра нужного нам пути из i в j, но также содержит и множество рёбер, которые лежат на других путях из i. Однако между нужными нам рёбрами и остальными рёбрами есть одно большое отличие: нужные рёбра будут содержаться в этом списке только один раз, причём в прямом направлении, а все остальные рёбра будут встречаться дважды: и в прямом, и в обратном направлении. Следовательно, разность T1[p..q-1] - T2[p..q-1] даст нам ответ (минус один нужно, потому что иначе мы захватим ещё лишнее ребро из вершины j куда-то вниз или вверх). Запрос суммы в дереве отрезков выполняется за O (log N).</p>
<p>Ответ на <b>запрос</b> вида 1 (о покраске какого-либо ребра) ещё проще - нам просто нужно обновить T1 и T2, а именно выполнить единичную модификацию того элемента, который соответствует нашему ребру (найти ребро в списке обхода, опять же, можно за O (1), если выполнить этот поиск в препроцессинге). Единичная модификация в дереве отрезков выполняется за O (log N).</p>
<h2>Реализация</h2>
<p>Здесь будет приведена полная реализация решения, включая LCA:</p>
<code>const int INF = 1000*1000*1000;

typedef vector < vector<int> > graph;

vector<int> dfs_list;
vector<int> ribs_list;
vector<int> h;

void dfs (int v, const graph & g, const graph & rib_ids, int cur_h = 1)
{
	h[v] = cur_h;
	dfs_list.push_back (v);
	for (size_t i=0; i<g[v].size(); ++i)
		if (h[g[v][i]] == -1)
		{
			ribs_list.push_back (rib_ids[v][i]);
			dfs (g[v][i], g, rib_ids, cur_h+1);
			ribs_list.push_back (rib_ids[v][i]);
			dfs_list.push_back (v);
		}
}

vector<int> lca_tree;
vector<int> first;

void lca_tree_build (int i, int l, int r)
{
	if (l == r)
		lca_tree[i] = dfs_list[l];
	else
	{
		int m = (l + r) >> 1;
		lca_tree_build (i+i, l, m);
		lca_tree_build (i+i+1, m+1, r);
		int lt = lca_tree[i+i],  rt = lca_tree[i+i+1];
		lca_tree[i] = h[lt] < h[rt] ? lt : rt;
	}
}

void lca_prepare (int n)
{
	lca_tree.assign (dfs_list.size() * 8, -1);
	lca_tree_build (1, 0, (int)dfs_list.size()-1);

	first.assign (n, -1);
	for (int i=0; i < (int)dfs_list.size(); ++i)
	{
		int v = dfs_list[i];
		if (first[v] == -1)  first[v] = i;
	}
}

int lca_tree_query (int i, int tl, int tr, int l, int r)
{
	if (tl == l && tr == r)
		return lca_tree[i];
	int m = (tl + tr) >> 1;
	if (r <= m)
		return lca_tree_query (i+i, tl, m, l, r);
	if (l > m)
		return lca_tree_query (i+i+1, m+1, tr, l, r);
	int lt = lca_tree_query (i+i, tl, m, l, m);
	int rt = lca_tree_query (i+i+1, m+1, tr, m+1, r);
	return h[lt] < h[rt] ? lt : rt;
}

int lca (int a, int b)
{
	if (first[a] > first[b])  swap (a, b);
	return lca_tree_query (1, 0, (int)dfs_list.size()-1, first[a], first[b]);
}


vector<int> first1, first2;
vector<char> rib_used;
vector<int> tree1, tree2;

void query_prepare (int n)
{
	first1.resize (n-1, -1);
	first2.resize (n-1, -1);
	for (int i = 0; i < (int) ribs_list.size(); ++i)
	{
		int j = ribs_list[i];
		if (first1[j] == -1)
			first1[j] = i;
		else
			first2[j] = i;
	}

	rib_used.resize (n-1);
	tree1.resize (ribs_list.size() * 8);
	tree2.resize (ribs_list.size() * 8);
}

void sum_tree_update (vector<int> & tree, int i, int l, int r, int j, int delta)
{
	tree[i] += delta;
	if (l < r)
	{
		int m = (l + r) >> 1;
		if (j <= m)
			sum_tree_update (tree, i+i, l, m, j, delta);
		else
			sum_tree_update (tree, i+i+1, m+1, r, j, delta);
	}
}

int sum_tree_query (const vector<int> & tree, int i, int tl, int tr, int l, int r)
{
	if (l > r || tl > tr)  return 0;
	if (tl == l && tr == r)
		return tree[i];
	int m = (tl + tr) >> 1;
	if (r <= m)
		return sum_tree_query (tree, i+i, tl, m, l, r);
	if (l > m)
		return sum_tree_query (tree, i+i+1, m+1, tr, l, r);
	return sum_tree_query (tree, i+i, tl, m, l, m)
		+ sum_tree_query (tree, i+i+1, m+1, tr, m+1, r);
}

int query (int v1, int v2)
{
	return sum_tree_query (tree1, 1, 0, (int)ribs_list.size()-1, first[v1], first[v2]-1)
		- sum_tree_query (tree2, 1, 0, (int)ribs_list.size()-1, first[v1], first[v2]-1);
}


int main()
{
	// чтение графа
	int n;
	scanf ("%d", &n);
	graph g (n), rib_ids (n);
	for (int i=0; i<n-1; ++i)
	{
		int v1, v2;
		scanf ("%d%d", &v1, &v2);
		--v1, --v2;
		g[v1].push_back (v2);
		g[v2].push_back (v1);
		rib_ids[v1].push_back (i);
		rib_ids[v2].push_back (i);
	}

	h.assign (n, -1);
	dfs (0, g, rib_ids);
	lca_prepare (n);
	query_prepare (n);

	for (;;) {
		if () {
			// запрос о покраске ребра с номером x;
			//    если start=true, то ребро красится, иначе покраска снимается
			rib_used[x] = start;
			sum_tree_update (tree1, 1, 0, (int)ribs_list.size()-1, first1[x], start?1:-1);
			sum_tree_update (tree2, 1, 0, (int)ribs_list.size()-1, first2[x], start?1:-1);
		}
		else {
			// запрос кол-ва покрашенных рёбер на пути между v1 и v2
			int l = lca (v1, v2);
			int result = query (l, v1) + query (l, v2);
			// result - ответ на запрос
		}
	}

}</code>