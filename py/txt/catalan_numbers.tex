\h1{Числа Каталана}

Числа Каталана --- числовая последовательность, встречающаяся в удивительном числе комбинаторных задач.

Эта последовательность названа в честь бельгийского математика Каталана (Catalan), жившего в 19 веке, хотя на самом деле она была известна ещё Эйлеру (Euler), жившему за век до Каталана.

\h2{Последовательность}

Первые несколько чисел Каталана $C_n$ (начиная с нулевого):
$$ 1,\ 1,\ 2,\ 5,\ 14,\ 42,\ 132,\ 429,\ 1430,\ \ldots $$

Числа Каталана встречаются в большом количестве задач комбинаторики. \bf{$n$-ое число Каталана} --- это:
\ul{
\li Количество корректных скобочных последовательностей, состоящих из $n$ открывающих и $n$ закрывающих скобок.
\li Количество корневых бинарных деревьев с $n+1$ листьями (вершины не пронумерованы).
\li Количество способов полностью разделить скобками $n+1$ множитель.
\li Количество триангуляций выпуклого $n+2$-угольника (т.е. количество разбиений многоугольника непересекающимися диагоналями на треугольники).
\li Количество способов соединить $2n$ точек на окружности $n$ непересекающимися хордами.
\li Количество неизоморфных полных бинарных деревьев с $n$ внутренними вершинами (т.е. имеющими хотя бы одного сына).
\li Количество монотонных путей из точки $(0,0)$ в точку $(n,n)$ в квадратной решётке размером $n \times n$, не поднимающихся над главной диагональю.
\li Количество перестановок длины $n$, которые можно отсортировать стеком (можно показать, что перестановка является сортируемой стеком тогда и только тогда, когда нет таких индексов $i<j<k$, что $a_k<a_i<a_j$).
\li Количество непрерывных разбиений множества из $n$ элементов (т.е. разбиений на непрерывные блоки).
\li Количество способов покрыть лесенку $1 \ldots n$ с помощью $n$ прямоугольников (имеется в виду фигура, состоящая из $n$ столбцов, $i$-ый из которых имеет высоту $i$).
}

\h2{Вычисление}

Имеется две формулы для чисел Каталана: рекуррентная и аналитическая. Поскольку мы считаем, что все приведённые выше задачи эквивалентны, то для доказательства формул мы будем выбирать ту задачу, с помощью которой это сделать проще всего.

\h3{Рекуррентная формула}

$$ C_n = \sum_{k=0}^{n-1} C_k C_{n-1-k} $$

Рекуррентную формулу легко вывести из задачи о правильных скобочных последовательностях.

Самой левой открывающей скобке l соответствует определённая закрывающая скобка r, которая разбивает формулу две части, каждая из которых в свою очередь является правильной скобочной последовательностью. Поэтому, если мы обозначим $k = r-l-1$, то для любого фиксированного $r$ будет ровно $C_k C_{n-1-k} $ способов. Суммируя это по всем допустимым $k$, мы и получаем рекуррентную зависимость на $C_n$.

\h3{Аналитическая формула}

$$ C_n = \frac{1}{n+1} C_{2n}^{n} $$

(здесь через $C_n^k$ обозначен, как обычно, \algohref=binomial_coeff{биномиальный коэффициент}).

Эту формулу проще всего вывести из задачи о монотонных путях. Общее количество монотонных путей в решётке размером $n \times n$ равно $C_{2n}^{n}$. Теперь посчитаем количество монотонных путей, пересекающих диагональ. Рассмотрим какой-либо из таких путей, и найдём первое ребро, которое стоит выше диагонали. Отразим относительно диагонали весь путь, идущий после этого ребра. В результате получим монотонный путь в решётке $(n-1) \times (n+1)$. Но, с другой стороны, любой монотонный путь в решётке $(n-1) \times (n+1)$ обязательно пересекает диагональ, следовательно, он получен как раз таким способом из какого-либо (причём единственного) монотонного пути, пересекающего диагональ, в решётке $n \times n$. Монотонных путей в решётке $(n-1) \times (n+1)$ имеется $C_{2n}^{n-1}$. В результате получаем формулу:

$$ C_n = C_{2n}^{n} - C_{2n}^{n-1} = \frac{1}{n+1} C_{2n}^{n} $$