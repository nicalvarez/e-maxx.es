\h1{Декомпозиция Линдона. Алгоритм Дюваля. Нахождение наименьшего циклического сдвига}


\h2{Понятие декомпозиции Линдона}

Определим понятие \bf{декомпозиции Линдона} (Lyndon decomposition).

Строка называется \bf{простой}, если она строго \bf{меньше} любого своего собственного \bf{суффикса}. Примеры простых строк: $a$, $b$, $ab$, $aab$, $abb$, $ababb$, $abcd$. Можно показать, что строка является простой тогда и только тогда, когда она строго \bf{меньше} всех своих нетривиальных \bf{циклических сдвигов}.

Далее, пусть дана строка $s$. Тогда \bf{декомпозицией Линдона} строки $s$ называется её разложение $s = w_1 w_2 \ldots w_k$, где строки $w_i$ просты, и при этом $w_1 \ge w_2 \ge \ldots \ge w_k$.

Можно показать, что для любой строки $s$ это разложение существует и единственно.


\h2{Алгоритм Дюваля}

\bf{Алгоритм Дюваля} (Duval's algorithm) находит для данной строки длины $n$ декомпозицию Линдона за время $O (n)$ с использованием $O (1)$ дополнительной памяти.

Работать со строками будем в 0-индексации.

Введём вспомогательное понятие предпростой строки. Строка $t$ называется \bf{предпростой}, если она имеет вид $t = w w w \ldots w \overline{w}$, где $w$ --- некоторая простая строка, а $\overline{w}$ --- некоторый префикс строки $w$.

Алгоритм Дюваля является жадным. В любой момент его работы строка S фактически разделена на три строки $s = s_1 s_2 s_3$, где в строке $s_1$ декомпозиция Линдона уже найдена и $s_1$ уже больше не используется алгоритмом; строка $s_2$ --- это предпростая строка (причём длину простых строк внутри неё мы также запоминаем); строка $s_3$ --- это ещё не обработанная часть строки $s$. Каждый раз алгоритм Дюваля берёт первый символ строки $s_3$ и пытается дописать его к строке $s_2$. При этом, возможно, для какого-то префикса строки $s_2$ декомпозиция Линдона становится известной, и эта часть переходит к строке $s_1$.

Опишем теперь алгоритм \bf{формально}. Во-первых, будет поддерживаться указатель $i$ на начало строки $s_2$. Внешний цикл алгоритма будет выполняться, пока $i < n$, т.е. пока вся строка $s$ не перейдёт в строку $s_1$. Внутри этого цикла создаются два указателя: указатель $j$ на начало строки $s_3$ (фактически указатель на следующий символ-кандидат) и указатель $k$ на текущий символ в строке $s_2$, с которым будет производиться сравнение. Затем будем в цикле пытаться добавить символ $s[j]$ к строке $s_2$, для чего необходимо произвести сравнение с символом $s[k]$. Здесь у нас возникают три различных случая:

\ul{

\li Если $s[j] = s[k]$, то мы можем дописать символ $s[j]$ к строке $s_2$, не нарушив её "предпростоты". Следовательно, в этом случае мы просто увеличиваем указатели $j$ и $k$ на единицу.

\li Если $s[j] > s[k]$, то, очевидно, строка $s_2 + s[j]$ станет простой. Тогда мы увеличиваем $j$ на единицу, а $k$ передвигаем обратно на $i$, чтобы следующий символ сравнивался с первым символом $s_2$.

\li Если $s[j] < s[k]$, то строка $s_2+s[j]$ уже не может быть предпростой. Поэтому мы разбиваем предпростую строку $s_2$ на простые строки плюс "остаток" (префикс простой строки, возможно, пустой); простые строки добавляем в ответ (т.е. выводим их позиции, попутно передвигая указатель $i$), а "остаток" вместе с символом $s[j]$ переводим обратно в строку $s_3$, и останавливаем выполнение внутреннего цикла. Тем самым мы на следующей итерации внешнего цикла заново обработаем остаток, зная, что он не мог образовать предпростую строку с предыдущими простыми строками. Осталось только заметить, что при выводе позиций простых строк нам нужно знать их длину; но она, очевидно, равна $j-k$.

}


\h3{Реализация}

Приведём реализацию алгоритма Дюваля, которая будет выводить искомую декомпозицию Линдона строки $s$:

\code
string s; // входная строка
int n = (int) s.length();
int i=0;
while (i < n) {
	int j=i+1, k=i;
	while (j < n && s[k] <= s[j]) {
		if (s[k] < s[j])
			k = i;
		else
			++k;
		++j;
	}
	while (i <= k) {
		cout << s.substr (i, j-k) << ' ';
		i += j - k;
	}
}
\endcode


\h3{Асимптотика}

Сразу заметим, что для алгоритма Дюваля требуется \bf{$O (1)$ памяти}, а именно три указателя $i$, $j$, $k$.

Оценим теперь \bf{время работы} алгоритма.

\bf{Внешний цикл while} делает не более $n$ итераций, поскольку в конце каждой его итерации выводится как минимум один символ (а всего символов выводится, очевидно, ровно $n$).

Оценим теперь количество итераций \bf{первого вложенного цикла while}. Для этого рассмотрим второй вложенный цикл while --- он при каждом своём запуске выводит некоторое количество $r \ge 1$ копий одной и той же простой строки некоторой длины $p = j-k$. Заметим, что строка $s_2$ является предпростой, причём её простые строки имеют длину как раз $p$, т.е. её длина не превосходит $r p + p - 1$. Поскольку длина строки $s_2$ равна $j-i$, а указатель $j$ увеличивается по единице на каждой итерации первого вложенного цикла while, то этот цикл выполнит не более $r p + p - 2$ итераций. Худшим случаем является случай $r = 1$, и мы получаем, что первый вложенный цикл while всякий раз выполняет не более $2 p - 2$ итераций. Вспоминая, что всего выводится $n$ символов, получаем, что для вывода $n$ символов требуется не более $2 n - 2$ итераций первого вложенного while-а.

Следовательно, \bf{алгоритм Дюваля выполняется за $O (n)$}.

Легко оценить и число сравнений символов, выполняемых алгоритмом Дюваля. Поскольку каждая итерация первого вложенного цикла while производит два сравнения символов, а также одно сравнение производится после последней итерации цикла (чтобы понять, что цикл должен остановиться), то общее \bf{число сравнений символов} не превосходит $4 n - 3$.


\h2{Нахождение наименьшего циклического сдвига}

Пусть дана строка $s$. Построим для строки $s+s$ декомпозицию Линдона (мы можем это сделать за $O (n)$ времени и $O (1)$ памяти (если не выполнять конкатенацию в явном виде)). Найдём предпростой блок, который начинается в позиции, меньшей $n$ (т.е. в первом экземпляре строки $s$), и заканчивается в позиции, большей или равной n (т.е. во втором экземпляре). Утверждается, что \bf{позиция начала} этого блока и будет началом искомого циклического сдвига (в этом легко убедиться, воспользовавшись определением декомпозиции Линдона).

Начало предпростого блока найти просто --- достаточно заметить, что указатель $i$ в начале каждой итерации внешнего цикла while указывает на начало текущего предпростого блока.

Итого мы получаем такую \bf{реализацию} (для упрощения кода она использует $O (n)$ памяти, явным образом дописывая строку к себе):

\code
string min_cyclic_shift (string s) {
	s += s;
	int n = (int) s.length();
	int i=0, ans=0;
	while (i < n/2) {
		ans = i;
		int j=i+1, k=i;
		while (j < n && s[k] <= s[j]) {
			if (s[k] < s[j])
				k = i;
			else
				++k;
			++j;
		}
		while (i <= k)  i += j - k;
	}
	return s.substr (ans, n/2);
}
\endcode


\h2{ Задачи в online judges }

Список задач, которые можно решить, используя алгоритм Дюваля:

\ul{

\li \href=http://uva.onlinejudge.org/index.php?option=onlinejudge&page=show_problem&problem=660{UVA #719 \bf{"Glass Beads"} ~~~~ [сложность: низкая]}

}