\h2{Количество помеченных графов}

Дано число $N$ вершин. Требуется посчитать количество $G_N$ различных помеченных графов с $N$ вершинами (т.е. вершины графа помечены различными числами от $1$ до $N$, и графы сравниваются с учётом этой покраски вершин). Рёбра графа неориентированы, петли и кратные рёбра запрещены.

Рассмотрим множество всех возможных рёбер графа. Для любого ребра $(i,j)$ положим, что $i<j$ (основываясь на неориентированности графа и отсутствии петель). Тогда множество всех возможных рёбер графа имеет мощность $C_N^2$, т.е. $\frac{ N (N-1) }{ 2 }$.

Поскольку любой помеченный граф однозначно определяется своими рёбрами, то количество помеченных графов с $N$ вершинами равно:
$$ G_N = 2^{ \frac{ N (N-1) }{ 2 } } $$

\h2{Количество связных помеченных графов}

По сравнению с предыдущей задачей мы дополнительно накладываем ограничение, что граф должен быть связным.

Обозначим искомое число через $Conn_N$.

Научимся, наоборот, считать количество \bf{несвязных} графов; тогда количество связных графов получится как $G_N$ минус найденное число. Более того, научимся считать количество \bf{корневых} (т.е. с выделенной вершиной - корнем) \bf{несвязных графов}; тогда количество несвязных графов будет получаться из него делением на $N$. Заметим, что, так как граф несвязный, то в нём найдётся компонента связности, внутри которой лежит корень, а остальной граф будет представлять собой ещё несколько (как минимум одну) компонент связности.

Переберём количество $K$ вершин в этой компоненте связности, содержащей корень (очевидно, $K = 1 \ldots N-1$), и найдём количество таких графов. Во-первых, мы должны выбрать $K$ вершин из $N$, т.е. ответ умножается на $C_N^K$. Во-вторых, компонента связности с корнем даёт множитель $Conn_K$. В-третьих, оставшийся граф из $N-K$ вершин является произвольным графом, а потому он даёт множитель $G_{N-K}$. Наконец, количество способов выделить корень в компоненте связности из $K$ вершин равно $K$. Итого, при фиксированном $K$ количество \bf{корневых несвязных} графов равно:
$$ K\ C_N^K\ Conn_K\ G_{N-K} $$

Значит, количество \bf{несвязных} графов с $N$ вершинами равно:
$$ \frac{1}{N} \sum_{K=1}^{N-1} K\ C_N^K\ Conn_K\ G_{N-K} $$

Наконец, искомое количество \bf{связных} графов равно:
$$ Conn_N = G_N - \frac{1}{N} \sum_{K=1}^{N-1} K\ C_N^K\ Conn_K\ G_{N-K} $$

\h2{Количество помеченных графов с $K$ компонентами связности}

Основываясь на предыдущей формуле, научимся считать количество помеченных графов с $N$ вершинами и $K$ компонентами связности.

Сделать это можно с помощью динамического программирования. Научимся считать \bf{$D[N][K]$} --- количество помеченных графов с $N$ вершинами и $K$ компонентами связности.

Научимся вычислять очередной элемент $D[N][K]$, зная предыдущие значения. Воспользуемся стандартным приёмом при решении таких задач: возьмём вершину с номером 1, она принадлежит какой-то компоненте, вот эту компоненту мы и будем перебирать. Переберём размер $S$ этой компоненты, тогда количество способов выбрать такое множество вершин равно $C_{N-1}^{S-1}$ (одну вершину --- вершину 1 --- перебирать не надо). Количество же способов построить компоненту связности из $S$ вершин мы уже умеем считать --- это $Conn_S$. После удаления этой компоненты из графа у нас остаётся граф с $N-S$ вершинами и $K-1$ компонентами связности, т.е. мы получили рекуррентную зависимость, по которой можно вычислять значения $D[][]$:

$$ D[N][K] = \sum_{S=1}^{N} C_{N-1}^{S-1}\ Conn_S\ D[N-S][K-1] $$

Итого получаем примерно такой код:
\code
int d[n+1][k+1]; // изначально заполнен нулями
d[0][0][0] = 1;
for (int i=1; i<=n; ++i)
	for (int j=1; j<=i && j<=k; ++j)
		for (int s=1; s<=i; ++s)
			d[i][j] += C[i-1][s-1] * conn[s] * d[i-s][j-1];
cout << d[n][k][n];
\endcode

Разумеется, на практике, скорее всего, нужна будет \algohref=big_integer{длинная арифметика}.