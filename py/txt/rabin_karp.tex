<h1>Алгоритм Рабина-Карпа поиска подстроки в строке за O (N)</h1>

<p>Этот алгоритм базируется на хэшировании строк, и тех, кто не знаком с темой, отсылаю к <algohref=string_hashes>"Алгоритмам хэширования в задачах на строки"</algohref>.</p>

<p> </p>

<p>Авторы алгоритма - Рабин (Rabin) и Карп (Karp), 1987 год.</p>
<p>Дана строка S и текст T, состоящие из маленьких латинских букв. Требуется найти все вхождения строки S в текст T за время O (|S| + |T|).</p>
<p>Алгоритм. Посчитаем хэш для строки S. Посчитаем значения хэшей для всех префиксов строки T. Теперь переберём все подстроки T длины |S| и каждую сравним с |S| за время O (1).</p>
<h2>Реализация</h2>
<code>string s, t; // входные данные

// считаем все степени p
const int p = 31;
vector<long long> p_pow (max (s.length(), t.length()));
p_pow[0] = 1;
for (size_t i=1; i<p_pow.size(); ++i)
	p_pow[i] = p_pow[i-1] * p;

// считаем хэши от всех префиксов строки T
vector<long long> h (t.length());
for (size_t i=0; i<t.length(); ++i)
{
	h[i] = (t[i] - \'a\' + 1) * p_pow[i];
	if (i)  h[i] += h[i-1];
}

// считаем хэш от строки S
long long h_s = 0;
for (size_t i=0; i<s.length(); ++i)
	h_s += (s[i] - \'a\' + 1) * p_pow[i];

// перебираем все подстроки T длины |S| и сравниваем их
for (size_t i = 0; i + s.length() - 1 < t.length(); ++i)
{
	long long cur_h = h[i+s.length()-1];
	if (i)  cur_h -= h[i-1];
	// приводим хэши к одной степени и сравниваем
	if (cur_h == h_s * p_pow[i])
		cout << i << \' \';
}</code>