<h1>Пересечение двух окружностей</h1>

<p>Даны две окружности, каждая определена координатами своего центра и радиусом. Требуется найти все их точки пересечения (либо одна, либо две, либо ни одной точки, либо окружности совпадают).</p>
<h2>Решение</h2>
<p>Сведём нашу задачу к задаче о <b><algohref=circle_line_intersection>Пересечении окружности и прямой</algohref></b>.</p>
<p>Предположим, не теряя общности, что центр первой окружности - в начале координат (если это не так, то перенесём центр в начало координат, а при выводе ответа будем обратно прибавлять координаты центра). Тогда мы имеем систему двух уравнений:</p>
<formula>x<sup>2</sup> + y<sup>2</sup> = r<sub>1</sub><sup>2</sup>
(x - x<sub>2</sub>)<sup>2</sup> + (y - y<sub>2</sub>)<sup>2</sup> = r<sub>2</sub><sup>2</sup></formula>
<p>Вычтем из второго уравнения первое, чтобы избавиться от квадратов переменных:</p>
<formula>x<sup>2</sup> + y<sup>2</sup> = r<sub>1</sub><sup>2</sup>
x (-2x<sub>2</sub>) + y (-2y<sub>2</sub>) + (x<sub>2</sub><sup>2</sup> + y<sub>2</sub><sup>2</sup> + r<sub>1</sub><sup>2</sup> - r<sub>2</sub><sup>2</sup>) = 0</formula>
<p>Таким образом, мы свели задачу о пересечении двух окружностей к задаче о пересечении первой окружности и следующей прямой:</p>
<formula>Ax + By + C = 0,
A = -2x<sub>2</sub>,
B = -2y<sub>2</sub>,
C = x<sub>2</sub><sup>2</sup> + y<sub>2</sub><sup>2</sup> + r<sub>1</sub><sup>2</sup> - r<sub>2</sub><sup>2</sup>.</formula>
<p>А решение последней задачи описано в <algohref=circle_line_intersection>соответствующей статье</algohref>.</p>
<p>Единственный <b>вырожденный случай</b>, который надо рассмотреть отдельно - когда центры окружностей совпадают. Действительно, в этом случае вместо уравнения прямой мы получим уравнение вида 0 = С, где C - некоторое число, и этот случай будет обрабатываться некорректно. Поэтому этот случай нужно рассмотреть отдельно: если радиусы окружностей совпадают, то ответ - бесконечность, иначе - точек пересечения нет.</p>