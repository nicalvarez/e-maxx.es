\h1{ Префикс-функция. Алгоритм Кнута-Морриса-Пратта }


\h2{ Префикс-функция. Определение }

Дана строка $s[0 \ldots n-1]$. Требуется вычислить для неё префикс-функцию, т.е. массив чисел $\pi[0 \ldots n-1]$, где $\pi[i]$ определяется следующим образом: это такая наибольшая длина наибольшего собственного суффикса подстроки $s[0 \ldots i]$, совпадающего с её префиксом (собственный суффикс --- значит не совпадающий со всей строкой). В частности, значение $\pi[0]$ полагается равным нулю.

Математически определение префикс-функции можно записать следующим образом:

$$ \pi[i] = \max_{k=0 \ldots i} ~ \{ ~ k ~ : ~ s[0 \ldots k-1] = s[i-k+1 \ldots i] ~ \}. $$

Например, для строки "abcabcd" префикс-функция равна: $[0, 0, 0, 1, 2, 3, 0]$, что означает:
\ul{
\li у строки "a" нет нетривиального префикса, совпадающего с суффиксом;
\li у строки "ab" нет нетривиального префикса, совпадающего с суффиксом;
\li у строки "abc" нет нетривиального префикса, совпадающего с суффиксом;
\li у строки "abca" префикс длины $1$ совпадает с суффиксом;
\li у строки "abcab" префикс длины $2$ совпадает с суффиксом;
\li у строки "abcabc" префикс длины $3$ совпадает с суффиксом;
\li у строки "abcabcd" нет нетривиального префикса, совпадающего с суффиксом.
}

Другой пример --- для строки "aabaaab" она равна: $[0, 1, 0, 1, 2, 2, 3]$.


\h2{ Тривиальный алгоритм }

Непосредственно следуя определению, можно написать такой алгоритм вычисления префикс-функции:

\code
vector<int> prefix_function (string s) {
	int n = (int) s.length();
	vector<int> pi (n);
	for (int i=0; i<n; ++i)
		for (int k=0; k<=i; ++k)
			if (s.substr(0,k) == s.substr(i-k+1,k))
				pi[i] = k;
	return pi;
}
\endcode

Как нетрудно заметить, работать он будет за $O(n^3)$, что слишком медленно.


\h2{ Эффективный алгоритм }

Этот алгоритм был разработан Кнутом (Knuth) и Праттом (Pratt) и независимо от них Моррисом (Morris) в 1977 г. (как основной элемент для алгоритма поиска подстроки в строке).

\h3{ Первая оптимизация }

Первое важное замечание --- что значение $\pi[i+1]$ не более чем на единицу превосходит значение $\pi[i]$ для любого $i$.

Действительно, в противном случае, если бы $\pi[i+1] > \pi[i] + 1$, то рассмотрим этот суффикс, оканчивающийся в позиции $i+1$ и имеющий длину $\pi[i+1]$ --- удалив из него последний символ, мы получим суффикс, оканчивающийся в позиции $i$ и имеющий длину $\pi[i+1]-1$, что лучше $\pi[i]$, т.е. пришли к противоречию. Иллюстрация этого противоречия (в этом примере $\pi[i-1]$ должно быть равно 3):

$$ \underbrace{ \overbrace{s_0 \ s_1}^{\pi[i-1]=2} \ s_2 \ s_3}_{\pi[i]=4} \ \ldots\ \underbrace{ s_{i-3}\ \overbrace{s_{i-2}\ s_{i-1}}^{\pi[i-1]=2} \ s_i}_{\pi[i]=4} $$

(на этой схеме верхние фигурные скобки обозначают две одинаковые подстроки длины 2, нижние фигурные скобки --- две одинаковые подстроки длины 4)

Таким образом, при переходе к следующей позиции очередной элемент префикс-функции мог либо увеличиться на единицу, либо не измениться, либо уменьшиться на какую-либо величину. Уже этот факт позволяет нам снизить асимптотику до $O(n^2)$ --- поскольку за один шаг значение могло вырасти максимум на единицу, то суммарно для всей строки могло произойти максимум $n$ увеличений на единицу, и, как следствие (т.к. значение никогда не могло стать меньше нуля), максимум $n$ уменьшений. В итоге получится $O(n)$ сравнений строк, т.е. мы уже достигли асимптотики $O(n^2)$.

\h3{ Вторая оптимизация }

Пойдём дальше --- \bf{избавимся от явных сравнений подстрок}. Для этого постараемся максимально использовать информацию, вычисленную на предыдущих шагах.

Итак, пусть мы вычислили значение префикс-функции $\pi[i]$ для некоторого $i$. Теперь, если $s[i+1] = s[\pi[i]]$, то мы можем с уверенностью сказать, что $\pi[i+1] = \pi[i] + 1$, это иллюстрирует схема:

$$ \underbrace{ \overbrace{s_0 \ s_1 \ s_2}^{\pi[i]} \ \overbrace{s_3}^{s_3=s_{i+1}}}_{\pi[i+1]=\pi[i]+1} \ \ldots\ \underbrace{ \overbrace{s_{i-2}\ s_{i-1}\ s_i}^{\pi[i]} \ \overbrace{s_{i+1}}^{s_3=s_{i+1}}}_{\pi[i+1]=\pi[i]+1} $$

(на этой схеме снова одинаковые фигурные скобки обозначают одинаковые подстроки)

Пусть теперь, наоборот, оказалось, что $s[i+1] \ne s[\pi[i]]$. Тогда нам надо попытаться попробовать подстроку меньшей длины. В целях оптимизации хотелось бы сразу перейти к такой (наибольшей) длине $j < \pi[i]$, что по-прежнему выполняется префикс-свойство в позиции $i$, т.е. $s[0 \ldots j-1] = s[i-j+1 \ldots i]$:

$$ \overbrace{\underbrace{s_0 \ s_1}_{j} \ s_2 \ s_3}^{\pi[i]} \ \ldots\ \overbrace{ s_{i-3}\ s_{i-2} \underbrace{s_{i-1}\ s_{i}}_{j}}^{\pi[i]} \ s_{i+1} $$

Действительно, когда мы найдём такую длину $j$, то нам будет снова достаточно сравнить символы $s[i+1]$ и $s[j]$ --- если они совпадут, то можно утверждать, что $\pi[i+1] = j+1$. Иначе нам надо будет снова найти меньшее (следующее по величине) значение $j$, для которого выполняется префикс-свойство, и так далее. Может случиться, что такие значения $j$ кончатся --- это происходит, когда $j=0$. В этом случае, если $s[i+1]=s[0]$, то $\pi[i+1]=1$, иначе $\pi[i+1]=0$.

Итак, общая схема алгоритма у нас уже есть, нерешённым остался только вопрос об эффективном нахождении таких длин $j$. Поставим этот вопрос формально: по текущей длине $j$ и позиции $i$ (для которых выполняется префикс-свойство, т.е. $s[0 \ldots j-1] = s[i-j+1 \ldots i]$) требуется найти наибольшее $k < j$, для которого по-прежнему выполняется префикс-свойство:

$$ \overbrace{\underbrace{s_0 \ s_1}_{k} \ s_2 \ s_3}^{j} \ \ldots\ \overbrace{ s_{i-3}\ s_{i-2} \underbrace{s_{i-1}\ s_{i}}_{k}}^{j} \ s_{i+1} $$

После столь подробного описания уже практически напрашивается, что это значение $k$ есть не что иное, как значение префикс-функции $\pi[j-1]$, которое уже было вычислено нами ранее (вычитание единицы появляется из-за 0-индексации строк). Таким образом, находить эти длины $k$ мы можем за $O(1)$ каждую.

\h3{ Итоговый алгоритм }

Итак, мы окончательно построили алгоритм, который не содержит явных сравнений строк и выполняет $O(n)$ действий.

Приведём здесь итоговую схему алгоритма:

\ul{

\li Считать значения префикс-функции $\pi[i]$ будем по очереди: от $i=1$ к $i=n-1$ (значение $\pi[0]$ просто присвоим равным нулю).

\li Для подсчёта текущего значения $\pi[i]$ мы заводим переменную $j$, обозначающую длину текущего рассматриваемого образца. Изначально $j = \pi[i-1]$.

\li Тестируем образец длины $j$, для чего сравниваем символы $s[j]$ и $s[i]$. Если они совпадают --- то полагаем $\pi[i] = j+1$ и переходим к следующему индексу $i+1$. Если же символы отличаются, то уменьшаем длину $j$, полагая её равной $\pi[j-1]$, и повторяем этот шаг алгоритма с начала.

\li Если мы дошли до длины $j=0$ и так и не нашли совпадения, то останавливаем процесс перебора образцов и полагаем $\pi[i] = 0$ и переходим к следующему индексу $i+1$.
}

\h3{ Реализация }

Алгоритм в итоге получился удивительно простым и лаконичным:

\code
vector<int> prefix_function (string s) {
	int n = (int) s.length();
	vector<int> pi (n);
	for (int i=1; i<n; ++i) {
		int j = pi[i-1];
		while (j > 0 && s[i] != s[j])
			j = pi[j-1];
		if (s[i] == s[j])  ++j;
		pi[i] = j;
	}
	return pi;
}
\endcode

Как нетрудно заметить, этот алгоритм является \bf{онлайновым} алгоритмом, т.е. он обрабатывает данные по ходу поступления --- можно, например, считывать строку по одному символу и сразу обрабатывать этот символ, находя ответ для очередной позиции. Алгоритм требует хранения самой строки и предыдущих вычисленных значений префикс-функции, однако, как нетрудно заметить, если нам заранее известно максимальное значение, которое может принимать префикс-функция на всей строке, то достаточно будет хранить лишь на единицу большее количество первых символов строки и значений префикс-функции.


\h2{ Применения }


\h3{ Поиск подстроки в строке. Алгоритм Кнута-Морриса-Пратта }

Эта задача является классическим применением префикс-функции (и, собственно, она и была открыта в связи с этим).

Дан текст $t$ и строка $s$, требуется найти и вывести позиции всех вхождений строки $s$ в текст $t$.

Обозначим для удобства через $n$ длину строки $s$, а через $m$ --- длину текста $t$.

Образуем строку $s + \# + t$, где символ $\#$ --- это разделитель, который не должен нигде более встречаться. Посчитаем для этой строки префикс-функцию. Теперь рассмотрим её значения, кроме первых $n+1$ (которые, как видно, относятся к строке $s$ и разделителю). По определению, значение $\pi[i]$ показывает наидлиннейшую длину подстроки, оканчивающейся в позиции $i$ и совпадающего с префиксом. Но в нашем случае это $\pi[i]$ --- фактически длина наибольшего блока совпадения со строкой $s$ и оканчивающегося в позиции $i$. Больше, чем $n$, эта длина быть не может --- за счёт разделителя. А вот равенство $\pi[i] = n$ (там, где оно достигается), означает, что в позиции $i$ оканчивается искомое вхождение строки $s$ (только не надо забывать, что все позиции отсчитываются в склеенной строке $s+\#+t$).

Таким образом, если в какой-то позиции $i$ оказалось $\pi[i] = n$, то в позиции $i - (n + 1) - n + 1 = i - 2 n$ строки $t$ начинается очередное вхождение строки $s$ в строку $t$.

Как уже упоминалось при описании алгоритма вычисления префикс-функции, если известно, что значения префикс-функции не будут превышать некоторой величины, то достаточно хранить не всю строку и префикс-функцию, а только её начало. В нашем случае это означает, что нужно хранить в памяти лишь строку $s + \#$ и значение префикс-функции на ней, а потом уже считывать по одному символу строку $t$ и пересчитывать текущее значение префикс-функции.

Итак, алгоритм Кнута-Морриса-Пратта решает эту задачу за $O(n+m)$ времени и $O(n)$ памяти.


\h3{ Подсчёт числа вхождений каждого префикса }

Здесь мы рассмотрим сразу две задачи. Дана строка $s$ длины $n$. В первом варианте требуется для каждого префикса $s[0 \ldots i]$ посчитать, сколько раз он встречается в самой же строке $s$. Во втором варианте задачи дана другая строка $t$, и требуется для каждого префикса $s[0 \ldots i]$ посчитать, сколько раз он встречается в $t$.

Решим сначала первую задачу. Рассмотрим в какой-либо позиции $i$ значение префикс-функции в ней $\pi[i]$. По определению, оно означает, что в позиции $i$ оканчивается вхождение префикса строки $s$ длины $\pi[i]$, и никакой больший префикс оканчиваться в позиции $i$ не может. В то же время, в позиции $i$ могло оканчиваться и вхождение префиксов меньших длин (и, очевидно, совсем не обязательно длины $\pi[i]-1$). Однако, как нетрудно заметить, мы пришли к тому же вопросу, на который мы уже отвечали при рассмотрении алгоритма вычисления префикс-функции: по данной длине $j$ надо сказать, какой наидлиннейший её собственный суффикс совпадает с её префиксом. Мы уже выяснили, что ответом на этот вопрос будет $\pi[j-1]$. Но тогда и в этой задаче, если в позиции $i$ оканчивается вхождение подстроки длины $\pi[i]$, совпадающей с префиксом, то в $i$ также оканчивается вхождение подстроки длины $\pi[\pi[i]-1]$, совпадающей с префиксом, а для неё применимы те же рассуждения, поэтому в $i$ также оканчивается и вхождение длины $\pi[\pi[\pi[i]-1]-1]$ и так далее (пока индекс не станет нулевым). Таким образом, для вычисления ответа мы должны выполнить такой цикл:

\code
vector<int> ans (n+1);
for (int i=0; i<n; ++i)
	++ans[pi[i]];
for (int i=n-1; i>0; --i)
	ans[pi[i-1]] += ans[i];
\endcode

Здесь мы для каждого значения префикс-функции сначала посчитали, сколько раз он встречался в массиве $\pi[]$, а затем посчитали такую в некотором роде динамику: если мы знаем, что префикс длины $i$ встречался ровно ${\rm ans}[i]$ раз, то именно такое количество надо прибавить к числу вхождений его длиннейшего собственного суффикса, совпадающего с его префиксом; затем уже из этого суффикса (конечно, меньшей чем $i$ длины) выполнится "пробрасывание" этого количества к своему суффиксу, и т.д.

Теперь рассмотрим вторую задачу. Применим стандартный приём: припишем к строке $s$ строку $t$ через разделитель, т.е. получим строку $s+\#+t$, и посчитаем для неё префикс-функцию. Единственное отличие от первой задачи будет в том, что учитывать надо только те значения префикс-функции, которые относятся к строке $t$, т.е. все $\pi[i]$ для $i \ge n+1$.


\h3{ Количество различных подстрок в строке }

Дана строка $s$ длины $n$. Требуется посчитать количество её различных подстрок.

Будем решать эту задачу итеративно. А именно, научимся, зная текущее количество различных подстрок, пересчитывать это количество при добавлении в конец одного символа.

Итак, пусть $k$ --- текущее количество различных подстрок строки $s$, и мы добавляем в конец символ $c$. Очевидно, в результате могли появиться некоторые новые подстроки, оканчивавшиеся на этом новом символе $c$. А именно, добавляются в качестве новых те подстроки, оканчивающиеся на символе $c$ и не встречавшиеся ранее.

Возьмём строку $t = s + c$ и инвертируем её (запишем символы в обратном порядке). Наша задача --- посчитать, сколько у строки $t$ таких префиксов, которые не встречаются в ней более нигде. Но если мы посчитаем для строки $t$ префикс-функцию и найдём её максимальное значение $\pi_{\rm max}$, то, очевидно, в строке $t$ встречается (не в начале) её префикс длины $\pi_{\rm max}$, но не большей длины. Понятно, префиксы меньшей длины уж точно встречаются в ней.

Итак, мы получили, что число новых подстрок, появляющихся при дописывании символа $c$, равно $s.{\rm length}() + 1 - \pi_{\rm max}$.

Таким образом, для каждого дописываемого символа мы за $O(n)$ можем пересчитать количество различных подстрок строки. Следовательно, за $O(n^2)$ мы можем найти количество различных подстрок для любой заданной строки.

Стоит заметить, что совершенно аналогично можно пересчитывать количество различных подстрок и при дописывании символа в начало, а также при удалении символа с конца или с начала.


\h3{ Сжатие строки }

Дана строка $s$ длины $n$. Требуется найти самое короткое её "сжатое" представление, т.е. найти такую строку $t$ наименьшей длины, что $s$ можно представить в виде конкатенации одной или нескольких копий $t$.

Понятно, что проблема является в нахождении длины искомой строки $t$. Зная длину, ответом на задачу будет, например, префикс строки $s$ этой длины.

Посчитаем по строке $s$ префикс-функцию. Рассмотрим её последнее значение, т.е. $\pi[n-1]$, и введём обозначение $k = n - \pi[n-1]$. Покажем, что если $n$ делится на $k$, то это $k$ и будет длиной ответа, иначе эффективного сжатия не существует, и ответ равен $n$.

Действительно, пусть $n$ делится на $k$. Тогда строку можно представить в виде нескольких блоков длины $k$, причём, по определению префикс-функции, префикс длины $n-k$ будет совпадать с её суффиксом. Но тогда последний блок должен будет совпадать с предпоследним, предпоследний - с предпредпоследним, и т.д. В итоге получится, что все блоки блоки совпадают, и такое $k$ действительно подходит под ответ.

Покажем, что этот ответ оптимален. Действительно, в противном случае, если бы нашлось меньшее $k$, то и префикс-функция на конце была бы больше, чем $n-k$, т.е. пришли к противоречию.

Пусть теперь $n$ не делится на $k$. Покажем, что отсюда следует, что длина ответа равна $n$. Докажем от противного --- предположим, что ответ существует, и имеет длину $p$ ($p$ делитель $n$). Заметим, что префикс-функция необходимо должна быть больше $n - p$, т.е. этот суффикс должен частично накрывать первый блок. Теперь рассмотрим второй блок строки; т.к. префикс совпадает с суффиксом, и и префикс, и суффикс покрывают этот блок, и их смещение друг относительно друга $k$ не делит длину блока $p$ (а иначе бы $k$ делило $n$), то все символы блока совпадают. Но тогда строка состоит из одного и того же символа, отсюда $k=1$, и ответ должен существовать, т.е. так мы придём к противоречию.

$$ \overbrace{s_0\ s_1\ s_2\ s_3}^{p}\ \overbrace{s_4\ s_5\ s_6\ s_7}^{p} $$
$$ s_0\ s_1\ s_2\ \underbrace{\overbrace{s_3\ s_4\ s_5\ s_6}^{p}\ s_7}_{\pi[7]=5} $$
$$ s_4=s_3,\ \ s_5=s_4,\ \ s_6=s_5,\ \ s_7=s_6\ \ \ \ \Longrightarrow\ \ \ \ s_0=s_1=s_2=s_3 $$


\h3{ Построение автомата по префикс-функции }

Вернёмся к уже неоднократно использованному приёму конкатенации двух строк через разделитель, т.е. для данных строк $s$ и $t$ вычисление префикс-функции для строки $s+\#+t$. Очевидно, что т.к. символ $\#$ является разделителем, то значение префикс-функции никогда не превысит $s.{\rm length}()$. Отсюда следует, что, как упоминалось при описании алгоритма вычисления префикс-функции, достаточно хранить только строку $s+\#$ и значения префикс-функции для неё, а для всех последующих символов префикс-функцию вычислять на лету:

$$ \underbrace{s_0\ s_1\ \ldots\ s_{n-1}\ \#}_{\rm need\ to\ save} \underbrace{t_0\ t_1\ \ldots\ t_{m-1}}_{\rm need\ not\ to\ save} $$

Действительно, в такой ситуации, зная очередной символ $c \in t$ и значение префикс-функции в предыдущей позиции, можно будет вычислить новое значение префикс-функции, никак при этом не используя все предыдущие символы строки $t$ и значения префикс-функции в них.

Другими словами, мы можем построить \bf{автомат}: состоянием в нём будет текущее значение префикс-функции, переходы из одного состояния в другое будут осуществляться под действием символа:

$$ s_0\ s_1\ \ldots\ s_{n-1}\ \# \underbrace{\ldots}_{\pi[i-1]}\ \ \Longrightarrow\ \ s_0\ s_1\ \ldots\ s_{n-1}\ \# \underbrace{\ldots}_{\pi[i-1]} + t_i\ \ \Longrightarrow\ \ s_0\ s_1\ \ldots\ s_{n-1}\ \# \ldots \underbrace{t_i}_{\pi[i]} $$

Таким образом, даже ещё не имея строки $t$, мы можем предварительно построить такую таблицу переходов $({\rm old}\_\pi,c) \rightarrow {\rm new}\_\pi$ с помощью того же алгоритма вычисления префикс-функции:

\code
string s; // входная строка
const int alphabet = 256; // мощность алфавита символов, обычно меньше

s += '#';
int n = (int) s.length();
vector<int> pi = prefix_function (s);
vector < vector<int> > aut (n, vector<int> (alphabet));
for (int i=0; i<n; ++i)
	for (char c=0; c<alphabet; ++c) {
		int j = i;
		while (j > 0 && c != s[j])
			j = pi[j-1];
		if (c == s[j])  ++j;
		aut[i][c] = j;
	}
\endcode

Правда, в таком виде алгоритм будет работать за $O(n^2 k)$ ($k$ --- мощность алфавита). Но заметим, что вместо внутреннего цикла $\rm while$, который постепенно укорачивает ответ, мы можем воспользоваться уже вычисленной частью таблицы: переходя от значения $j$ к значению $\pi[j-1]$, мы фактически говорим, что переход из состояния $(j, c)$ приведёт в то же состояние, что и переход $(\pi[j-1], c)$, а для него ответ уже точно посчитан (т.к. $\pi[j-1] < j$):

\code
string s; // входная строка
const int alphabet = 256; // мощность алфавита символов, обычно меньше

s += '#';
int n = (int) s.length();
vector<int> pi = prefix_function (s);
vector < vector<int> > aut (n, vector<int> (alphabet));
for (int i=0; i<n; ++i)
	for (char c=0; c<alphabet; ++c)
		if (i > 0 && c != s[i])
			aut[i][c] = aut[pi[i-1]][c];
		else
			aut[i][c] = i + (c == s[i]);
\endcode

В итоге получилась крайне простая реализация построения автомата, работающая за $O(n k)$.

Когда может быть полезен такой автомат? Для начала вспомним, что мы считаем префикс-функцию для строки $s+\#+t$, и её значения обычно используют с единственной целью: найти все вхождения строки $s$ в строку $t$.

Поэтому самая очевидная польза от построения такого автомата --- \bf{ускорение вычисления префикс-функции} для строки $s+\#+t$. Построив по строке $s+\#$ автомат, нам уже больше не нужна ни строка $s$, ни значения префикс-функции в ней, не нужны и никакие вычисления --- все переходы (т.е. то, как будет меняться префикс-функция) уже предпосчитаны в таблице.

Но есть и второе, менее очевидное применение. Это случай, когда строка $t$ \bf{является гигантской строкой, построенной по какому-либо правилу}. Это может быть, например, строка Грея или строка, образованная рекурсивной комбинацией нескольких коротких строк, поданных на вход.

Пусть для определённости мы решаем \bf{такую задачу}: дан номер $k \le 10^5$ строки Грея, и дана строка $s$ длины $n \le 10^5$. Требуется посчитать количество вхождений строки $s$ в $k$-ю строку Грея. Напомним, строки Грея определяются таким образом:

$$ g_1 = "a" $$
$$ g_2 = "aba" $$
$$ g_3 = "abacaba" $$
$$ g_4 = "abacabadabacaba" $$
$$ \ldots $$

В таких случаях даже просто построение строки $t$ будет невозможным из-за её астрономической длины (например, $k$-ая строка Грея имеет длину $2^k-1$). Тем не менее, мы сможем посчитать значение префикс-функции на конце этой строки, зная значение префикс-функции, которое было перед началом этой строки.

Итак, помимо самого автомата также посчитаем такие величины: $G[i][j]$ --- значение автомата, достигаемое после "скармливания" ему строки $g_i$, если до этого автомат находился в состоянии $j$. Вторая величина --- $K[i][j]$ --- количество вхождений строки $s$ в строку $g_i$, если до "скармливания" этой строки $g_i$ автомат находился в состоянии $j$. Фактически, $K[i][j]$ --- это количество раз, которое автомат принимал значение $s.{\rm length}()$ за время "скармливания" строки $g_i$. Понятно, что ответом на задачу будет величина $K[k][0]$.

Как считать эти величины? Во-первых, базовыми значениями являются $G[0][j] = j$, $K[0][j] = 0$. А все последующие значения можно вычислять по предыдущим значениям и используя автомат. Итак, для вычисления этих значений для некоторого $i$ мы вспоминаем, что строка $g_i$ состоит из $g_{i-1}$ плюс $i$-ый символ алфавита плюс снова $g_{i-1}$. Тогда после "скармливания" первого куска ($g_{i-1}$) автомат перейдёт в состояние $G[i-1][j]$, затем после "скармливания" символа ${\rm char}_i$ он перейдёт в состояние:

$$ {\rm mid} = {\rm aut}[\ G[i-1][j]\ ][{\rm char}_i] $$

После этого автомату "скармливается" последний кусок, т.е. $g_{i-1}$:

$$ G[i][j] = G[i-1][{\rm mid}] $$

Количества $K[i][j]$ легко считаются как сумма количеств по трём кускам $g_i$: строка $g_{i-1}$, символ ${\rm char}_i$, и снова строка $g_{i-1}$:

$$ K[i][j] = K[i-1][j] + ({\rm mid} == s.{\rm length}()) + K[i-1][mid] $$

Итак, мы решили задачу для строк Грея, аналогично можно решить целый класс таких задач. Например, точно таким же методом решается \bf{следующая задача}: дана строка $s$, и образцы $t_i$, каждый из которых задаётся следующим образом: это строка из обычных символов, среди которых могут встречаться рекурсивные вставки других строк в форме $t_k[\rm cnt]$, которая означает, что в это место должно быть вставлено $\rm cnt$ экземпляров строки $t_k$. Пример такой схемы:

$$ t_1 = "abdeca" $$
$$ t_2 = "abc" + t_1[30] + "abd" $$
$$ t_3 = t_2[50] + t_1[100] $$
$$ t_4 = t_2[10] + t_3[100] $$

Гарантируется, что это описание не содержит в себе циклических зависимостей. Ограничения таковы, что если явным образом раскрывать рекурсию и находить строки $t_i$, то их длины могут достигать порядка $100^{100}$.

Требуется найти количество вхождений строки $s$ в каждую из строк $t_i$.

Задача решается так же, построением автомата префикс-функции, затем надо вычислять и добавлять в него переходы по целым строкам $t_i$. В общем-то, это просто более общий случай по сравнению с задачей о строках Грея.


\h2{ Задачи в online judges }

Список задач, которые можно решить, используя префикс-функцию:

\ul{

\li \href=http://uva.onlinejudge.org/index.php?option=onlinejudge&page=show_problem&problem=396{UVA #455 \bf{"Periodic Strings"} ~~~~ [сложность: средняя]}

\li \href=http://uva.onlinejudge.org/index.php?option=onlinejudge&page=show_problem&problem=1963{UVA #11022 \bf{"String Factoring"} ~~~~ [сложность: средняя]}

\li \href=http://uva.onlinejudge.org/index.php?option=onlinejudge&page=show_problem&problem=2447{UVA #11452 \bf{"Dancing the Cheeky-Cheeky"} ~~~~ [сложность: средняя]}

\li \href=http://acm.sgu.ru/problem.php?contest=0&problem=284{SGU #284 \bf{"Grammar"} ~~~~ [сложность: высокая]}

}
