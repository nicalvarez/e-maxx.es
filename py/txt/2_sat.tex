<h1>Задача 2-SAT</h1>
<p>Задача 2-SAT (2-satisfiability) - это задача распределения значений булевым переменным таким образом, чтобы они удовлетворяли всем наложенным ограничениям.</p>
<p>Задачу 2-SAT можно представить в виде конъюнктивной нормальной формы, где в каждом выражении в скобках стоит ровно по две переменной; такая форма называется 2-CNF (2-conjunctive normal form). Например:</p>
<formula>(a || c) && (a || !d) && (b || !d) && (b || !e) && (c || d)</formula>
<h2>Приложения</h2>
<p>Алгоритм для решения 2-SAT может быть применим во всех задачах, где есть набор величин, каждая из которых может принимать 2 возможных значения, и есть связи между этими величинами:</p>
<ul>
<li><b>Расположение текстовых меток на карте или диаграмме</b>.<br>Имеется в виду нахождение такого расположения меток, при котором никакие две не пересекаются.<br>Стоит заметить, что в общем случае, когда каждая метка может занимать множество различных позиций, мы получаем задачу general satisfiability, которая является NP-полной. Однако, если ограничиться только двумя возможными позициями, то полученная задача будет задачей 2-SAT.</li>
<li><b>Расположение рёбер при рисовании графа</b>.<br>Аналогично предыдущему пункту, если ограничиться только двумя возможными способами провести ребро, то мы придём к 2-SAT.</li>
<li><b>Составление расписания игр</b>.<br>Имеется в виду такая система, когда каждая команда должна сыграть с каждой по одному разу, а требуется распределить игры по типу домашняя-выездная, с некоторыми наложенными ограничениями.</li>
<li>и т.д.</li>
</ul>
<h2>Алгоритм</h2>
<p>Сначала приведём задачу к другой форме - так называемой импликативной форме. Заметим, что выражение вида a || b эквивалентно !a => b или !b => a. Это можно воспринимать следующим образом: если есть выражение a || b, и нам необходимо добиться обращения его в true, то, если a=false, то необходимо b=true, и наоборот, если b=false, то необходимо a=true.</p>
<p>Построим теперь так называемый <b>граф импликаций</b>: для каждой переменной в графе будет по две вершины, обозначим их через x<sub>i</sub> и !x<sub>i</sub>. Рёбра в графе будут соответствовать импликативным связям.</p>
<p>Например, для 2-CNF формы:</p>
<formula>(a || b) && (b || !c)</formula>
<p>Граф импликаций будет содержать следующие рёбра (ориентированные):</p>
<formula>!a => b
!b => a
!b => !c
c => b</formula>
<p>Стоит обратить внимание на такое свойство графа импликаций, что если есть ребро a => b, то есть и ребро !b => !a.</p>
<p>Теперь заметим, что если для какой-то переменной x выполняется, что из x достижимо !x, а из !x достижимо x, то задача решения не имеет. Действительно, какое бы значение для переменной x мы бы ни выбрали, мы всегда придём к противоречию - что должно быть выбрано и обратное ему значение. Оказывается, что это условие является не только достаточным, но и необходимым (доказательством этого факта будет описанный ниже алгоритм). Переформулируем данный критерий в терминах теории графов. Напомним, что если из одной вершины достижима другая, а из той вершины достижима первая, то эти две вершины находятся в одной сильно связной компоненте. Тогда мы можем сформулировать <b>критерий существования решения</b> следующим образом:</p>
<p>Для того, чтобы данная задача 2-SAT <b>имела решение</b>, необходимо и достаточно, чтобы для любой переменной x вершины x и !x находились <b>в разных компонентах сильной связности</b> графа импликаций.</p>
<p>Этот критерий можно проверить за время O (N + M) с помощью <algohref=strong_connected_components>алгоритма поиска сильно связных компонент</algohref>.</p>
<p>Теперь построим собственно <b>алгоритм</b> нахождения решения задачи 2-SAT в предположении, что решение существует.</p>
<p>Заметим, что, несмотря на то, что решение существует, для некоторых переменных может выполняться, что из x достижимо !x, или (но не одновременно), из !x достижимо x. В таком случае выбор одного из значений переменной x будет приводить к противоречию, в то время как выбор другого - не будет. Научимся выбирать из двух значений то, которое не приводит к возникновению противоречий. Сразу заметим, что, выбрав какое-либо значение, мы должны запустить из него обход в глубину/ширину и пометить все значения, которые следуют из него, т.е. достижимы в графе импликаций. Соответственно, для уже помеченных вершин никакого выбора между x и !x делать не нужно, для них значение уже выбрано и зафиксировано. Нижеописанное правило применяется только к непомеченным ещё вершинам.</p>
<p><b>Утверждается</b> следующее. Пусть comp[v] обозначает номер компоненты сильной связности, которой принадлежит вершина v, причём номера упорядочены в порядке топологической сортировки компонент сильной связности в графе компонентов (т.е. более ранним в порядке топологической сортировки соответствуют большие номера: если есть путь из v в w, то comp[v] <= comp[w]). Тогда, если comp[x] < comp[!x], то выбираем значение !x, иначе, т.е. если comp[x] > comp[!x], то выбираем x.</p>
<p><b>Докажем</b>, что при таком выборе значений мы не придём к противоречию. Пусть, для определённости, выбрана вершина x (случай, когда выбрана вершина !x, доказывается симметрично).</p>
<p>Во-первых, докажем, что из x не достижимо !x. Действительно, так как номер компоненты сильной связности comp[x] больше номера компоненты comp[!x], то это означает, что компонента связности, содержащая x, расположена левее компоненты связности, содержащей !x, и из первой никак не может быть достижима последняя.</p>
<p>Во-вторых, докажем, что никакая вершина y, достижимая из x, не является "плохой", т.е. неверно, что из y достижимо !y. Докажем это от противного. Пусть из x достижимо y, а из y достижимо !y. Так как из x достижимо y, то, по свойству графа импликаций, из !y будет достижимо !x. Но, по предположению, из y достижимо !y. Тогда мы получаем, что из x достижимо !x, что противоречит условию, что и требовалось доказать.</p>
<p>Итак, мы построили алгоритм, который находит искомые значения переменных в предположении, что для любой переменной x вершины x и !x находятся в разных компонентах сильной связности. Выше показали корректность этого алгоритма. Следовательно, мы одновременно доказали указанный выше критерий существования решения.</p>
<p>Теперь мы можем собрать <b>весь алгоритм</b> воедино:</p>
<ul>
<li>Построим граф импликаций.</li>
<li>Найдём в этом графе компоненты сильной связности за время O (N + M), пусть comp[v] - это номер компоненты сильной связности, которой принадлежит вершина v.</li>
<li>Проверим, что для каждой переменной x вершины x и !x лежат в разных компонентах, т.е. comp[x] ≠ comp[!x]. Если это условие не выполняется, то вернуть "решение не существует".</li>
<li>Если comp[x] > comp[!x], то переменной x выбираем значение true, иначе - false.</li>
</ul>
<h2>Реализация</h2>
<p>Ниже приведена реализация решения задачи 2-SAT для уже построенного графа импликаций g и обратного ему графа gt (т.е. в котором направление каждого ребра изменено на противоположное).</p>
<p>Программа выводит номера выбранных вершин, либо фразу "NO SOLUTION", если решения не существует.</p>
<code>int n;
vector < vector<int> > g, gt;
vector<bool> used;
vector<int> order, comp;

void dfs1 (int v) {
	used[v] = true;
	for (size_t i=0; i<g[v].size(); ++i) {
		int to = g[v][i];
		if (!used[to])
			dfs1 (to);
	}
	order.push_back (v);
}

void dfs2 (int v, int cl) {
	comp[v] = cl;
	for (size_t i=0; i<gt[v].size(); ++i) {
		int to = gt[v][i];
		if (comp[to] == -1)
			dfs2 (to, cl);
	}
}

int main() {
	... чтение n, графа g, построение графа gt ...

	used.assign (n, false);
	for (int i=0; i<n; ++i)
		if (!used[i])
			dfs1 (i);

	comp.assign (n, -1);
	for (int i=0, j=0; i<n; ++i) {
		int v = order[n-i-1];
		if (comp[v] == -1)
			dfs2 (v, j++);
	}

	for (int i=0; i<n; ++i)
		if (comp[i] == comp[i^1]) {
			puts ("NO SOLUTION");
			return 0;
		}
	for (int i=0; i<n; ++i) {
		int ans = comp[i] > comp[i^1] ? i : i^1;
		printf ("%d ", ans);
	}

}</code>