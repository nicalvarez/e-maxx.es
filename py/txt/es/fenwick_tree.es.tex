<h1>el Árbol de Fenwicks</h1>

<p><b>Árbol de Fenwicks</b> es una estructura de datos de árbol en la matriz, el cual posee las siguientes propiedades:</p>
<p>1) permite calcular el valor de cierta reversible de la operación de G en cualquier punto de [L; R] durante la <b>O (log N)</b>;</p>
<p>2) permite cambiar el valor de cualquier elemento de <b>O (log N)</b>;</p>
<p>3) requiere <b>O (N) de la memoria</b>, es decir, exactamente la misma cantidad de una matriz de N elementos;</p>
<p>4) fácilmente se resume en el caso de las matrices multidimensionales.</p>
<p>el Más común el uso de la madera Fenwicks - para calcular la suma en el tramo, es decir, la función G (X1, ..., Xk) = X1 + ... + Xk.</p>
<p>el Árbol de Fenwicks se describió por primera vez en el artículo "A new data structure for cumulative frequency tables" (Peter M. Fenwick, 1994).</p>
<h2>Descripción</h2>
<p>Para simplificar la descripción suponemos que la operación de G, por la que construimos el árbol es el <b>cantidad</b>.</p>
<p>Deje que dada una matriz A[0..N-1]. El árbol de Fenwicks - matriz de <b>T</b>, [0..N-1], en cada elemento que se almacena la suma de algunos de los elementos de la matriz A:</p>
<formula><b>T<sub>i</sub> = importe A<sub>j</sub></b> de los <b>F(i) <= j <= i</b></formula>
<p>donde F(i) es una función de la que definiremos más adelante.</p>
<p>Ahora ya podemos escribir <b>pseudocódigo</b> para la función de cálculo de la suma en el intervalo [0; R] y para la función de cambiar la celda:</p>
<code>int suma (int r)
{
int result = 0;
while (r >= 0) {
result += t[r];
r = f(r) - 1;
}
return result;
}

void inc (int i, int delta)
{
para todos los j, para los cuales F(j) <= i <= j
{
t[j] += delta;
}
}</code>
<p>la Función sum funciona de la siguiente manera. En vez de ir por todos los elementos de la matriz A, se mueve por la matriz T, haciendo "saltar" a través de cortes, donde esto sea posible. En primer lugar, se añade a la respuesta de la cantidad de valor en el intervalo [F(R); R] y, a continuación, toma la cantidad en el intervalo [F(F(R)-1); F(R)-1], y así sucesivamente, hasta que llega a cero.</p>
<p>la Función inc se mueve en la dirección opuesta - hacia el aumento de los índices mediante la actualización de los valores de la suma de T<sub>j</sub> sólo para las posiciones, para lo cual es necesario, es decir, para todo j, para los cuales F(j) <= i <= j.</p>
<p>es Evidente que, desde la selección de la función F dependerá de la velocidad de ejecución de ambas operaciones. Ahora vamos a ver una función que le permitirá alcanzar una productividad en ambos casos.</p>
<p><b>Definir valor de F(X)</b> de la siguiente manera. Veamos binaria para la grabación de este número, y mira a su bit menos significativo. Si es cero, entonces F(X) = X. de lo Contrario, la representación binaria de un número X termina en un grupo de una o varias unidades. Sustituir todas las unidades de este grupo en ceros, y asignamos el número resultante el valor de la función F(X).</p>
<p>Este es bastante compleja la descripción se corresponde muy sencilla fórmula:</p>
<formula><b>F(X) = X (X+1)</b></formula>
<p>donde & es una operación bit a bit lógico "Y".</p>
<p>es Fácil ver que esta fórmula corresponde словесному la descripción de la función dada por un superior.</p>
<p> </p>
<p>Nos quedan sólo aprender a encontrar rápidamente como el número de j para el que F(j) <= i <= j.</p>
<p>sin Embargo, es fácil asegurarse de que todos estos números j se obtienen de la i sucesivas sustituciones del derecho (el más pequeño) cero en representación binaria. Por ejemplo, para i = 10 obtenemos que j = 11, 15, 31, 63, etc.</p>
<p>aunque parezca extraño, este tipo de operación (sustitución del más joven de cero en la unidad también cumple muy sencilla fórmula:</p>
<formula><b>H(X) = X | (X+1)</b></formula>
<p>donde | es una operación bit a bit lógico "O".</p>
<h2>la Implementación de madera Fenwicks de la suma para el caso unidimensional</h2>
<code>vector<int> t;
int n;

void init (int nn)
{
n = nn;
t.assign (n, 0);
}

int sum (int r)
{
int result = 0;
for (; r >= 0; r = (r & (r+1)) - 1)
result += t[r];
return result;
}

void inc (int i, int delta)
{
for (; i < n; i = i | (i+1)))
t[i] += delta;
}

int sum (int l, int r)
{
return sum (r) - sum (l-1);
}

void init (vector<int> a)
{
init ((int) a.size());
for (unsigned i = 0; i < a.size(); i++)
inc (i, a[i]);
}</code>
<h2>la Implementación de madera Fenwicks para un mínimo para el caso unidimensional</h2>
<p>inmediatamente se Debe notar que, dado que el árbol de Fenwicks permite encontrar el valor de la función en cualquier intervalo [0;R], nosotros no seremos capaces de encontrar un mínimo en el intervalo [L, R], donde L > 0. Además, todos los valores deben ser sólo en sentido descendente (de nuevo, ya que no funciona llamar a la función min). Esto es una limitación importante.</p>
<code>vector<int> t;
int n;

const int INF = 1000*1000*1000;

void init (int nn)
{
n = nn;
t.assign (n, INF);
}

int getmin (int r)
{
int result = INF;
for (; r >= 0; r = (r & (r+1)) - 1)
result = min (result, t[r]);
return result;
}

void update (int i, int new_val)
{
for (; i < n; i = i | (i+1)))
t[i] = min (t[i], new_val);
}

void init (vector<int> a)
{
init ((int) a.size());
for (unsigned i = 0; i < a.size(); i++)
update (i, a[i]);
}</code>
<h2>la Implementación de madera Fenwicks de la suma para el caso bidimensional</h2>
<p>Como ya se ha señalado, el árbol de Fenwicks fácilmente se resume en el multidimensional de la funda.</p>
<code>vector <vector <int> > t;
int n, m;

int suma (int x, int y)
{
int result = 0;
for (int i = x; i >= 0; i = (i & (i+1)) - 1)
for (int j = y; j >= 0; j = (j & j+1)) - 1)
result += t[i][j];
return result;
}

void inc (int x, int y, int delta)
{
for (int i = x; i < n; i = i | (i+1)))
for (int j = y; j < m; j = (j | (j+1)))
t[i][j] += delta;
}</code>