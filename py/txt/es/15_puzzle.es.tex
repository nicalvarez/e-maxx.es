\h1{Juego de corre que te pillo: la existencia de la solución}

Recordemos que el juego es un campo de 4 $$ $4$, en el que se encuentran $15$ en fichas numeradas de 1 $a$ a $15$, y un campo se deja en blanco. Es necesario, moviendo a cada paso alguna ficha en la posición libre, para llegar finalmente a la siguiente posición:
$$ \matrix{
1&2&3&4\cr
5&6&7&8\cr
9&10&11&12\cr
13&14&15&\bigcirc\cr
} $$

Juego de corre que te pillo ("15 puzzle") se inventó en 1880 Нойес Чэпман (Noyes Chapman).

\h2{la Existencia de la solución}

Aquí vamos a ver una de ellas: de la posición en el tablero decir, si existe una secuencia de movimientos que conducen a la solución, o no.

Que dan una cierta posición en la pizarra:

$$ \matrix{
a_1 & a_2 & a_3 & a_4 \cr 
a_5 & a_6 & a_7 & a_8 \cr 
a_9 & a_{10} & a_{11} & a_{12} \cr 
a_{13} & a_{14} & a_{15} & a_{16} \cr 
} $$
donde uno de los elementos es igual a cero, indica una celda vacía de $a_z = 0$.

Considere una permutación:
$$ a_1 a_2 \ldots a_{z-1} a_{z+1} \sum_ a_{15} a_{16} $$
(es decir, la transposición de números correspondiente a la posición en el tablero, sin el cero del elemento)

Se denota por $N$ el número de reversiones de dicha en este movimiento (es decir, el número de tales elementos $a_i$ y $a_j$ que $i < j$ pero $a_i > a_j$).

A continuación, dejar que $K$ --- el número de la línea en la que se encuentra el elemento vacío (es decir, en nuestros leyenda de $K = (z-1)\ {\rm div}\ 4 + 1)$.

Entonces, \bf{existe una solución entonces, y sólo entonces, cuando $N+K$ uniforme}.

\h2{Aplicación}

Ilustraremos indicado anteriormente, el algoritmo mediante el software de código:
\code
int a[16];
for (int i=0; i<16; i++)
cin >> a[i];

int inv = 0;
for (int i=0; i<16; i++)
if (a[i])
for (int j=0; j<i; j++)
if (a[j] > a[i])
++inv;
for (int i=0; i<16; i++)
if (a[i] == 0)
inv += 1 + i / 4;

puts ((inv & 1) ? "No Solution" : "Solution Exists");
\endcode

\h2{Prueba}
Johnson (Johnson) en 1879 ha demostrado que si $N+K$ es impar, entonces no hay una solución a Story) en el mismo año ha demostrado que todas las posiciones, de los cuales $N+K$ uniforme, tienen solución.

Sin embargo, dos de estas pruebas eran bastante complejas.

En 1999, archer (arquero) propuso mucho más simple prueba (descargar su artículo \href=http://www.cs.cmu.edu/afs/cs/academic/class/15859-f01/www/notes/15-puzzle.pdf{aquí}).
