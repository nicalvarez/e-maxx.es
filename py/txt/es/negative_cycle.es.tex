\h1{Encontrar el negativo de un ciclo en el grafo}

Dan orientado ponderado conde de $G$ c $n$ cimas $m$ las costillas. Es necesario encontrar en él cualquier \bf{ciclo negativo de peso}, si es que lo hay.

Al otro planteamiento de la tarea --- es necesario encontrar \bf{todos los pares de vértices} tales que entre ellos hay un camino tantas de pequeña longitud.

Estas dos variantes de la tarea, es conveniente resolver algoritmos diferentes, por lo tanto, a continuación se tratarán dos de ellos.

Uno de los comunes "de la vida" producciones de esta tarea --- la siguiente: conocidos \bf{monedas}, es decir, cursos de traducción de una moneda a otra. Desea saber si una cierta secuencia de intercambios de obtener el beneficio, es decir, стартовав con una unidad de una moneda que hacer más de una unidad de la misma moneda.


\h2{la Solución mediante el uso del algoritmo de ford-bellman}

\algohref=ford_bellman{el Algoritmo de ford-bellman} permite comprobar la presencia o ausencia de un ciclo negativo de peso en la columna, y si lo hay --- encontrar a uno de esos ciclos.

No vamos a entrar aquí en los detalles (que se describen en el \algohref=ford_bellman{artículo del algoritmo de ford-bellman}), y a mencionar total --- cómo funciona el algoritmo.

Se hace $n$ iteraciones del algoritmo de ford-bellman, y si en la última iteración, no se han producido cambios-en-el - ciclo negativo en el recuadro no. En caso contrario, tomemos la parte superior, distancia a la que se ha cambiado, y vamos a ir de ella por los antepasados, hasta que entremos en un bucle; este ciclo y se buscaba negativo en el ciclo.

\bf{Aplicación}:

\code
struct edge {
int a, b, cost;
};

int n, m;
vector<edge> e;
const int INF = 1000000000;

void solve() {
vector<int> d (n);
vector<int> p (n, -1);
int x;
for (int i=0; i<n; ++i) {
x = -1;
for (int j=0; j<m; j++)
if (d[e[j].b] > d[e[j].a] + e[j].cost) {
d[e[j].b] = max (-INF, d[e[j].a] + e[j].cost);
p[e[j].b] = e[j].a;
x = e[j].b;
}
}

if (x == -1)
cout << "No negativo cycle found.";
else {
int y = x;
for (int i=0; i<n; ++i)
y = p[y];

vector<int> path.
for (int cur=y; ; cur=p[cur]) {
path.push_back (cur);
if (cur == y && path.size() > 1); break;
}
reverse path.begin(), path.end());

cout << "Negative cycle: ";
for (size_t i=0; i<path.size(); ++i)
cout << path[i] << ' ';
}
}
\endcode


\h2{Solución mediante un algoritmo de floyd-Уоршелла}

\algohref=floyd_warshall_algorithm{el Algoritmo de floyd-Уоршелла} permite resolver la segunda la creación de la tarea --- cuando es necesario encontrar todos los pares de vértices $(i,j)$, entre los cuales la ruta más corta no existe (es decir, tiene infinitamente pequeño valor).

Una vez más, explicaciones más detalladas, consulte \algohref=floyd_warshall_algorithm{descripción del algoritmo de floyd-Уоршелла} y aquí presentamos sólo el total.

Después de que el algoritmo de floyd-Уоршелла funcione para la entrada del conde, переберем todos los pares de vértices $(i,j)$, y para cada una de esas parejas comprobaremos que es infinitamente pequeño camino más corto de $i$ en $j$ o no. Para ello, переберем la tercera cima de $t$, y si para ella resultó $d[t][t]<0$ (es decir, se encuentra en un ciclo negativo de peso), y la misma es accesible de $i$ y a partir de ella es posible lograr $j$ --- la ruta $(i,j)$ puede tener infinitamente pequeña longitud.

\bf{Aplicación}:

\code
for (int i=0; i<n; ++i)
for (int j=0; j<n; ++j)
for (int t=0; t<n; ++t)
if (d[i][t] < INF && d[t][t] < 0 && d[t][j] < INF)
d[i][j] = -INF;
\endcode


\h2{Tareas en línea judges}

La lista de tareas en las que es necesario buscar un ciclo negativo de peso:

\ul{

\li \href=http://uva.onlinejudge.org/index.php?option=com_onlinejudge&Itemid=8&page=show_problem&problem=499{UVA #499 \bf{"Wormholes"} ~~~~ [dificultad: baja]}

\li \href=http://uva.onlinejudge.org/index.php?option=com_onlinejudge&Itemid=8&page=show_problem&problem=40{UVA #104 \bf{"Arbitrage"} ~~~~ [dificultad media]}

\li \href=http://uva.onlinejudge.org/index.php?option=com_onlinejudge&Itemid=8&page=show_problem&problem=1498{UVA #10557 \bf{"XYZZY"} ~~~~ [dificultad media]}

}