\h1{ Tarea de johnson con una máquina }

Esta es la tarea de elaboración de una óptima programación del procesamiento de $n$ de piezas en una máquina, si $i$-aja detalle es tratado en él por el tiempo $t_i$, y $t$ de segundos de espera antes de la elaboración de esta pieza, se cobrará una multa de $f_i(t)$.

Por lo tanto, la tarea consiste en la búsqueda de ese reorganizar las piezas que el siguiente valor (importe de la multa) es mínima. Si nos denota por $\pi$ permutación de las piezas ($\pi_1$ --- el primer número de la pieza, $\pi_2$ --- la segunda, y así sucesivamente), el tamaño de la multa de $f(\pi)$ es igual a:

$$ F(\pi) = f_{\pi_1}(0) + f_{\pi_2}(t_{\pi_1}) + f_{\pi_3}(t_{\pi_1} + t_{\pi_2}) + \ldots + f_{\pi_n}\left(\sum_{i=1}^{n-1} t_{\pi_i}\right). $$

A veces esta tarea se llama objetivo monoprocesador apropiado mantenimiento de la gran cantidad de solicitudes.


\h2{ la Solución de la tarea en algunos casos particulares }


\h3{ Primera privada de caso: lineal de la función de la multa }

Aprenderemos a hacer esta tarea, cuando todos los $f_i(t)$ son lineales, es decir, tienen la forma:

$$ f_i(t) = c_i \cdot t, $$

donde $c_i$ --- número no negativo. Tenga en cuenta que en estas lineales de las funciones de miembro libre es igual a cero, ya que en caso contrario, a la respuesta de inmediato se puede añadir a este miembro libre, y abordar la tarea de cero en un miembro gratis.

Introduzcamos algunos horarios --- permutación de $\pi$. Introduzcamos un número $i=1, \ldots, n-1$, y dejar que la transposición de $\pi^\prime$ es un movimiento de $\pi$, en el que cotizan $i$-ro y $i+1$de la realidad los elementos. Vamos a ver, ¿cuánto se si este ha cambiado la multa:

$$ F(\pi^\prime) - F(\pi) = $$

es fácil de entender, que los cambios se han producido con sólo $i$en la primera y $i+1$en la primera ingredientes básicos para lograr buenos:

$$ = c_{\pi^\prime_i} \cdot \sum_{k=1}^{i-1} t_{\pi^\prime_k} + c_{\pi^\prime_{i+1}} \cdot \sum_{k=1}^{i} t_{\pi^\prime_k} - c{\pi_i} \cdot \sum_{k=1}^{i-1} t_{\pi_k} - c_{\pi_{i+1}} \cdot \sum_{k=1}^{i} t_{\pi_k} = $$
$$ = c_{\pi_{i+1}} \cdot \sum_{k=1}^{i-1} t_{\pi^\prime_k} + c_{\pi_i} \cdot \sum_{k=1}^{i} t_{\pi_k^\prime} - c{\pi_i} \cdot \sum_{k=1}^{i-1} t_{\pi_k} - c{\pi_{i+1}} \cdot \sum_{k=1}^{i} t_{\pi_k} = $$
$$ = c_{\pi_i} \cdot t_{\pi_{i+1}} - c{\pi_{i+1}} \cdot t_{\pi_i}. $$

Está claro que si la programación de $\pi$ es el óptimo, cualquier cambio provoca un aumento de la multa (o a la conservación del valor anterior), por lo tanto, para optimizar el plan, puede grabar la condición:

$$ \forall i=1, \ldots, n-1 ~~~:~~ c_{\pi_i} \cdot t_{\pi_{i+1}} - c{\pi_{i+1}} \cdot t_{\pi_i} \ge 0. $$

Convirtiendo, obtenemos:

$$ \forall i=1, \ldots, n-1 ~~~:~~ \frac{ c{\pi_i} }{ t_{\pi_i} } \ge \frac{ c{\pi_{i+1}} }{ t_{\pi_{i+1}} }. $$

Por lo tanto, \bf{óptima de programación} se puede obtener simplemente \bf{ordenando} todos los detalles en relación con $c_i$ a $t_i$ en orden inverso.

Cabe señalar que hemos recibido este algoritmo llamado \bf{перестановочным recepción}: tratamos de cambiar asientos y dos vecinos elemento de programación, se han calculado, en la medida de si este ha cambiado la multa, y de ahí han sacado el algoritmo de búsqueda de la óptima programación.


\h3{ el Segundo caso especial: exponencial de la función de la multa }

Que ahora, la función de la multa tienen la forma:

$$ f_i(t) = c_i \cdot e^{\alpha \cdot t}, $$

donde todos los números $c_i$ неотрицательны, la constante $\alpha$ positiva.

Entonces, aplicando de manera similar aquí перестановочный la recepción, fácil de obtener que los detalles es necesario ordenar en orden descendente de magnitud:

$$ v_i = \frac{ 1 - e^{ \alpha \cdot t_i } }{ c_i }. $$


\h3{ el Tercer caso particular: la misma monótona de la función de la multa }

En este caso, se considera que de los $f_i(t)$ coinciden con alguna función $\phi(t)$, que es creciente.

Está claro que, en este caso de forma óptima colocación de las piezas en orden creciente de tiempo de procesamiento de $t_i$.


\h2{ Teorema de livshitsa-Кладова }

El teorema de livshitsa-Кладова establece que перестановочный recepción sólo se aplica a los anteriores tres casos particulares, y sólo ellos, es decir:

\ul{
\li caso Lineal: $f_i(t) = c_i \cdot t + d_i$, donde $c_i$ --- no negativo de la constante,
\li Exponencial de caso: $f_i(t) = c_i \cdot e^{\alpha \cdot t} + d_i$, donde $c_i$ y $\alpha$ --- una constante positiva,
\li Idéntico caso: $f_i(t) = \phi(t)$, donde $\phi$ --- de incremento de la función.
}

Este teorema es demostrado en el supuesto de que la función de la multa son lo suficientemente suaves (existen terceros derivados).

En los tres casos se aplica перестановочный recepción, gracias a la cual la búsqueda de la óptima programación se puede encontrar fácil de ordenación y, por lo tanto, por el tiempo $O (n \log n)$.

