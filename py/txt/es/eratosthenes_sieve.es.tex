\h1{ Tamiz eratosfena }

Tamiz eratosfena --- es un algoritmo que permite encontrar todos los primos en el tramo de $[1, n]$ a $O (n \log \log n)$ de operaciones.

La idea es simple --- escribir una serie de números de $1 \ldots, n$, y vamos a centrado primero todos los números que se dividen en $2$, excepto el número $de$ 2 y, a continuación, деляющиеся $3$, excepto el número $3$, luego $5$, luego $7$, $11$, y todos los demás simples a $n$.


\h2{ Aplicación }

En seguida presentamos la implementación de un algoritmo:

\code
int n;
vector<char> prime (n+1, true);
prime[0] = prime[1] = false;
for (int i=2; i<=n; ++i)
if (prime[i])
if (i * 1ll * i <= n)
for (int j=i*i; j<=n; j+=i)
prime[j] = false;
\endcode

Este código marca primero de todos los números, excepto el cero y la unidad, como simples y, a continuación, comienza el proceso de cribado de compuestos de números. Para ello, hemos перебираем en el ciclo de todos los números de 2 $a$ a $n$, y si el número actual de $i$ simple, помечаем todos los números múltiplos, como compuestos.

Cuando empezamos a caminar desde $i^2$, ya que todos menos el número de múltiplos de $i$, necesariamente tienen un divisor primo menor que $i$, lo que significa que ya estaban cribado antes. (Pero ya que $i^2$ fácilmente puede desbordar el tipo de $int$, en el código antes de la segunda adjunto ciclo se hace una comprobación adicional con el uso de un tipo $long~long$.)

Con esta implementación el algoritmo consume $O (n)$ de memoria (que es evidente) y realiza la $O (n \log \log n)$ de acción (esto se demuestra en la siguiente sección).


\h2{ Asíntotas }

Demostramos que asíntotas algoritmo es de O (n \log \log n)$.

Así, para cada una simple $p \le n$ se ejecutará el bucle interior, que hará de $\frac{n}{p}$ de acción. Por lo tanto, necesitamos estimar el siguiente valor:

$$ \sum_{ \scriptstyle p \le n, \atop\scriptstyle p~is~prime } \frac{n}{p} = n \cdot \sum_{ \scriptstyle p \le n, \atop\scriptstyle p~is~prime } \frac{1}{p}. $$

Recordemos aquí dos importantes hechos: que el número de los simples, de menor o de la igualdad de $n$ es aproximadamente igual a $\frac{n}{\ln n}$ y que $k$-tes simple número es aproximadamente igual a $k \ln k$ (esto se deduce de la primera aprobación). Entonces la suma se puede escribir así:

$$ \sum_{ \scriptstyle p \le n, \atop\scriptstyle p~is~prime } \frac{1}{p} \approx \frac{1}{2} + \sum_{k=2}^{\frac{n}{\ln n}} \frac{1}{k \ln k}. $$

Aquí, hemos identificado el primer sencillo de esta cantidad, ya que si $k = 1$ de acuerdo a aproximar $k \ln k$ resultará $0$, lo que llevará a la división por cero.

Ahora ponemos la misma cantidad con el desarrollo integral de la misma función de $k$ de $2 a$ a $\frac{n}{\ln n}$ (podemos producir esta aproximación, ya que, de hecho, la cantidad se refiere a la integral como una aproximación de la fórmula de los rectángulos):

$$ \sum_{k=2}^{\frac{n}{\ln n}} \frac{1}{k \ln k} \approx \int\limits_2^{\frac{n}{\ln n}} \frac{1}{k \ln k}\ dk. $$

Первообразная para подынтегральной de una función $\ln \ln k$. Realizando la sustitución y eliminación de los miembros de menor orden, obtenemos:

$$ \int\limits_2^{\frac{n}{\ln n}} \frac{1}{k \ln k}\ dk = \ln \ln \frac{n}{\ln n} - \ln \ln 2 = \ln (\ln n - \ln \ln n) - \ln \ln 2 \approx \ln \ln n. $$

Ahora, volviendo a la cantidad inicial, obtenemos su acercado evaluación:

$$ \sum_{ \scriptstyle p \le n, \atop\scriptstyle p~is~prime } \frac{n}{p} \approx n \ln \ln n + o(n) $$

que se quería demostrar.

Más estricta de la prueba (y le da una estimación más exacta con una precisión hasta de constantes multiplicador) se puede encontrar en el libro de Hardy y Wright "An Introduction to the Theory of Numbers" (p. 349).


\h2{ Diferentes optimizar el tamiz eratosfena }

El mayor inconveniente del algoritmo --- lo que se "pasea" por la memoria, la constante más allá de la memoria caché, por lo que es una constante, oculta en $O (n \log \log n)$, es relativamente grande.

Además, lo suficientemente grandes $n$ el cuello de botella se convierte en el consumo de memoria.

A continuación se describen los métodos, la posibilidad de reducir el número de operaciones y reducir considerablemente el consumo de memoria.


\h3{ Tamizado simples hasta la raíz }

La más obvia momento --- que, a fin de encontrar todas las cosas simples a $n$, basta con realizar el tamizado sólo simples, no superiores a raíz de $n$.

Por lo tanto, cambia el bucle exterior del algoritmo:

\code
for (int i=2; i*i<=n; ++i)
\endcode

En асимптотику esta optimización no afecta (de hecho, repitiendo la anterior prueba, obtenemos una estimación de $n \ln \ln \sqrt{n} + o(n)$, que, por las propiedades de los logaritmos, asintóticamente es lo mismo), aunque el número de operaciones visiblemente sucia.


\h3{ Tamiz sólo por una extraña números }

Ya que todos los números pares, además de los $2$, --- compuestas, puede no controlar de ningún modo, los números pares, y operar sólo números impares.

En primer lugar, esto permitirá reducir a la mitad la cantidad de memoria requerida. En segundo lugar, se reducirá el número de делаемых el algoritmo de las operaciones en aproximadamente la mitad.


\h3{ Reducción del volumen de consumo de memoria }

Tenga en cuenta que el algoritmo de eratosfena en realidad opera con $n$ bits de memoria. Por lo tanto, se puede ahorrar el consumo de memoria, almacenando el no $n$ bytes --- variables bulevskogo tipo, y $n$ bits, es decir, $n/8$ bytes de memoria.

Sin embargo, este enfoque --- \bf{"compresión de mapa de bits"} --- difícil la operación de estos bits. Cualquier lectura o escritura de un bit se presentar algunas de las operaciones aritméticas, lo que finalmente conduciría a la desaceleración del algoritmo.

Por lo tanto, este enfoque es adecuado sólo si $n$ es tan grande, que el $n$ bytes de memoria para asignar ya no se puede. Ahorro de memoria (en $8$ más), te pagaremos por esto es importante desaceleración en el algoritmo.

En conclusión, vale la pena señalar que en el lenguaje C++ se han aplicado los contenedores, automáticamente encargadas de bits compresión: vector<bool> y bitset<>. Sin embargo, si la velocidad de trabajo es muy importante, lo mejor es implementar bits compresión manual, con la ayuda de bit a bit --- hoy en día uno de los compiladores de los mismos no son capaces de generar suficiente código rápido.


\h3{ Colocar el tamiz }

De la optimización de la "tamizado simples hasta la raíz" que no hay necesidad de almacenar todo el tiempo todo el conjunto $prime[1 \ldots, n]$. Para realizar el tamizado suficiente para almacenar sólo simples hasta la raíz de $n$, es decir $prime[1 \ldots \sqrt{n}]$ y el resto de la matriz $prime$ a construir bloques, manteniendo en este momento, en un único bloque.

Deje que el $s$ --- una constante que define el tamaño de la unidad, entonces el total sería de $\left\lceil \frac{n}{s} \right\rceil$ bloques, $k$-primer bloque ($k = 0 \ldots \left\lfloor \frac{n}{s} \right\rfloor$) contiene los números en el tramo de $[ks; ks+s-1]$. Vamos a procesar bloques de la cola, es decir, por cada $k$del bloque vamos a ordenar a través de todas las cosas simples (de $1 a$ a $\sqrt{n}$) y realizar sus tamizado sólo dentro del bloque actual. Suavemente vale la pena procesar el primer bloque --- en primer lugar, simples de $[1; \sqrt{n}]$ no tiene que quitar a nosotros mismos, y en segundo lugar, el número de de $0$ y $de 1$ se marcarán como no simples. Cuando se procesa el último bloque también no se debe olvidar que el último número deseado de $n$ no necesariamente está en la final de la unidad.

Llevaremos la implementación del bloque de tamiz. El programa lee un número $n$ y encuentra una serie de sencillas desde $1 a$ a $n$:

\code
const int SQRT_MAXN = 100000; // la raíz de maximizar el valor de la N
const int S = 10000;
bool nprime[SQRT_MAXN], bl[S];
int primos[SQRT_MAXN], la cnt;

int main() {

int n;
cin >> n;
int nsqrt = (int) sqrt (n + .0);
for (int i=2; i<=nsqrt; ++i)
if (!nprime[i]) {
primos[cnt++] = i;
if (i * 1ll * i <= nsqrt)
for (int j=i*i; j<=nsqrt; j+=i)
nprime[j] = true;
}

int result = 0;
for (int k=0, maxk=n/S; k<=maxk; k++) {
memset (bl, 0, sizeof bl);
int start = k * S;
for (int i=0; i<cnt; ++i) {
int start_idx = (start + primos[i] - 1) / primos[i];
int j = max(start_idx,2) * primos[i] - start;
for (; j<S; j+=primos[i])
bl[j] = true;
}
if (k == 0)
bl[0] = bl[1] = true;
for (int i=0; i<S && start+i<=n; ++i)
if (!bl[i])
++result;
}
cout << result;

}
\endcode

Asíntotas de bloque tamiz mismo, como es normal y el tamiz eratosfena (si, por supuesto, el tamaño de la $s$ de los bloques no será muy pequeño), pero la cantidad de memoria que se reducirá a $O(\sqrt{n} + s)$, y en la menor "navegar" por la memoria. Pero, por otro lado, para cada bloque para cada simple de $[1; \sqrt{n}]$ se ejecutará la división, que será muy negativamente a menor tamaño de la unidad. Por tanto, la elección de la constante de $s$, debe lograrse un equilibrio.

Como muestran los experimentos, mejor velocidad de trabajo se logra cuando el $s$ es el valor de aproximadamente $10^4$ a $10^5$.


\h3{ Mejora de hasta línea de tiempo de trabajo }

El algoritmo de eratosfena se puede convertir en un algoritmo diferente, que ya va a funcionar por un tiempo lineal --- véase el artículo \algohref=prime_sieve_linear{"Tamiz eratosfena con un tiempo de trabajo"}. (Sin embargo, este algoritmo tiene y desventajas.)