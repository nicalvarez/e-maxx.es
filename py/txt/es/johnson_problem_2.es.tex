\h1{ Tarea de johnson con dos máquinas }

Hay $n$ de piezas y dos de la máquina. Cada detalle debe primero someterse a un tratamiento en la primera máquina y, a continuación, - - - en el segundo. Al $i$-aja detalle es tratado en la primera máquina por $a_i$ la hora, y la segunda - - - $b_i$ la hora. Cada máquina en cada momento sólo puede trabajar con una pieza.

Es necesario establecer un orden de presentación de las piezas en las máquinas, para un total al tiempo de procesamiento de todas las piezas, sería mínimo.

Esta tarea se denomina a veces la tarea de двухпроцессорного servicio de las tareas, o la tarea de johnson (bajo el nombre de S. M. Johnson, que en 1954 propuso un algoritmo para su solución).

Vale la pena señalar que cuando el número de máquinas de más de dos, esto es NP-completo (como ha demostrado gary (Garey en 1976).


\h2{ la Construcción de un algoritmo }

Tenga en cuenta, primero, que se puede considerar que el orden de procesamiento de las piezas \bf{en el primer y segundo máquinas debe coincidir}. En realidad, ya que los detalles para la segunda máquina sólo están disponibles después del tratamiento en la primera, y si hay varios disponibles para el segundo de la máquina de piezas el tiempo de procesamiento será igual a la suma de los $b_i$ independientemente de su orden --- es más rentable a enviar en la segunda máquina de la pieza de la que antes ha sido tratado en la primera máquina.

Veamos el orden de presentación de las piezas en las máquinas, coincidiendo con su entrada en el orden de: $1, 2, \ldots, n$.

Se denota por $x_i$ \bf{el tiempo de inactividad} segundo de la máquina justo antes de la tramitación de $i$-oh detalles (después del tratamiento $i-1$-oh de la pieza). Nuestro objetivo --- \bf{minimizar la suma simple}:

$$ F(x) = \sum x_i \longrightarrow \min. $$

Para la primera pieza tenemos:

$$ x_1 = a_1. $$

Para la segunda --- ya que se convierte listo para el envío en la segunda máquina en el momento $a_1+a_2$, y el segundo, la máquina se libera en el momento $x_1 + b_1$, entonces tenemos:

$$ x_2 = \max \Big( (a_1+a_2) - (x_1+b_1), 0 \Big). $$

La tercera pieza se convierte en disponible para la segunda de la máquina en el momento de la $a_1+a_2+a_3$, y la máquina se libera en $x_1+b_1+x_2+b_2$, por lo tanto:

$$ x_3 = \max \Big( (a_1+a_2+a_3) - (x_1+b_1+x_2+b_2), 0 \Big). $$

Por lo tanto, el aspecto general de $x_i$ se ve así:

$$ x_k = \max \left( \sum_{i=1}^{k} a_i - \sum_{i=1}^{k-1} b_i - \sum_{i=1}^{k-1} x_i, 0 \right). $$

Consideramos ahora \bf{la suma simple} $F(x)$. Se afirma que él es:

$$ F(x) = \max_{k=1 \ldots, n} K_i, $$

donde

$$ K_i = \sum_{i=1}^{k} a_i - \sum_{i=1}^{k-1} b_i. $$

(En esto se puede ver en la inducción, sucesivamente encontrar una expresión para la suma de los primeros dos, tres, y así sucesivamente $x_i$.)

Usaremos ahora \bf{перестановочным recepción}: vamos a tratar de canjear alguno de los dos vecinos de un elemento $j$ y $j+1$ y ver cómo cambia la suma simple.

Por la forma de la función de las expresiones de $K_i$ claro que cambiarán sólo $K_j$ y $K_{j+1}$; daremos el nombre de los nuevos valores a través de $K_j^\prime$ y $K_{j+1}^\prime$.

Por lo tanto, para pieza $j$ iba a pieza $j+1$, suficiente (aunque no necesario), para:

$$ \max \left( K_j, K_{j+1} \right) \le \max \left( K_j^\prime, K_{j+1}^\prime \right). $$

(es decir, hemos ignorado el resto, no han cambiado, los argumentos máximo en la expresión para $F(x)$, obteniendo lo suficiente, pero no es una condición necesaria, que el viejo $F(x)$ es menor o igual que el valor nuevo)

Restando $ \sum_{i=1}^{j+1} a_i - \sum_{i=1}^{j-1} b_i $ de ambas partes de la desigualdad, obtenemos:

$$ \max (-a_{j+1}, -b_j) \le \max (-b_{j+1}, -a_j), $$

o deshacerse de los números negativos, se obtiene:

$$ \min (a_j, b_{j+1}) \le \min (b_j, a_{j+1}). $$

Así, hemos recibido \bf{comparador}: ordenando la pieza de él, de nosotros, de acuerdo con la anterior выкладкам, llegamos a una óptima orden de las piezas, en la que no se puede intercambiar ninguna de las dos piezas, mejorando así el tiempo final.

Sin embargo, la mayor \bf{simplificar} la clasificación, si se mira en el comparador con la otra parte. De hecho, él nos dice que si un mínimo de cuatro números $(a_j, a_{j+1}, b_{j}, b_{j+1})$ se logra en un elemento de la matriz $a$, la pieza debe ir antes, y si en el elemento de la matriz $b$ --- más tarde. Así que tenemos otra forma de algoritmo que ordenar las piezas de un mínimo de $(a_i, b_i)$, y si el actual piezas mínimo es de $a_i$, entonces este artículo es necesario tratar la primera de las restantes, de otro modo --- de la última de las restantes.

De un modo o de otro, resulta que la tarea de johnson con dos máquinas se reduce a ordenar las piezas con una función de comparación de elementos. Por lo tanto, asíntotas de la solución es de $O (n \log n)$.


\h2{ Aplicación }

Realizamos la segunda opción se describe el algoritmo anterior, cuando las piezas se ordenan por un mínimo de $(a_i, b_i)$, y luego se envían al inicio o al final de la lista actual.

\code
struct item {
int a, b, id;

bool operator< (item p) const {
return min(a,b) < min(p.a,p.b);
}
};


sort (v.begin(), v.end());
vector<item> a, b;
for (int i=0; i<n; ++i)
(v[i].a<=v[i].b ? a : b) .push_back (v[i]);
a.insert (a.end(), b.rbegin(), b.rend());

int t1=0 y t2=0;
for (int i=0; i<n; ++i) {
t1 += a[i].a;
t2 = max(t2,t1) + a[i].b;
}
\endcode

Aquí todos los detalles se almacenan en forma de estructuras de $\rm item$, cada uno de los cuales contiene un valor de $a$ y $b$ y el número original de la pieza.

Las piezas se ordenan y, a continuación, se distribuyen en las listas de $a$ (estos son los detalles que se han enviado al principio de la cola) y $b$ (aquellos que fueron enviados a la final). Después de esto, dos de la lista se combinan (y de la segunda lista se toma en el orden inverso), y luego de ubicación del orden se calcula la búsqueda de tiempo mínimo: se admiten dos variables $t_1$ y $t_2$ --- tiempo de la liberación de la primera y de la segunda máquina, respectivamente.


\h2{ Literatura }

\ul{

\li \href=http://www.rand.org/pubs/papers/2008/P402.pdf{S. M. Johnson. \bf{Optimal two - and three-stage production schedules with setup times included} [1954]}

\li M. R. Garey. \bf{The Complexity of Flowshop and Jobshop Scheduling} [1976]

}