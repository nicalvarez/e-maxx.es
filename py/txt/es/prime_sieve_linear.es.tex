\h1{ Tamiz eratosfena con un tiempo de trabajo }

Dado el número de $n$. Es necesario encontrar \bf{todas las cosas simples} en el tramo de $[2, n]$.

La manera clásica de hacer frente a esta tarea --- \bf{\algohref=eratosthenes_sieve{tamiz eratosfena}}. Este algoritmo es muy simple, pero funciona por hora $O (n \log \log n)$.

Aunque en la actualidad sabemos bastante de los algoritmos que trabajan en el сублинейное tiempo (es decir, $o(n)$), descrito a continuación, el algoritmo es interesante su \bf{sencillez} --- casi no es más difícil de lo clásico tamiz eratosfena.

Además, el aquí, el algoritmo como un "efecto secundario" de hecho, calcula \bf{factorización de todos los números} en el tramo de $[2, n]$, lo que puede ser útil en muchos casos prácticos.

La desventaja del algoritmo es que se utiliza \bf{más memoria} que el clásico de tamiz eratosfena: se requiere tener, en conjunto, $n$ de números, mientras que el clásico de la решету eratosfena sólo $n$ bits de memoria (lo que resulta en $32$ veces menos).

Por lo tanto, se describe el algoritmo tiene sentido aplicar sólo a los números en el orden de $10^7$, no más.

La autoría del algoritmo, al parecer, pertenece a la Грайсу y misra (Gries, Misra, 1978 --- véase la lista de referencias al final). (Y, de hecho, decir que el algoritmo "решетом eratosfena" correctamente: demasiado la diferencia entre estas dos algoritmos.)


\h2{ Descripción del algoritmo }

Nuestro objetivo --- contar para cada número $i$ a en el tramo de $[2, n]$ su \bf{mínimo un divisor primo} $lp[i]$.

Además, necesitamos almacenar una lista de todos los números primos --- lo llamaremos matriz $pr[]$.

Inicialmente, todos los valores de $lp[i]$ completar con ceros, lo que significa que aún estamos asumiendo todos los números simples. Durante el funcionamiento del algoritmo de esta matriz se llenarse poco a poco.

Ahora vamos a escoger el número actual de $i$ de $2 a$ a $n$. Podemos tener dos casos:

\ul{

\li $lp[i] = 0$ --- esto significa que el número $i$ --- simple, porque para él, y no se han descubierto otros делителей.

Por lo tanto, es necesario asignar $lp[i] = i$ y agregar $i$ al final de la lista $pr[]$.

\li $lp[i] \ne 0$ --- esto significa que el número actual de $i$ --- compuesto, y un mínimo de un simple divisor es de $lp[i]$.

}

En ambos casos, entonces comienza el proceso de \bf{colocación de valores} en el array $lp[]$: vamos a tomar el número de, \bf{múltiplos} $i$, y actualizar el valor de $lp[]$. Sin embargo, nuestro objetivo --- aprender a hacerlo de esta manera, para llegar a cada número un valor de $lp[]$ sería instalado más de una vez.

Se afirma que, para ello, se puede proceder de esta manera. Considere el número de tipos:

$$ x_j = i \cdot p_j, $$

donde la secuencia de $p_j$ --- todo esto es simple, no superiores a $lp[i]$ (apenas para ello necesitamos almacenar la lista de todos los números primos).

Todos los números de este tipo de проставим el nuevo valor de $lp[x_j]$ --- obviamente, será igual a $p_j$.

¿Por qué este algoritmo es válido, y por qué funciona lineal tiempo --- ver más abajo, hasta que veamos su aplicación.


\h2{ Aplicación }

El tamiz se realiza antes de la fecha especificada en la constante del número de $N$.

\code
const int N = 10000000;
int lp[N+1];
vector<int> pr;

for (int i=2; i<=N; ++i) {
if (lp[i] == 0) {
lp[i] = i;
pr.push_back (i);
}
for (int j=0; j<(int)pr.size() && pr[j]<=lp[i] && i*pr[j]<=N; ++j)
lp[i * pr[j]] = pr[j];
}
\endcode

Esta aplicación poco se puede acelerar mediante la eliminación del vector de $pr$ (reemplazando a su normal en una matriz con contador), así como deshacerse de duplicado de la multiplicación en una subconsulta ciclo $for$ (para lo cual el resultado de la obra es necesario simplemente recordar en alguna variable).


\h2{ Prueba de la validez }

Demostremos \bf{corrección} el algoritmo, es decir, que se pone correctamente todos los valores de $lp[]$, y cada uno de ellos se determina exactamente una vez. De aquí se seguiría que funciona el algoritmo de tiempo lineal --- ya que todo el resto de los pasos del algoritmo, obviamente, trabajan por $O (n)$.

Para ello, tenga en cuenta que cualquier cantidad de $i$ \bf{la única representación} de esta forma:

$$ i = lp[i] \cdot x, $$

donde $lp[i]$ --- (como antes) un mínimo de un simple divisor de un número $i$, y el número de $x$, no tiene делителей menor de $lp[i]$, es decir:

$$ lp[i] \le lp[x]. $$

Ahora comparemos esto con lo que hace nuestro algoritmo --- que de hecho para cada $x$ itera sobre todas las cosas simples, en que se puede домножить, es decir, simples hasta $lp[x]$, ambos inclusive, para obtener el número anterior de la vista.

Por lo tanto, el algoritmo lugar realmente para cada compuesto de un número exactamente una vez, poniendo tiene el valor correcto es de $lp[]$.

Esto significa la corrección de un algoritmo y lo que funciona lineal del tiempo.


\h2{ Tiempo de trabajo y la memoria requerida }

Aunque asíntotas $O (n)$ mejor асимптотики $O (n \log \log n)$ clásico tamiz eratosfena, la diferencia entre ellas es pequeña. En la práctica, esto significa sólo двукратную la diferencia en la velocidad y optimizados opciones de tamiz eratosfena y no pierden de aquí al algoritmo.

Teniendo en cuenta los costos de la memoria que requiere este algoritmo --- matriz de números de $lp[]$ longitud $n$ y la matriz de todos los simples $pr[]$ longitud de aproximadamente $n / \ln n$ --- este algoritmo parece ser el segundo clásico de la решету todos los artículos.

Sin embargo, salva el hecho de que la matriz $lp[]$, calculado este algoritmo permite buscar factorización de cualquier número en el tramo de $[2, n]$ por el tiempo de la orden del tamaño de esta факторизации. Además, el precio de un adicional de la matriz se puede hacer para que en este факторизации no requería de una operación de división.

El conocimiento de la факторизации todos los números --- información muy útil para ciertas tareas, y de este algoritmo es uno de los pocos que permiten buscar por lineal del tiempo.


\h2{ Literatura }

\ul{

\li David Gries, Jayadev Misra. \bf{A Linear Sieve Algorithm for Finding Prime Numbers} [1978]

}