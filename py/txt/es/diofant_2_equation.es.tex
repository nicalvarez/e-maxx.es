\h1{ diofantovy las ecuaciones Lineales con dos variables }

Диофантово una ecuación con dos incógnitas tiene la forma:

$$ a \cdot x + b \cdot y = c, $$

donde $a, b, c$ --- definidas enteros, $x$ y $y$ --- desconocidos enteros.

A continuación se describen varios de los clásicos de tareas en estas ecuaciones: encontrar cualquier decisión, la obtención de todas las decisiones, la búsqueda de un número de decisiones y las soluciones en un determinado tramo de encontrar una solución con la menor suma de desconocidos.



\h2{ un caso de degeneración }

Uno de un caso de degeneración de inmediato nos excluiremos a partir de la revisión: cuando $a = b = 0$. En este caso, claro, la ecuación tiene una o un número infinito de las soluciones arbitrarias, o que no tiene soluciones en absoluto (dependiendo de $c = 0$ o no).



\h2{ Encontrar una solución }

Encontrar una solución диофантова ecuaciones con dos incógnitas mediante el \algohref=extended_euclid_algorithm{del algoritmo Extendido de euclides}. Supongamos, primero, que el número de $a$ y $b$ неотрицательны.

Un avanzado algoritmo de euclides a partir de un valor no negativo números de $a$ y $b$ encuentra el máximo común divisor de $g$, así como los factores de $x_g$ y $y_g$ que:

$$ a \cdot x_g + b \cdot y_g = g. $$

Se afirma que si $c$ divide a $g = {\rm mcd} (a,b)$, entonces диофантово la ecuación $a \cdot x + b \cdot y = c$ tiene la solución; en caso contrario, диофантово la ecuación de decisiones no tiene. La prueba debe de hecho obvio de que la combinación lineal de dos números todavía tiene que compartir su común divisor.

Supongamos que $c$ divide a $g$, entonces, obviamente, se realiza:

$$ a \cdot x_g \cdot (c/g) + b \cdot y_g \cdot (c/g) = c, $$

es decir, una de las soluciones диофантова de la ecuación son los números:

$$ \cases{
x_0 = x_g \cdot (c / g), \cr
y_0 = y_g \cdot (c / g).
} $$

Hemos descrito la decisión en el caso, cuando el número de $a$ y $b$ неотрицательны. Si uno de ellos o ambos son negativos, entonces se puede proceder de esta manera: tomar por el módulo y aplicar el algoritmo de euclides, como se ha descrito anteriormente y, a continuación, cambiar el signo de los encontrados en $x_0$ y $y_0$ de conformidad con la presente el signo de los números $a$ y $b$, respectivamente.

La implementación de la (recordemos aquí creemos que los datos de entrada $a=b=0$ no válidas):

\code
int mcd (int a, int b, int & x, int & y) {
if (a == 0) {
x = 0; y = 1;
return b;
}
int x1, y1;
int d = gcd (b%a, a, x1, y1);
x = y1 - (b / a) * x1;
y = x1;
return d;
}

bool find_any_solution (int a, int b, int c, int & x 0, int & y0, int & g) {
g = gcd (abs(a), abs(b), (x0, y0);
if (c % g != 0)
return false;
x0 *= c / g;
y0 *= c / g;
if (a < 0) x0 *= -1;
if (b < 0) y0 *= -1;
return true;
}
\endcode



\h2{ Obtener todas las soluciones }

Le mostraremos cómo obtener el resto de la solución (y sus infinitas) диофантова de la ecuación, sabiendo que una de las soluciones de $(x_0,y_0)$.

Así, que $g = {\rm mcd}(a,b)$, y la cantidad de $x_0, y_0$ cumplen la condición:

$$ a \cdot x_0 + b \cdot y_0 = c. $$

Entonces tenga en cuenta que, añadiendo a $x_0$ número $b/g$ y al mismo tiempo destruyendo $a/g$ de $y_0$ no violaremos de la igualdad:

$$ a \cdot (x_0 + b/g) + b \cdot (y_0 - a/g) = a \cdot x_0 + b \cdot y_0 + a \cdot b/g - b \cdot a/g = c. $$

Es evidente que este proceso se puede repetir todo lo que quieras, es decir, todos los números de la especie:

$$ \cases{
x = x_0 + k \cdot b/g, \cr
y = y_0 - k \cdot a/g,
} ~~~~ k \in Z $$

son soluciones de диофантова de la ecuación.

Además, sólo en el número de este tipo y son las soluciones, es decir, describimos el conjunto de todas las soluciones диофантова de la ecuación (que ha resultado infinito, si no se imponen condiciones adicionales).



\h2{ Encontrar el número de decisiones y las soluciones en un determinado tramo }

Que dados dos segmentos de $[min_x;max_x]$ y $[min_y;max_y]$, y es necesario encontrar el número de soluciones de $(x,y)$ диофантова las ecuaciones subyacentes en los datos de los segmentos, respectivamente.

Tenga en cuenta que si uno de los números de $a,$ b es igual a cero, significa que la tarea no tiene más de una solución, por lo tanto, estos casos estamos en esta sección excluye de la consideración.

Primero encontraremos la solución con un mínimo adecuado de $x$, es decir $x \ge min_x$. Para ello, primero encontraremos cualquier decisión диофантова la ecuación (véase el párrafo 1). A continuación, la recibiremos de él la solución con el menor $x \ge min_x$ --- para eso utilizaremos el procedimiento descrito en el párrafo anterior, y vamos a reducir/aumentar de $x$, mientras que no sea $\ge min_x$, y el mínimo. Esto se puede hacer por $O(1)$, considerando con qué factor debe aplicar esta transformación, para obtener el número mínimo mayor o igual a $min_x$. Se denota encontrado $x$ $lx1$.

De manera similar se puede encontrar una solución con la máxima adecuado $x=rx1$, es decir $x \le max_x$.

A continuación, desplácese a la satisfacción de las restricciones de $y$, es decir, a la revisión del corte de $[min_y;max_y]$. El modo descrito más arriba, encontraremos la solución con un mínimo de $y \ge min_y$, así como la decisión de, con un máximo de $y \le max_y$. Se denota $x$-factores de estas soluciones a través de $lx2$ y $rx2$, respectivamente.

Recorren tramos de $[lx1;rx1]$ y $[lx2;rx2]$; denota el tramo de vía $[lx;rx]$. Se afirma que cualquier decisión que $x$es el factor está en $[lx;rx]$ --- esa decisión es el adecuado. (Esto es cierto en vigor de la construcción de este segmento: primero hemos separado satisfecho restricciones en $x$ y $y$, obteniendo dos tintas, y luego cruzaron, habiendo recibido el ámbito en el que se satisfacen las dos condiciones.)

Por lo tanto, la cantidad de decisiones será igual a la longitud de este segmento dividido por $|b|$ (ya que $x$es el coeficiente puede variar sólo en $\pm b$), y más uno.

Presentamos la aplicación (que resultó bastante complicado, ya que es necesario considerar cuidadosamente los casos positivos y negativos de los factores de $a$ y $b$):

\code
void shift_solution (int & x, int & y, int a, int b, int cnt) {
x += cnt * b;
 y -= cnt * a;
}

int find_all_solutions (int a, int b, int c, int minx, int maxx, int miny, int maxy) {
int x, y, g;
if (! find_any_solution (a, b, c, x, y, g))
return 0;
a /= g; b /= g;

int sign_a = a>0 ? +1 : -1;
int sign_b = b>0 ? +1 : -1;

shift_solution (x, y, z, a, b, (minx - x) / b);
if (x < minx)
shift_solution (x, y, z, a, b, sign_b);
if (x > maxx)
return 0;
int lx1 = x;

shift_solution (x, y, z, a, b, (maxx - x) / b);
if (x > maxx)
shift_solution (x, y, z, a, b, -sign_b);
int rx1 = x;

shift_solution (x, y, z, a, b, - (miny - y) / a);
if (y < miny)
shift_solution (x, y, z, a, b, -sign_a);
if (y > maxy)
return 0;
int lx2 = x;

shift_solution (x, y, z, a, b, - (maxy - y) / a);
if (y > maxy)
shift_solution (x, y, z, a, b, sign_a);
int rx2 = x;

if (lx2 > rx2)
swap (lx2, rx2);
int lx = max (lx1, lx2);
int rx = min (rx1, rx2);

return (rx - lx) / abs(b) + 1;
}
\endcode

También es fácil de añadir a la aplicación de la retirada de todas las soluciones identificadas para ello, basta con recorrer la $x$ en el tramo de $[lx;rx]$ con un incremento de $|b|$, encontrando para cada uno de ellos correspondiente a $y$ directamente de la ecuación $ax+by=c$.


\h2{ de Encontrar una solución en un tramo con la menor suma de x+y }

Aquí en $x$ y $y$ también se deberían imponer ningún límite, de lo contrario, la respuesta casi siempre es menos infinito.

La idea de la solución es la misma que en el párrafo anterior: primero encontramos cualquier decisión диофантова de la ecuación y, a continuación, aplicando el descrito en el párrafo anterior procedimiento, llegar a la mejor solución.

Realmente, tenemos el derecho de realizar la siguiente conversión (véase el apartado anterior):

$$ \cases{
x^\prime = x + k \cdot (b/g), \cr
y^\prime = y - k \cdot (a/g),
} ~~~~ k \in Z.$$

Tenga en cuenta que la suma de $x+y$ varía de la siguiente manera:

$$x^\prime + y^\prime = x + y + k \cdot b/g a a/g) = x + y + k \cdot (b - a) / g.$$

Es decir, si $a < b$, debe elegir el menor valor de $k$, si $a > $ b, debe seleccionar un valor de $k$.

Si $a = b$, entonces nosotros no podemos mejorar la solución --- todas las decisiones tendrán la misma cantidad.



\h2{ Tareas en línea judges }

La lista de tareas que se pueden dar en el tema de la диофантовых de ecuaciones con dos incógnitas:

\ul{

\li \href=http://acm.sgu.es/problem.php?contest=0&problem=106{SGU #106 \bf{"The Equation} ~~~~ [dificultad media]}

}
