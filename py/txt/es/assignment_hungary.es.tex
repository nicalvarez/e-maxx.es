\h1{ Húngaro el algoritmo de solución de la tarea acerca de las asignaciones }



\h2{ Planteamiento del problema acerca de las asignaciones }

La tarea de las asignaciones se pone muy natural.

A continuación se presentan varios \bf{opciones planteamiento} (como es fácil de ver, todos ellos son equivalentes entre sí):

\ul{

\li. $n$ trabajo $y $ n$ de trabajos. Para cada área de trabajo se sabe cuánto dinero se le preguntará por la ejecución de un trabajo. Cada equipo de trabajo puede tomar sólo un trabajo. Es necesario distribuir las tareas de trabajo para minimizar los costos totales.

\li Dada la matriz $a$ el tamaño de la $n \times n$. Se requiere en cada fila seleccionar un número para que en cualquier columna, también fue elegido exactamente por el mismo número, y la suma de números seleccionados sería mínima.

\li Dada la matriz $a$ el tamaño de la $n \times n$. Es necesario encontrar una permutación de $p$ de longitud $n$, que es el valor de $\sum a[i][p[i]]$ --- es mínima.

\li Dan completa двудольный el conde con $n$ vértices; cada eje de adscripción algo de peso. Se desea encontrar el perfecto паросочетание de peso mínimo.

}

Obsérvese que todas estas obras "\bf{aplicados populares son cuadrados}": en ambas dimensiones siempre coinciden (son iguales a $n$). En la práctica, a menudo se encuentran similares "\bf{rectangulares}" la puesta en escena, cuando $n \ne m$, y hay que elegir $\min(n,m)$ de los elementos. Sin embargo, como es fácil de ver, de "rectangular" de la tarea siempre puedes ir a "cuadrado", añadiendo filas/columnas con cero/un sinfín de valores, respectivamente.

También tenga en cuenta que, por analogía con la búsqueda de \bf{mínimo} solución también se puede poner la tarea de búsqueda \bf{máximo} solución. Sin embargo, estas dos tareas son equivalentes: basta con todo el peso multiplicado por $-1$.



\h2{ el algoritmo Húngaro }


\h3{ Antecedentes }

El algoritmo fue desarrollado y publicado por harold \bf{Куном} (Harold Kuhn) en 1955, El kun le dio el nombre de algoritmo de "el húngaro", porque él fue, en gran medida, se basa en estudios anteriores de los dos húngaros matemáticos: Денеша \bf{koenig} (Dénes Kőnig) y Эйгена \bf{Эгервари} (Jenő Egerváry).

En 1957, james \bf{Манкрес} (James Munkres) ha demostrado que este algoritmo funciona (estrictamente) полиномиальное tiempo (es decir, durante el tiempo de la orden polinómica de $n$, no depende de la magnitud de valores).

Por lo tanto, en la literatura de este algoritmo es conocido no sólo como "el húngaro", sino como "el algoritmo de kuhn-Манкреса" o "algoritmo de Манкреса".

Sin embargo, recientemente (2006) reveló que exactamente el mismo algoritmo se ha inventado \bf{en un siglo antes de kuna} el matemático alemán carl gustav \bf{jacobi} (Carl Gustav Jacobi). El hecho de que su trabajo "About the research of the order of a system of arbitrary ordinary differential equations", impreso a título póstumo en el año 1890, en la que se determinaron, además de otros resultados y полиномиальный el algoritmo de solución de la tarea sobre los nombramientos, fue escrita en latín, y su publicación \bf{ha pasado desapercibido} entre los matemáticos.

También vale la pena señalar que el original, el algoritmo de kuhn tenía асимптотику $O(n^4)$, y sólo más tarde, jack \bf{edmonds} (Jack Edmonds) y richard \bf{Carpa} (Richard Karp) (independientemente \bf{Томидзава} (Tomizawa)) han demostrado cómo mejorar su hasta асимптотики $O(n^3)$.


\h3{ la Construcción de un algoritmo por el $O(n^4)$ }

Ten en cuenta para evitar ambigüedades, que principalmente nos referimos en esta tarea acerca de las asignaciones en la matriz de planteamiento (es decir, dada una matriz $a$, y es necesario seleccionar de ella $n$ celdas que se encuentran en diferentes filas y columnas). La indexación de matrices comenzamos con la unidad, es decir, por ejemplo, la matriz $a$ con índices de $a[1 \ldots, n][1 \ldots, n]$.

También vamos a suponer que todos los números en la matriz $a[][]$ \bf{неотрицательны} (si no es así, siempre puedes ir a неотрицательной matriz, añadiendo a todos los números un número).

Llamaremos \bf{potencial} dos arbitrarias de la matriz de números de $u[1 \ldots, n]$ y $v[1 \ldots, n]$ tales que se cumple la condición:

$$ u[i] + v[j] \le de a[i][j] ~~~~ (i = 1, \ldots, n, ~~ j = 1 \ldots, n). $$

(Como se ve, el número de $u[i]$ corresponden a las filas, y la cantidad de $v[j]$ --- las columnas de la matriz.)

Llamaremos \bf{un valor de $f$ la capacidad de} la suma de sus números:

$$ f = \sum_{i=1}^n u[i] + \sum_{i=1}^n v[i]. $$

Por un lado, es fácil observar que el costo de la búsqueda de la solución $sol$ \bf{no menos} el valor de cualquier capacidad de:

$$ sol \ge f. $$

(La prueba. La búsqueda de la solución de la tarea es de $n$ las celdas de la matriz, y para cada uno de ellos se cumple la condición $u[i] + v[j] \le de a[i][j]$. Dado que todos los elementos se encuentran en diferentes filas y columnas, lo sumando estas desigualdades en todos seleccionados $a[i][j]$, en la parte izquierda de la desigualdad obtenemos $f$, y en la derecha --- $sol$, que se quería demostrar.)

Por otro lado, resulta que siempre hay una solución y el potencial, que es la desigualdad \bf{se dirige a la igualdad}. El húngaro, el algoritmo se describe a continuación, será constructiva de la prueba de este hecho. Mientras tanto, sólo prestemos atención al hecho de que, si una solución tiene un costo igual a la más grande de cualquier capacidad, entonces esta solución --- \bf{óptimo}.

Introduzcamos algún potencial. Llamaremos a la arista $(i,j)$ \bf{duro}, si se cumple:

$$ u[i] + v[j] = a[i][j]. $$

Recordemos sobre la alternativa planteamiento de la tarea sobre los nombramientos, con la ayuda de двудольного conde. Se denota por $H$ двудольный conde, compuesto solamente de las costillas. De hecho, el húngaro, el algoritmo admite para las capacidades actuales \bf{máximo por la cantidad de aristas паросочетание $M$} conde de $H$: y como sólo es паросочетание será contener $n$ costillas, costillas de este паросочетания y será buscamos la mejor solución (ya que esto será una solución cuyo valor coincide con el valor de la capacidad).

Pasaremos directamente a \bf{descripción del algoritmo}.

\ul{

\li En el comienzo del algoritmo de la capacidad se basa en cero $u[i] = v[i] = 0$, y паросочетание $M$ confía en blanco.

\li Sucesivamente, en cada paso del algoritmo tratamos de no cambiar de capacidad, aumentar la potencia de la actual паросочетания $M$ por unidad (recordamos паросочетание que se busca en la columna duros de las aletas $H$).

Para ello, se utiliza efectivamente normal \algohref=kuhn_matching{el algoritmo de kuhn de la búsqueda de la máxima паросочетания en dicotiledóneas columnas}. Recordemos aquí este algoritmo.

Todas las aristas de паросочетания $M$ se orientan en la dirección de la segunda cuota a la primera, el resto de la aleta del conde de $H$ se orientan en la dirección opuesta.

Recordemos (de la terminología de la búsqueda паросочетаний), que la punta de la llama saturada, si le смежно el borde de la actual паросочетания. La cima, que no смежно ninguna arista de la actual паросочетания, se denomina poco. La ruta impar de longitud, en el que la primera costilla no pertenece паросочетанию, y para todas las costillas se produce la rotación (pertenece/no pertenece) --- se llama amplían el camino.

De todos los insaturados de los vértices de la primera cuota se inicia el rastreo \algohref=dfs{en profundidad}/\algohref=bfs{ancho}. Si como resultado de rastreo pudo alcanzar un poco de la cima de la segunda fracción, esto significa que hemos encontrado que aumenta la ruta de acceso de la primera fracción de segundo. Si прочередовать las costillas a lo largo de este camino (es decir, el primer borde incluir en паросочетание, la segunda excluir, la tercera incluir, etc.), lo que habríamos potencia паросочетания en la unidad.

Si reducen el camino, no se, esto significa que el actual паросочетание $M$ --- máximo en la columna de la $H$, por lo tanto, en tal caso, pasamos al siguiente punto.

\li Si en el paso actual no se puede aumentar la potencia de la actual паросочетания, se realiza un recuento de la capacidad de tal manera que en los pasos siguientes, hay más oportunidades para aumentar la паросочетания.

Se denota por $Z_1$ muchos de los vértices de la primera cuota, que fueron visitados por omisión de un algoritmo de kuhn intenta buscar lo que aumenta la cadena; a través de $Z_2$ --- muchas visitados por los vértices de la segunda parte.

Calcular la cantidad de $\Delta$:

$$ \Delta = \min_{i \in Z_1, j \notin Z_2} \{ a[i][j] - u[i] - v[j] \}. $$

Este valor estrictamente positivo.

(La prueba. Supongamos que $\Delta = 0$. Entonces existe una estricta arista $(i,j)$ y $i \in Z_1$ y $j \notin Z_2$. De ello se desprende que el borde del $(i,j)$ debe ser orientado de la segunda cuota a la primera, es decir, rígido arista $(i,j)$ debe entrar en паросочетание $M$. Sin embargo, esto es imposible, porque no podíamos caer en la sequedad cima de la $i$, excepto a través del eje de $j$ en $i$. Han llegado a una contradicción, entonces $\Delta > 0$.)

Ahora \bf{volver a calcular el potencial}, por lo tanto: para todos los vértices de $i \in Z_1$ haremos $u[i] += \Delta$, y para todos los vértices de $j \in Z_2$ --- haremos $v[j] -= \Delta$. El potencial sigue siendo correcto potencial.

(La prueba. Para ello, es necesario mostrar que se sigue para todas las $i$ y $j$ se: $u[i] + v[j] \le de a[i][j]$. Para los casos cuando $i \in Z_1 \& j \in Z_2$ o $i \notin Z_1 \& j \notin Z_2$ --- es así, ya que para ellos la suma de $u[i]$ y $v[j]$ no ha cambiado. Cuando $i \notin Z_1 \& j \in Z_2$ --- la desigualdad sólo se ha intensificado. Por último, para el caso de una $i \in Z_1 \& j \notin Z_2$ --- aunque la parte izquierda de la desigualdad y aumenta la desigualdad sigue existiendo, dado que el valor de $\Delta$, como se desprende de su definición --- este es el máximo aumento que no conduzca a la violación de la desigualdad.)

Además, el viejo паросочетание $M$ duros de las costillas puede dejar, es decir, todas las aristas de паросочетания quedarán duros.

(La prueba. Para un rígido arista $(i,j)$ ha dejado de ser difícil en el resultado de la modificación de la capacidad, es necesario, para la igualdad de $u[i] + v[j] = a[i][j]$ se ha convertido en la desigualdad de $u[i] + v[j] < a[i][j]$. Sin embargo, la parte izquierda podría disminuir en un solo caso: cuando $i \notin Z_1 \& j \in Z_2$. Pero en vez de $i \notin Z_1$, entonces esto significa que el borde del $(i,j)$ no podía ser de aleta de паросочетания que se quería demostrar.)

Por último, para mostrar que el cambio de la capacidad \bf{no pueden ocurrir infinitamente}, tenga en cuenta que cada vez que cambio de la capacidad de número de vértices, factibles de omisión, es decir, $|Z_1|+|Z_2|$, estrictamente aumenta. (No se puede decir que aumenta el número de duros de las aletas.)

(La prueba. En primer lugar, cualquier vértice que era viable, factible y se quedará. En realidad, si un vértice es posible, antes de ella hay algo de la ruta de acceso de los vértices, que comienza en poco la cima de la primera parte; y porque para los bordes de la vista $(i,j), i \in Z_1 \& j \in Z_2$ monto $u[i] + v[j]$ no cambia, todo este camino se mantenga y después de la modificación de la capacidad, que se quería demostrar. En segundo lugar, mostraremos que como resultado de la conversión de la capacidad aparecido al menos una nueva alcanzada la cima. Pero es casi obvio, si volvemos a la definición de $\Delta$: lo de la costilla $(i,j)$, en el que se ha alcanzado el mínimo, ahora será duro y, por lo tanto, la cima de $j$ será factible gracias a la arista y la cima de la $i$.)

Por lo tanto, todo puede pasar, no más de $n$ de los nuevos cálculos de la capacidad antes de que se descubriera incrementa la cadena y la potencia de la паросочетания $M$ aumentará.

}

Por lo tanto, tarde o temprano, se ha encontrado un potencial, que corresponde a la perfecta паросочетание $M$, que es la respuesta a la tarea.

Si hablamos de \bf{asíntota de la función} algoritmo, es $O(n^4)$, ya que todo debe ocurrir $n$ de los aumentos паросочетания, antes de cada uno de los cuales se produce no más de $n$ de los nuevos cálculos de capacidad, cada uno de los cuales se ejecuta por hora $O(n^2)$.

La aplicación por $O(n^4)$ estamos aquí para dar como resultado, no vamos a, ya que de todos modos resulta no más corto que se describe a continuación la implementación de la $O(n^3)$.


\h3{ la Construcción de un algoritmo por el $O(n^3)$ ($O(n^2 m)$) }

Aprendamos ahora aplicar el mismo algoritmo de асимптотику $O (n^3)$ (rectangulares de tareas $n \times m$ --- $O(n^2 m)$).

La idea clave: ahora vamos a \bf{agregar en el examen de las filas de la matriz, una a una,}, en lugar de considerar todos a la vez. Por lo tanto, se describe el algoritmo anterior toma la forma de:

\ul{

\li Añadimos a la consideración de la siguiente fila de la matriz $a$.

\li Hasta que no amplificada de la cadena que comienza en esta línea, calcular la tasa potencial.

\li vez que se incrementa la cadena, чередуем паросочетание a lo largo de ella (incluyendo por tanto la última fila en паросочетание), y pasamos al principio (a la revisión de la siguiente línea).

}

Para obtener la асимптотики, hay que realizar los pasos 2 y 3, se ejecutan para cada fila de la matriz, por hora $O(n^2)$ (rectangulares de tareas - - - $O (n m)$).

Para ello, nos recuerda dos hechos probados de la que ya hemos mencionado:

\ul{

\li Cuando se cambia la capacidad de la cima, que fueron posibles de rastrear kuna, alcanzables y permanecerán.

\li Sólo podría ocurrir sólo $O(n)$ de los nuevos cálculos de la capacidad antes de que se alcanzará la mejora de la cadena.

}

De aquí derivan \bf{ideas clave} para obtener la асимптотики:

\ul{

\li Para verificar la presencia de lo que la cadena no es necesario iniciar el rastreo kuna de nuevo después de cada conversión de la capacidad. En lugar de ello, se puede hacer el rastreo de kuhn en \bf{итеративном}: después de cada conversión de la capacidad buscamos добавившиеся duros de la aleta, y si la izquierda fines eran alcanzables, помечаем su derecha los extremos también como alcanzables y seguimos con el rastreo de ellos.

\li Desarrollando esta idea más lejos, se puede llegar a este punto de vista del algoritmo: es un ciclo, en cada paso que primero se vuelve a calcular la capacidad de, a continuación se encuentra la columna, que se convirtió en accesible (y tal siempre hay, ya que después de la calibración de la capacidad siempre aparecen nuevos vértices alcanzables), y si esta columna se ненасыщен, se encontró que eleva la cadena, y si la columna se ha saciado --- la le en паросочетании la cadena también se convierte en una meta alcanzable.

Ahora el algoritmo toma la forma: el ciclo de agregar columnas, en cada uno de los que primero se vuelve a calcular el potencial y, a continuación, una nueva columna marcada como alcanzable.

\li rápidamente Para volver a calcular la capacidad (más rápido que el ingenuo opción por $O(n^2)$), es necesario mantener auxiliares mínimos de cada una de las columnas de $j$:

$$ minv[j] = \min_{i \in Z_1} \{ a[i][j] - u[i] - v[j] \}. $$

Como es fácil de ver, buscando el valor de $\Delta$ se expresa a través de ellos de la siguiente manera:

$$ \Delta = \min_{j \notin Z_2} \{ minv[j] \}. $$

Por lo tanto, la búsqueda de $\Delta$ ahora se puede producir por $O(n)$.

Mantener esta matriz $minv[]$ debe a la aparición de nuevos visitados de filas. Esto, obviamente, se puede hacer por $O(n)$ una se añade a la cadena (lo que equivale a $O(n^2)$). Actualizar la matriz $minv[]$ es necesario cuando se vuelve a calcular la capacidad, que también se hace por el tiempo $O(n)$ en un recuento de la capacidad (ya que $minv []$, se cambia sólo para недостигнутых hasta que de columnas, es decir, se reduce en $\Delta$).

}

Por lo tanto, el algoritmo toma el siguiente aspecto: en un ciclo añadimos en el examen de las filas de la matriz, una detrás de la otra. Cada fila se procesa por hora $O(n^2)$, ya que podría producirse solamente $O(n)$ de los nuevos cálculos de capacidad (cada uno --- por hora $O(n)$), para que por el tiempo $O(n^2)$ se admite la matriz $minv[]$; el algoritmo de kuhn total de trabajo ha terminado por hora $O(n^2)$ (ya que se presenta en forma de un $O(n)$ iteraciones, en cada una de las cuales se visita la nueva columna).

Total asíntotas es $O(n^3)$ --- o, si la tarea прямоугольна, $O(n^2 m)$.


\h3{ Implementación del algoritmo húngaro $O(n^3)$ ($O(n^2 m)$) }

La aplicación en realidad se ha diseñado \bf{andrés Лопатиным} hace unos años. La diferencia increíble la brevedad: el algoritmo se coloca en \bf{30 líneas de código}.

Esta aplicación busca de una solución para la rectangular de entrada de la matriz $a[1 \ldots, n][1 \ldots m]$, donde $n \le m$. La matriz se almacena en $1$-indexación para facilitar y abreviar el código. El caso es que en esta implementación se introducen falsas cero, la cadena y el cero de la columna, lo que permite escribir muchos ciclos en forma general, sin comprobaciones adicionales.

Las matrices de $u[0 \ldots, n]$ y $v[0 \ldots m]$ almacenan potencial. Originalmente fue de cero, lo que es cierto para la matriz formada por cero de la fila. (Tenga en cuenta que esta aplicación no es importante, hay o no hay en la matriz $a[][]$ números negativos.)

La matriz $p[0 \ldots m] $ contiene паросочетание: para cada columna de la $i = 1, \ldots m$ almacena el número de la fila seleccionada $p[i]$ (o $0$, si aún no está seleccionada). Cuando este $p[0]$ para la comodidad de la aplicación se supone que el número actual de la cuestión de la línea.

La matriz $minv[1 \ldots m]$ contiene para cada columna $j$ auxiliares mínimos necesarios para la rápida conversión de la capacidad de:

$$ minv[j] = \min_{i \in Z_1} \{ a[i][j] - u[i] - v[j] \}. $$

La matriz $way[1 \ldots m]$ contiene información sobre dónde estos mínimos se obtienen, para que posteriormente fueron capaces de recuperar mejoran la cadena. A primera vista, parece que en el array $way[]$ para cada columna se debe almacenar el número de la línea, así como tener otra matriz: para cada fila de recordar el número de la columna, de la que nosotros llegamos. Sin embargo, en lugar de ello, se puede observar que el algoritmo de kuhn siempre se mete en la fila, pasando por el eje паросочетания de las columnas, por lo tanto, los números de las filas para la recuperación de la cadena siempre se puede sacar de la паросочетания (es decir, a partir de la matriz $p[]$). Por lo tanto, $way[j]$ para cada columna $j$ contiene el número anterior de la columna (o $0$, si tal no existe).

El algoritmo es externo \bf{ciclo de las filas de la matriz}, dentro de la cual se agrega en el examen de la $i$-ésima fila de la matriz. La parte interior es un bucle do-while (p[j0] != 0)", que funciona hasta que no se encuentra libre de la columna $j0$. Cada iteración del ciclo de marca посещенным una nueva columna con el número de $j0$ (посчитанным en la última iteración; y en un principio cero --- es decir, стартуем nos ficticio de la columna), así como una nueva línea de $i0$ --- adyacente a él en паросочетании (es decir, $p[j0]$; y en un principio si $j0=0$ toma $i$aja-cadena). Debido a la aparición de una nueva visitado de $i0$ es necesario en consecuencia, volver a calcular la matriz $minv[]$, al mismo tiempo, nos encontramos con un mínimo en él --- cantidad de $delta$, y en qué columna $j1$, este mínimo se ha logrado (tenga en cuenta que cuando tal aplicación $delta$ podría ser igual a cero, lo que significa que en el paso de la capacidad no se puede cambiar: el nuevo alto de la columna y sin). Después de esto, se realiza el recuento de capacidad $u[], v[]$, el cambio de la matriz $minv[]$. Cuando finaliza el ciclo de "do-while" hemos encontrado que más de la cadena, en la columna de $j0$, "desenrollar" que se puede, usando la matriz de los antepasados de $way[]$.

La constante de $INF$ --- es "infinito", es decir, un número, a sabiendas de que es el mayor de todos los posibles números en la entrada de la matriz $a[][]$.

\code
vector<int> u (n+1), v (m+1), p (m+1), way (m+1);
for (int i=1; i<=n; ++i) {
p[0] = i;
int j0 = 0;
vector<int> minv (m+1, INF);
vector<char> used (m+1, false);
do {
used[j0] = true;
int i0 = p[j0], delta = INF, j1;
for (int j=1; j<=m; j++)
if (!used[j]) {
int cur = a[i0][j]-u[i0]-v[j];
if (cur < minv[j])
minv[j] = cur way[j] = j0;
if (minv[j] < delta)
delta = minv[j], j1 = j;
}
for (int j=0; j<=m; j++)
if (used[j])
u[p[j]] += delta, v[j] -= delta;
else
minv[j] -= delta;
j0 j1=;
} while (p[j0] != 0);
do {
int j1 = way[j0];
p[j0] = p[j1];
j0 j1=;
} while (j0);
}
\endcode

La recuperación de la respuesta más habitual la forma, es decir, encontrar para cada fila de la $i = 1, \ldots, n$ de la habitación seleccionada en la columna de $ans[i]$, se hace de la siguiente manera:

\code
vector<int> ans (n+1);
for (int j=1; j<=m; j++)
ans[p[j]] = j;
\endcode

Precio encontrado паросочетания se puede tomar como un potencial cero de la columna (tomado con signo contrario). En realidad, como es fácil de seguir por el código, $-v[0]$ contiene en sí mismo la suma de todos los valores de $delta$, es decir, el cambio total de la capacidad. Aunque cada vez que cambia la capacidad de cambiar puedan varios valores de $u[i]$ y $v[j]$, el cambio total de la magnitud de la capacidad es exactamente igual al valor de $delta$, ya que hasta ahora no amplificada de la cadena, alcanzables en el número de filas es igual a la unidad mayor que el número de acceso de las columnas (sólo la fila actual $i$ no tiene "parejas" en forma visitado de la columna):

\code
int cost = -v[0];
\endcode


\h2{ Ejemplos de tareas }

Aquí algunos ejemplos en la solución de los problemas de las designaciones: desde muy triviales, y a los menos obvios objetivos:

\ul{

\li Dan двудольный conde, es necesario encontrar en él паросочетание \bf{máximo паросочетание de peso mínimo} (es decir, en primer lugar, se maximiza el tamaño de la паросочетания, en la segunda --- minimiza su costo).

Para resolver simplemente construyendo una tarea acerca de las asignaciones, poniendo en el lugar que faltan de aristas número de "infinito". Después de eso, decidimos que la tarea de la oficina húngara de un algoritmo, y retiramos de la respuesta de la aleta infinito de peso (que puedan entrar en la respuesta, si la tarea no tiene una solución en forma de perfecto паросочетания).

\li Dan двудольный conde, es necesario encontrar en él паросочетание \bf{máximo паросочетание el peso máximo}.

La solución una vez más es evidente, solo todo el peso que hay que multiplicar al menos una unidad (o en húngaro algoritmo de reemplazar todos los mínimos a máximos, y el infinito --- a menos infinito).

\li Tarea \bf{detección de objetos en movimiento de imágenes}: se efectuaron dos disparos, tras las cuales se recibieron dos conjuntos de coordenadas. Es necesario relacionar los objetos en la primera y en la segunda imagen, es decir, determinar para cada punto de la segunda toma, ¿en que punto de la primera imagen que coincida con. Se requiere minimizar la suma de las distancias entre los asociados de puntos (es decir, buscamos una solución en la que los objetos en total han pasado más pequeño de la ruta).

Para resolver simplemente estamos construyendo y resolviendo la tarea de las asignaciones, en tanto que la balanza de las costillas sobresalen евклидовы la distancia entre puntos.

\li Tarea \bf{detección de objetos en movimiento de локаторам}: hay dos localizador, que son capaces de identificar la posición de un objeto en el espacio, sino la dirección. Con ambos localizadores (ubicados en diferentes puntos a) se ha obtenido información en forma de $n$ de tales direcciones. Se desea determinar la posición de los objetos, es decir, determinar los supuestos de la posición de objetos y sus correspondientes pares de direcciones a fin de minimizar la suma de las distancias de los objetos a los rayos-direcciones.

Solución --- de nuevo, simplemente construyendo y resolviendo la tarea sobre la distribución, donde los vértices de la primera cuota es $n$ destinos desde el primer localizador, los vértices de la segunda cuota --- $n$ esferas de la segunda localizador, y los pesos de las costillas --- la distancia entre los rayos.

\li Capa \bf{basada ациклического conde de caminos}: dan orientado ациклический conde, es necesario encontrar el menor número de rutas de acceso (en caso de empate, - - - con la menor cantidad de peso), para cada vértice del conde estaba haría exactamente en el mismo camino.

Solución --- construir el grafo correspondiente a la двудольный conde, y de encontrar en ella la máxima паросочетание de peso mínimo. Para más información, ver \algohref=path_cover{un artículo}.

\li \bf{para Colorear de árbol}. Dado un árbol, en el que cada vértice, además de las hojas, tiene exactamente $k-1$ de los hijos. Es necesario seleccionar de cada vértice un cierto color de $k$ de colores para que ninguna de las dos vértices adyacentes no tengan el mismo color. Además, para cada vértice y cada color se conoce el costo de la pintura de la cima en este color, y desea minimizar el costo total.

Para la decisión, utilizaremos el método de la programación dinámica. Es decir, aprender a pensar la cantidad de $d[v][c]$, donde $v$ --- número de vértices, $c$ --- número de colores, y el valor de $d[v][c]$ --- este es el precio mínimo de colorear los vértices $v$, junto con sus descendientes, y la propia cima de $v$ es de color $c$. Para calcular esta cantidad de $d[v][c]$, es necesario distribuir el resto de la $k-1$ de colores de los hijos de la cima de $v$, y para ello tenemos que construir y resolver el problema de las asignaciones (en la que la cima de un porcentaje --- los colores de los vértices de la otra fracción de --- de la cima de los hijos, y el peso de las aletas --- este es el significado de los altavoz $d[][]$).

Por lo tanto, cada valor de $d[v][c]$ se considera mediante la solución de una tarea acerca de las asignaciones, lo que finalmente da асимптотику $O(n k^4)$.

\li Si en la tarea de asignación de peso no se especifican los bordes, y de los picos, de los cuales sólo \bf{cerca de las cimas de una proporción}, entonces se puede prescindir del algoritmo húngaro, y sólo ordenar la cima de peso y ejecutar regular \algohref=kuhn_matching{el algoritmo de kuhn} (para más detalles, consulte \algohref=vertex_weighted_matching{un artículo}).

\li Considere el siguiente \bf{un caso}. Pida a cada vértice de la primera cuota se atribuye un número de $\alpha[i]$, y cada vértice de la segunda cuota --- $\beta[j]$. Dejar que el peso de cualquier aleta $(i,j)$ es $\alpha[i] \cdot \beta[j]$ de $\alpha[i]$ y $\beta[j]$ conocemos). Resolver el problema de las asignaciones.

Para resolver sin húngaro algoritmo consideremos primero el caso, cuando en ambas fracciones por dos vértices. En este caso, como es fácil ver, en beneficio de unir los vértices en orden inverso: el vértice con menos de $\alpha[i],$ conectar con la cima con más de $\beta[j]$. Es una regla fácil de generalizar a cualquier número de vértices: es necesario ordenar la cima de la primera cuota en orden creciente de $\alpha[i]$, la segunda cuota --- en orden decreciente de $\beta[j]$, y unir los vértices de dos en dos, en ese orden. Por lo tanto, obtenemos la solución асимптотикой $O (n \log n)$.

\li \bf{Tarea de capacidades}. Dada la matriz $a[1 \ldots, n][1 \ldots m]$. Es necesario encontrar dos matrices $u[1 \ldots, n]$ y $v[1 \ldots m]$ tales que para todo $i$ y $j$ se $u[i] + v[j] \le de a[i][j]$, pero la suma de los elementos de las matrices $u[]$ y $v[]$ es mayor.

Sabiendo húngaro algoritmo, la solución de esta tarea no es nada de trabajo: el algoritmo húngaro apenas encuentra exactamente un potencial de $u[], v[]$, que cumpla la condición de la tarea. Por otro lado, sin el conocimiento de la húngara algoritmo de resolver esta tarea es casi imposible.

}



\h2{ Literatura }

\ul{

\li \book{Ravindra Ahuja, Thomas Magnanti, James Orlin}{Network Flows}{1993}{ahuja_flows.djvu}

\li \book{Harold Kuhn}{The Hungarian Method for the Assignment Problem}{1955}

\li \book{James Munkres}{Algorithms for Assignment and Transportation Problems}{1957}

}



\h2{ Tareas en línea judges }

La lista de tareas en la solución de los problemas de las designaciones:

\ul{

\li \href=http://uva.onlinejudge.org/index.php?option=com_onlinejudge&Itemid=8&page=show_problem&problem=1687{UVA #10746 \bf{"Crime Wave – The Sequel"} ~~~~ [dificultad: baja]}

\li \href=http://uva.onlinejudge.org/index.php?option=com_onlinejudge&Itemid=8&page=show_problem&problem=1829{UVA #10888 \bf{"Warehouse"} ~~~~ [dificultad media]}

\li \href=http://acm.sgu.es/problem.php?contest=0&problem=210{SGU #210 \bf{"Beloved Sons"} ~~~~ [dificultad media]}

\li \href=http://livearchive.onlinejudge.org/index.php?option=com_onlinejudge&Itemid=8&page=show_problem&problem=1277{UVA #3276 \bf{"The Great Wall Game"} ~~~~ [nivel de dificultad: alto]}

\li \href=http://uva.onlinejudge.org/index.php?option=com_onlinejudge&Itemid=8&page=show_problem&problem=1237{UVA #10296 \bf{"Jogging Trails"} ~~~~ [nivel de dificultad: alto]}

}
