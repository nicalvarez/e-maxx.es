<h1>Modificación del método de Empujar a la предпотока para encontrar el flujo máximo por O (N<sup>3</sup>)</h1>

<p>se Supone que usted ya ha leído <algohref=preflow_push>Método de Empujar a la предпотока encontrar el flujo máximo por O (N<sup>4</sup>)</algohref>.</p>
<h2>Descripción</h2>
<p>una Modificación muy simple: en cada iteración entre todos los atestados de los vértices seleccionamos sólo los vértices que tienen <b>набольшую altura</b>, y seguimos empujando a través de un/elevación sólo a esas cimas. Además, para la selección de los vértices con mayor altura en nosotros no necesita ninguna de las estructuras de datos, basta con almacenar una lista de vértices con mayor altura y simplemente recalcular, si todos los vértices de esta lista han sido procesados (entonces en la lista se añadirán a la cima con menor altura), y cuando aparezca una nueva lleno de picos de mayor altura que en la lista, borrar la lista y añadir la cima en la lista.</p>
<p>a Pesar de su simplicidad, esta modificación le permite reducir асимптотику un orden de magnitud. Para ser exactos, asíntotas obtuvo el algoritmo es igual a la <b>O (N M + N<sup>2</sup> sqrt (M))</b>, que en el peor de los casos es de <b>O (N<sup>3</sup>)</b>.</p>
<p>Esta modificación fue propuesta Черияном (Cheriyan) y Махешвари (Maheshvari) en 1989</p>
<h2>Realización</h2>
<p>Aquí se muestra acabada de la aplicación de este algoritmo.</p>
<p>a Diferencia del algoritmo de empujar - sólo en presencia de la matriz maxh, en la que se almacenarán las habitaciones abarrotadas de vértices con una altura máxima de. El tamaño de la matriz se especifica en la variable de sz. Si en cualquier iteración resulta que esta matriz vacía (sz==0), entonces nos llenamos de su (simplemente pasando por todas las cimas); si después de este conjunto sigue en blanco, entonces el hacinamiento de los vértices no, y el algoritmo se detiene. De lo contrario, pasamos por las cimas en esta lista, aplicando o empujando a través de la elevación. Después de realizar la operación de empujar a la actual en la cima puede dejar de ser lleno de gente, en este caso, eliminamos de la lista de maxh. Si después de una operación de elevar la altura de la cima se hace mayor que la altura a la cima en la lista de maxh, lo eliminamos de la lista (sz=0), y en seguida pasamos a la siguiente iteración del algoritmo de perforaciones (en la que se construirá la nueva lista de maxh).</p>
<code>const int INF = 1000*1000*1000;

int main() {

int n;
vector < vector<int> > c (n, vector<int> (n));
int s, t;
... la lectura de n, c, s, t ...

vector<int> e (n);
vector<int> h (n);
h[s] = n-1;
vector < vector<int> > f (n, vector<int> (n));

for (int i=0; i<n; ++i) {
f[s][i] = c[s][i];
f[i][s] = -f[s][i];
e[i] = c[s][i];
}

vector<int> maxh (n);
int sz = 0;
for (;;) {
if (!sz)
for (int i=0; i<n; ++i)
if (i != s && i != t && e[i] > 0) {
if (sz && h[i] > h[maxh[0]])
sz = 0;
if (!sz || h[i] == h[maxh[0]])
maxh[sz++] = i;
}
if (!sz); break;
while (sz) {
int i = maxh[sz-1];
bool pushed = false;
for (int j=0; j<n && e[i]; j++)
if (c[i][j]-f[i][j] > 0 && h[i] == h[j]+1) {
pushed = true;
int addf = min (c[i][j]-f[i][j], e[i]);
f[i][j] += addf, f[j][i] -= addf;
e[i] -= addf, e[j] += addf;
if (e[i] == 0) --sz;
}
if (!pushed) {
h[i] = INF;
for (int j=0; j<n; ++j)
if (c[i][j]-f[i][j] > 0 && h[j]+1 < h[i])
h[i] = h[j]+1;
if (h[i] > h[maxh[0]]) {
sz = 0;
break;
}
}
}
}

... la salida de flujo de f ...

}</code>