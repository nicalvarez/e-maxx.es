\h1{Tarea de josé}

La condición de la tarea. Dado naturales $n$ y $k$. Por el círculo de la prescripción de todos los números enteros de 1 a $n$. Primero cuenta $k$de número, a partir de la primera, y lo eliminan. Luego de él cuenta $k$ de números y $k$-tes quitan, y así sucesivamente, el Proceso se detiene cuando queda un número. Es necesario encontrar este número.

Objetivo se basa \bf{josé Флавием} (Flavius josé) aún en el siglo 1 (la verdad es que en poco más restringida: si $k = 2$).

Abordar esta tarea de modelado. Una sencilla simulación funcionará $O (n^2)$. Usando \algohref=segment_tree{Árbol de regiones}, se puede producir una simulación por $O (n \log n)$.

\h2{Solución a $O(n)$}

Trataremos de encontrar regularidad, expresando la respuesta para la tarea $J_{n,k}$ a través de la solución de tareas anteriores.

A través de la simulación construiremos una tabla de valores, por ejemplo, este:

$$ \bordermatrix {
n \setminus k&1&2&3&4&5&6&7&8&9&10 \cr
1&1&1&1&1&1&1&1&1&1&1& \cr
2&2&1&2&1&2&1&2&1&2&1& \cr
3&3&3&2&2&1&1&3&3&2&2& \cr
4&4&1&1&2&2&3&2&3&3&4& \cr
5&5&3&4&1&2&4&4&1&2&4& \cr
6&6&5&1&5&1&4&5&3&5&2& \cr
7&7&7&4&2&6&3&5&4&7&5& \cr
8&8&1&7&6&3&1&4&4&8&7& \cr
9&9&3&1&1&8&7&2&3&8&8& \cr
10&10&5&4&5&3&3&9&1&7&8& \cr
} $$

Y es claramente visible la siguiente \bf{patrón}:
$$ J_{n,k} = \left( J_{(n-1),k} + k - 1 \right)\ \%\ n + 1 $$
$$ J_{1,k} = 1 $$

Aquí 1-indexación varios estropea la elegancia de la fórmula, si numerar la posición de cero, resulta muy evidente la fórmula:

$$ J_{n,k} = \left( J_{(n-1),k} + k \right)\ \%\ n = \sum_{i=1}^n k\ \%\ i $$

Así, hemos encontrado la solución para la tarea de josé, que trabaja por el $O (n)$ de operaciones.

Simple \bf{recursiva de la implementación} (1-índice):
\code
int joseph (int n, int k) {
return n>1 ? (joseph (n-1, k) + k - 1) % n + 1 : 1;
}
\endcode

\bf{Нерекурсивная forma}:
\code
int joseph (int n, int k) {
int res = 0;
for (int i=1; i<=n; ++i)
res = res + k) % i;
return res + 1;
}
\endcode

\h2{Solución a $O(k \log n)$}

Para cantidades relativamente pequeñas de $k$ se puede llegar a una mejor solución de la que hablé antes a una solución por la $O(n)$. Si $k$ es pequeña, incluso intuitivamente claro que el algoritmo hace una gran cantidad de exceso de acción: cambios graves se producen sólo cuando se produce la toma de módulo $n$, y hasta este momento el algoritmo simplemente varias veces añade a la respuesta número $k$. En consecuencia, puede deshacerse de estos pasos innecesarios, 

Una pequeña causada en este caso, la complejidad radica en el hecho de que después de la eliminación de estos números, nos resultar una tarea con menos de $n$, pero el punto de partida no es en el primer número, y en algún otro lugar. Por lo tanto, llamando recursivamente a sí mismo de la tarea con el nuevo $n$, a continuación, se deben con cuidado traducir el resultado en nuestro sistema de numeración de su propia.

También de manera independiente, es necesario desmontar el caso, cuando $n$ es inferior a $k$ --- en este caso específico la optimización de degenerará en un bucle infinito.

\bf{Aplicación} (para instalaciones en 0-índice):
\code
int joseph (int n, int k) {
if (n == 1) return 0;
if (k == 1) return n-1;
if (k > n) return (joseph (n-1, k) + k) % n;
int cnt = n / k;
int res = joseph (n - cnt, k);
res -= n % k;
if (res < 0) res += n;
else res += res / (k - 1);
return res;
}
\endcode

Ponemos \bf{асимптотику} este algoritmo. Inmediatamente tenga en cuenta que el caso de la $n < k$ versado nosotros antigua solución que funcione en este caso, $O(k)$. Ahora veamos el algoritmo. De hecho, en cada iteración en lugar de $n$ de números obtenemos aproximadamente $n \left( 1 - \frac{1}{k} \right)$ de números, por lo que el número total de $x$ iteraciones del algoritmo de alrededor se puede encontrar de la ecuación:
$$ n \left( 1 - \frac{1}{k} \right) ^ x = 1, $$
логарифмируя, obtenemos:
$$ \ln n + x \ln \left( 1 - \frac{1}{k} \right) = 0, $$
$$ x = - \frac{ \ln n }{ \ln \left( 1 - \frac{1}{k} \right) }, $$
usando la descomposición de los logaritmos en una serie de taylor, se obtiene una estimación aproximada de:
$$ x \approx k \ln n $$

Por lo tanto, asíntotas algoritmo realmente $O(k \log n)$.

\h2{Analítico en la solución de $k=2$}

En este caso concreto (en el cual se fijó la tarea de josé Флавием) tarea se resuelve mucho más fácil.

En el caso de la par $n$ recibimos, que se cruza de todos los números pares, y luego quedará la tarea para $\frac{n}{2}$, entonces la respuesta para $n$ resultarn de respuesta para $\frac{n}{2}$ en la multiplicación de dos y resta de unidades (por cuenta del desplazamiento de las posiciones):

$$ J_{2n,2} = 2 J_{n,2} - 1 $$

De manera similar, en el caso impar $n$ se cruza de todos los números pares, entonces el primer número, y seguirá siendo un desafío para $\frac{n-1}{2}$, y teniendo en cuenta el desplazamiento de las posiciones recibimos la segunda fórmula:

$$ J_{2n+1,2} = 2 J_{n,2} + 1 $$

Cuando la aplicación se puede usar directamente este рекуррентную la dependencia. Es posible que este patrón de traducir en otra forma: $J_{n,2}$ son la secuencia de los números impares, "перезапускающуюся" con una unidad cada vez, cuando $n$ resulta grado de doses. Se puede escribir en forma de una fórmula:

$$ J_{n,2} = 1 + 2 \left( n - 2^{\lfloor \log_2 n \rfloor} \right) $$

\h2{Analítico en la solución de $k>2$}

A pesar de un tipo más simple de las tareas y un gran número de artículos sobre este y otros objetivos, a simple vista analítica de la solución de la tarea de josé hasta el momento no se han encontrado. Para los pequeños de $k$ sacado algunas fórmulas, pero, al parecer, todos ellos трудноприменимы en la práctica (por ejemplo, véase Halbeisen, Hungerbuhler "The josé Problem" y Odlyzko, Wilf "Functional iteration and the josé problem").