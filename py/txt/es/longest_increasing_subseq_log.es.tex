\h1{ Encontrar наидлиннейшей creciente подпоследовательности }

\bf{Condición de la tarea} la siguiente. Dada la matriz de $n$ de números: $a[0 \ldots, n-1]$. Es necesario encontrar en esta sucesión estrictamente creciente de подпоследовательность de mayor longitud.

\bf{Formalmente} esto se ve de la siguiente manera: es necesario encontrar una secuencia de índices de $i_1 \ldots i_k$ que:

$$ i_1 < i_2 < \ldots < i_k, $$
$$ a[i_1] < a[i_2] < \ldots < a[i_k]. $$

En este artículo se describen los algoritmos de resolución de la tarea, así como algunas de las tareas que se puede reducir a esta tarea.



\h2{ Solución a $O(n^2)$: el método de la programación dinámica }

Dinámica de la programación --- es muy metodología general que permite resolver la enorme clase de tareas. Aquí vamos a ver esta metodología con respecto a nuestra específicas de la tarea.

Aprenderemos a buscar primero \bf{longitud} наидлиннейшей creciente подпоследовательности, y la recuperación de la подпоследовательности nos ocuparemos un poco más adelante.


\h3{ Dinámica de la programación para la búsqueda de la longitud de la respuesta }

Para ello, aprenderemos a leer el array $d[0 \ldots, n-1]$, donde $d[i]$ --- es la longitud de la наидлиннейшей creciente подпоследовательности, оканчивающейся precisamente en el elemento con el índice de la $i$. La matriz de este (es --- la propia dinámica) vamos a contar poco a poco: primero $d[0]$, entonces $d[1],$ y así sucesivamente, Al final, cuando esta matriz se contabilizarán nosotros, la respuesta a la tarea será igual al máximo de la matriz $d[]$.

Por tanto, que el actual índice de - - - de $i$, es decir, queremos calcular el valor de $d[i]$, y todos los valores anteriores $d[0] 
\ldots d[i-1]$ ya contados. Entonces tenga en cuenta que tenemos dos opciones:

\ul{

\li o $d[i] = 1$, es decir, buscando подпоследовательность compuesto por un número único de $a[i]$.

\li o $d[i] > 1$. Entonces, antes del número $a[i]$ en la подпоследовательности vale la pena algún otro número. Vamos a переберем este número: puede ser cualquier elemento de $a[j]$ $(j = 0 \ldots i-1)$, pero tal que $a[j] < a[i]$. Supongamos que estamos considerando un índice actual de $j$. Porque la dinámica de $d[j]$ para él ya calculado, se obtiene que este número es $a[j]$ junto con el número de $a[i]$ da respuesta $d[j] + 1$. Por lo tanto, $d[i]$ se puede contar de esta fórmula:

$$ d[i] = \max_{j=0 \ldots i-1, \atop a[j] < a[i]} ( d[j] + 1 ). $$

}

Al combinar estas dos opciones, en uno, obtenemos la final de un algoritmo para el cálculo de $d[i]$:

$$ d[i] = \max \Big( 1, \max_{j=0 \ldots i-1, \atop a[j] < a[i]} ( d[j] + 1 ) \Big). $$

Este algoritmo --- y sí, es el altavoz.


\h3{ Aplicación }

Presentamos la aplicación de lo descrito anteriormente, el algoritmo que detecta y muestra la longitud de la наидлиннейшей creciente подпоследовательности:

\code
int d[MAXN]; // constante de MAXN es igual a la de mayor valor posible de n

for (int i=0; i<n; ++i) {
d[i] = 1;
for (int j=0; j<i; j++)
if (a[j] < a[i])
d[i] = max (d[i], 1 + d[j]);
}

int ans = d[0];
for (int i=0; i<n; ++i)
ans = max (ans, d[i]);
cout << ans << endl;
\endcode


\h3{ la Recuperación de la respuesta }

Hasta ahora sólo hemos aprendido a buscar la longitud de la respuesta, pero la наидлиннейшую подпоследовательность podemos sacar no podemos, porque no guardamos ningún tipo de información adicional acerca de dónde se obtienen los máximos históricos.

Para poder recuperar la respuesta, además de la dinámica de $d[0 \ldots, n-1]$ es necesario también para mantener la auxiliar de $matriz p[0 \ldots, n-1]$ --- entonces, ¿en qué lugar достигся máximo para cada valor de $d[i]$. En otras palabras, el índice $p[i]$ significará el mismo índice $j$, con el que ha resultado el mayor valor de $d[i]$. (Esta matriz $p[]$ en la programación dinámica, a menudo se denomina "matriz de los antepasados".)

Entonces, para ver la respuesta, solo hay que ir de un elemento con un valor máximo de $d[i]$ de sus antepasados hasta que no nos sacaremos toda la подпоследовательность, es decir, hasta que no lleguemos un elemento con un valor de $d = 1$.


\h3{ Aplicación de la recuperación de la respuesta }

Así, nos cambia y el código de la dinámica, y se agrega el código que produce la salida de наидлиннейшей подпоследовательности (se muestran los índices de los elementos подпоследовательности, 0-indexación).

Para su comodidad, hemos inicialmente pusieron los índices de $p[i] = -1$: elementos que la dinámica ha resultado igual a la unidad, es que el valor de un antepasado y se quedará menos por la unidad de la que poco a poco es más conveniente cuando se restaura la respuesta.

\code
int d[MAXN], p[MAXN]; // constante de MAXN es igual a la de mayor valor posible de n

for (int i=0; i<n; ++i) {
d[i] = 1;
p[i] = -1;
for (int j=0; j<i; j++)
if (a[j] < a[i])
if (1 + d[j] > d[i]) {
d[i] = 1 + d[j];
p[i] = j;
}
}

int ans = d[0], pos = 0;
for (int i=0; i<n; ++i)
if (d[i] > ans) {
ans = d[i];
pos = i;
}
cout << ans << endl;

vector<int> path.
while (pos != -1) {
path.push_back (pos);
pos = p[pos];
}
reverse path.begin(), path.end());
for (int i=0; i<(int)path.size(); ++i)
cout << path[i] << ' ';
\endcode


\h3{ una forma Alternativa de recuperación de la respuesta }

Sin embargo, como casi siempre en el caso de la programación dinámica, para la recuperación de la respuesta puede no mantener adicional de la matriz de los antepasados de $p[]$, y simplemente volver a пересчитывая el elemento actual de la dinámica y que busca, en que el índice se ha alcanzado el máximo.

De esta manera cuando la aplicación conduce a una un poco más larga que el código, pero a cambio obtenemos un ahorro de memoria y absoluta coincidencia de la lógica del programa en el proceso de conteo de los altavoces y en el proceso de recuperación.



\h2{ Solución a $O (n \log n)$: dinámica de la programación con el binario de búsqueda }

Para obtener una resolución más rápida de tareas, construiremos otra variante de la programación dinámica por $O (n^2)$ y, a continuación, daremos cuenta de cómo esta la opción de acelerar hasta $O (n \log n)$.

\bf{Dinámica} ahora será tal: que $d[i]$ $(i = 0, \ldots, n)$ --- este es el número que termina con la creciente подпоследовательность de la longitud de la $i$ (y si estos números, varios --- es el menor de ellos).

Inicialmente, creemos $d[0] = -\infty$, y el resto de los elementos $d[i] = \infty$.

Considerar esta dinámica vamos poco a poco, tratando número $a[0]$, entonces $a[1],$, etc.

Presentamos la aplicación de esta dinámica, por $O (n^2)$:

\code
int d[MAXN];
d[0] = -INF;
for (int i=1; i<=n; ++i)
d[i] = INF;

for (int i=0; i<n; i++)
for (int j=1; j<=n; j++)
if (d[j-1] < a[i] && a[i] < d[j])
d[j] = a[i];
\endcode

Tenga en cuenta ahora que el esta dinámica, hay una \bf{propiedad muy importante de la}: $d[i-1] \le d[i]$ para todo $i = 1, \ldots, n$. Otra propiedad de --- que cada elemento de $a[i]$ actualiza hasta un máximo de una celda $d[j]$.

Por lo tanto, esto significa que procesar ordinaria $a[i]$ podemos $O (\log n)$, haciendo que la búsqueda binaria de la matriz $d[]$. En realidad, sólo estamos buscando en el array $d[]$ el primer número, que es estrictamente mayor que $a[i]$, y tratando de hacer la actualización de este elemento similar a la anterior aplicación.


\h3{ Aplicación en $O (n \log n)$ }

Utilizando el estándar en el lenguaje C++ del algoritmo de búsqueda binaria $upper\_bound$ (que devuelve la posición del primer elemento, estrictamente mayor de este), obtenemos esa simple aplicación de:

\code
int d[MAXN];
d[0] = -INF;
for (int i=1; i<=n; ++i)
d[i] = INF;

for (int i=0; i<n; i++) {
int j = int (upper_bound (d.begin () d.end(), a[i]) - d.begin());
if (d[j-1] < a[i] && a[i] < d[j])
d[j] = a[i];
}
\endcode


\h3{ la Recuperación de la respuesta }

De esta dinámica también puede restaurar la respuesta, para lo cual, de nuevo, además de la dinámica de $d[i]$ también es necesario almacenar una matriz de "antepasados" $p[i]$ --- entonces, en un elemento de cualquier índice que termina en una óptima подпоследовательность de la longitud de la $i$. Además, para cada elemento de la matriz $a[i]$ será necesario almacenar su "antepasado" --- es decir, el índice del elemento que debe preceder a la $a[i]$ óptima подпоследовательности.

Al mantener estas dos matrices sobre un curso de cálculo de la dinámica, al final, es fácil de recuperar lo que busca подпоследовательность.

(Es interesante señalar que, en cuanto a la dinámica de la respuesta solo puede restaurar así, a través de las matrices de los antepasados --- y sin ellos restaurar la respuesta después de que el cálculo de la dinámica no será posible. Este es uno de los raros casos, cuando a la dinámica no se aplica el método alternativo de recuperación --- sin matrices de los antepasados).



\h2{ Solución a $O (n \log n)$: estructuras de datos }

Si el método anterior por $O (n \log n)$ es muy hermoso, sin embargo, no trivial, ideológico, es decir, otra manera de aprovechar uno de los famosos simples estructuras de datos.

En realidad, volvamos a la primera dinámica, donde el estado era simplemente la posición actual. El valor actual de la dinámica de $d[i]$ se calcula como el máximo de los valores de $d[i]$ entre todos los elementos del $j$ que $a[j] < a[i]$.

Por lo tanto, si nosotros a través de $t[]$ denota un \bf{matriz}, en el que vamos a grabar los valores de la dinámica de los números:

$$ t[a[i]] = d[i], $$

resulta que todo lo que tenemos que ser capaces de: - - - es buscar \bf{máximo en el prefijo} de la matriz $t$: $t[0 \ldots a[i]-1]$.

La tarea de la búsqueda del máximo a prefijos de la matriz (teniendo en cuenta que la matriz puede variar) se soluciona muchas estándar, con estructuras de datos, por ejemplo, \algohref=segment_tree{árbol de regiones} o \algohref=fenwick_tree{árbol Fenwicks}.

Aprovechando cualquier estructura de datos que recibamos la decisión de la $O (n \log n)$.

Este método de solución que hay claras \bf{inconvenientes}: por la longitud y la complejidad de la implementación de esta ruta será, en cualquier caso, es peor que la descrita anteriormente dinámica en $O (n \log n)$. Además, si de entrada el número de $a[i]$ pueden ser bastante grandes, es probable que tenga que comprimir (es decir, volver a numerar desde $0 a$ a $n-1$) --- sin que ello muchos de los accesorios de la estructura de datos de trabajar no pueden debido al alto consumo de memoria.

Por otro lado, el camino es y \bf{ventajas}. En primer lugar, de esta manera, la decisión no tiene que preocuparse acerca de una dinámica. En segundo lugar, este método permite resolver algunas generalizaciones de nuestra tarea (véase a continuación).



\h2{ Conexos de la tarea }

Aquí algunas de las tareas estrechamente relacionadas con la tarea de búsqueda de наидлиннейшей creciente подпоследовательности.


\h3{ Наидлиннейшая неубывающая подпоследовательность }

De hecho, esta es la misma tarea, sólo que ahora en la подпоследовательности se permiten los números iguales (es decir, debemos encontrar débilmente creciente подпоследовательность).

La solución de esta tarea, de hecho no es diferente de la nuestra la tarea original, simplemente al realizar comparaciones cambian los signos de las desigualdades, y también será necesario modificar ligeramente la búsqueda binaria.


\h3{ Número de наидлиннейших crecientes подпоследовательностей }

Para ello, se puede utilizar la primera dinámica por $O (n^2)$ o enfoque mediante el uso de estructuras de datos para la solución de los $O (n \log n)$. Y, de hecho, en el caso de todos los cambios son sólo en el sentido de que, además de los valores de la dinámica de $d[i]$ es necesario almacenar, de cuántas maneras este valor puede ser obtenido.

Aparentemente, la solución a través de la dinámica de la $O (n \log n)$ a esta tarea no se puede aplicar.


\h3{ el Menor número de невозрастающих подпоследовательностей que cubren esta secuencia }

\bf{Condición} es tal. Dada la matriz de $n$ de números $a[0 \ldots, n-1]$. Desea pintar su número en el menor número de colores de modo que cada color se originaba sería невозрастающая подпоследовательность.

\bf{para Decisión}. Se afirma que la cantidad mínima necesaria de colores igual a la longitud de la наидлиннейшей creciente подпоследовательности.

\bf{Prueba}. De hecho, lo tiene que demostrar \bf{dualidad} de esta tarea y de la búsqueda de наидлиннейшей creciente подпоследовательности.

Se denota por $x$ longitud наидлиннейшей creciente подпоследовательности, y a través de la $y$ --- buscado el menor número de невозрастающих подпоследовательностей. Tenemos que demostrar que $x=y$.

Por un lado, claro, ¿por qué no puede ser $y<x$: después de todo, si tenemos $x$ estrictamente crecientes de los elementos, hay dos de ellos no pueden entrar en una невозрастающую подпоследовательность y, por lo tanto, $y \ge x$.

Mostraremos ahora que, por el contrario, $y$ no puede ser $> x$. Demostremos es a la inversa: se supone que $y > x$. Veamos entonces cualquier el conjunto óptimo de $y$ невозрастающих подпоследовательностей. Transformar este conjunto, por lo tanto: mientras hay dos подпоследовательности que la primera se inicia antes de la segunda, pero la primera comienza con un número mayor o igual que el inicio de la segunda --- отцепим es la pantalla de inicio, el número de la primera подпоследовательности y прицепим en el comienzo de la segunda. Por lo tanto, a través de un número finito de pasos nos quedará $y$ подпоследовательностей, y sus respectivas número formarán creciente подпоследовательность de longitud $y$. Pero $y > x$, es decir, hemos llegado a una contradicción (ya que no puede ser crecientes подпоследовательностей largo $x$).

Por lo tanto, en realidad, $y = x$, que se quería demostrar.

\bf{la Recuperación de la respuesta}. Se afirma que la búsqueda de una ruptura en la подпоследовательности puede buscar con avidez, es decir, yendo de izquierda a derecha y declarando que el número actual en la подпоследовательность, que ahora termina en menor número, mayor o igual a la actual.



\h2{ Tareas en línea judges }

La lista de tareas que se pueden resolver sobre el tema:

\ul{

\li \href=http://informatics.mccme.es/moodle/mod/statements/view3.php?chapterid=1793{MCCME #1793 \bf{"la Mayor creciente подпоследовательность por O(n*log(n))"} ~~~~ [dificultad: baja]}

\li \href=http://community.topcoder.com/stat?c=problem_statement&pm=5922&rd=8075{TopCoder SRM 278 \bf{"500 IntegerSequence"} ~~~~ [dificultad: baja]}

\li \href=http://community.topcoder.com/stat?c=problem_statement&pm=3937&rd=6532{TopCoder SRM 233 \bf{"DIV2 1000 AutoMarket"} ~~~~ [dificultad: baja]}

\li \href=http://codeforces.es/contest/76/problem/F{solicitó la ayuda de las olimpiadas escolares de ciencias de la computación --- la tarea de F \bf{"Turista"} ~~~~ [dificultad media]}

\li \href=http://codeforces.es/problemset/problem/10/D{Codeforces Beta Round #10 --- la tarea de la D \bf{"НОВП"} ~~~~ [dificultad media]}

\li \href=http://acm.tju.edu.cn/toj/showp2707.html{ACM.TJU.EDU.CN 2707 \bf{"Greatest Common Increasing Subsequence"} ~~~~ [dificultad media]}

\li \href=http://www.spoj.pl/problemas/SUPPER/{SPOJ #57 \bf{"SUPPER. Supernumbers in a permutation"} ~~~~ [dificultad media]}

\li \href=http://acm.sgu.es/problem.php?contest=0&problem=521{ACM.SGU.RU #521 \bf{"North-East"} ~~~~ [nivel de dificultad: alto]}

\li \href=http://community.topcoder.com/stat?c=problem_statement&pm=2967&rd=5881{TopCoder Open 2004 --- Round 4 --- \bf{"1000. BridgeArrangement"} ~~~~ [nivel de dificultad: alto]}

}













