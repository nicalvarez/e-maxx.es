\h1{ Emblemática área del triángulo y el predicado de "agujas del reloj" }


\h2{ Definición }

Que dados tres puntos $p_1$, $p_2$, $p_3$. Encontraremos el valor de \bf{simbólica de la plaza} $S$ triángulo de $p_1 p_2 p_3$, es decir, la plaza del triángulo, tomada con signo positivo o negativo dependiendo del tipo de giro, formado por los puntos de $p_1$, $p_2$, $p_3$: en sentido contrario a las agujas del reloj o en ella, respectivamente.

Claro que, si aprendemos a calcular esa histórica ("orientado") el área, lo que podemos encontrar regular el área de cualquier triángulo, así como podemos comprobar, en el sentido de las agujas del reloj o en contra de la busca de alguna clase de tres puntos.


\h2{ Cálculo }

Utilizaremos el concepto de \bf{oblicuo} (псевдоскалярного) de la obra de vectores. Es apenas igual al doble de la simbólica del área del triángulo:

$$ a \land b = |a| |b| \sin \angle (a, b) = 2 S, $$

donde el ángulo $\angle (a, b)$ se toma orientado, es decir, el ángulo de rotación de entre estos vectores en contra de las agujas del reloj.

(Módulo oblicuo de la multiplicación de dos vectores de igual módulo \bf{vectorial} de la obra.)

Косое la obra se calcula como el valor de un identificador compuesto de coordenadas de los puntos:

$$ 2 S = \left| \matrix{
x_1 & y_1 & 1 \cr
x_2 & y_2 & 1 \cr
x_3 & y_3 & 1 \cr
} \right| . $$

Revelador determinante, se puede obtener esta fórmula:

$$ 2 S = x_1 (y_2 - y_3) + x_2 (y_3 - y_1) + x_3 (y_1 - y_2). $$

Se puede agrupar el tercer término de los dos primeros, que renuncie de una multiplicación:

$$ 2 S = (x_2 - x_1) (y_3 - y_1) - (y_2 - y_1) (x_3 - x_1). $$

Última fórmula es conveniente grabar y recordar en forma de matriz como la siguiente determinante:

$$ 2 S = \left| \matrix{
x_2 - x_1 & y_2 - y_1 \cr
x_3 - x_1 & y_3 - y_1 \cr
} \right| . $$


\h2{ Aplicación }

La función que calcula la doble histórica el área del triángulo:

\code
int triangle_area_2 (int x1, int y1, int x2, int y2, int x3, int y3) {
return (x2 - x1) * (y3 - y1) - (y2 - y1) * (x3 - x1);
}
\endcode

Función de regular el área del triángulo:

\code
double triangle_area (int x1, int y1, int x2, int y2, int x3, int y3) {
return abs (triangle_area_2 (x1, y1, x2, y2, x3, y3)) / 2.0;
}
\endcode

La función que verifica, forma si la troika de los puntos de giro en el sentido contrario al de las agujas del reloj:

\code
bool clockwise (int x1, int y1, int x2, int y2, int x3, int y3) {
return triangle_area_2 (x1, y1, x2, y2, x3, y3) < 0;
}
\endcode

La función que verifica, forma si la troika de los puntos de giro hacia la izquierda:

\code
bool counter_clockwise (int x1, int y1, int x2, int y2, int x3, int y3) {
return triangle_area_2 (x1, y1, x2, y2, x3, y3) > 0;
}
\endcode