\h1{ Binario exponenciación }


Binario (binary) exponenciación --- es una técnica que permite construir cualquier número en $n$-ésimo grado $O(\log n)$ умножений (en lugar de $n$ умножений durante el uso normal de enfoque).

Además, se describe aquí recepción aplica a cualquier \bf{asociativa} la operación, y no sólo a la multiplicación de los números. Recordemos que la operación se denomina asociativa, si para cualquier $a, b, c$ se:
$ $ a \cdot b) \cdot c = a \cdot b \cdot c) $$

El más obvio síntesis de la --- a los restos de un módulo (obviamente, la asociatividad se guarda). La siguiente de la "popularidad" es una síntesis de la obra de las matrices (su asociatividad conocido en general).



\h2{ el Algoritmo }

Tenga en cuenta que para cualquier cantidad de $a$ y \bf{un} número de $n$ es factible la aparente identidad (el siguiente de la asociatividad de la multiplicación):
$$ a^n = a^{n/2})^2 = a^{n/2} \cdot a^{n/2} $$
Y esta es la principal en el método binario de exponenciación. De hecho, para el par $n$ nos mostraron cómo, después de una operación de multiplicación, se puede reducir el problema a la mitad menor medida.

Queda por entender que hacer, si el grado de $n$ \bf{нечетна}. Aquí lo hacemos es muy simple: pasar a la medida $n-1$, que será ya el par:
$$ a^n = a^{n-1} \cdot a $$

Así que, en realidad, hemos encontrado рекуррентную la fórmula: de grado $n$ pasamos, si se четна, a $n/a 2$, y de otro modo --- a $n-1$. Está claro que todo no será más de $2 \log n$ de navegación, antes de que lleguemos a $n = 0$ (base recurrente de la fórmula). Por lo tanto, tenemos un algoritmo que trabaja en el $O (\log n)$ умножений.



\h2{ Aplicación }

Sencillo recursiva de la implementación:

\code
int binpow (int a, int n) {
if (n == 0)
return 1;
if (n % 2 == 1)
return binpow (a, n-1) * a;
else {
int b = binpow (a, n/2);
return b * b;
}
}
\endcode

Нерекурсивная la implementación, también optimizado (división en 2 reemplazados por las operaciones bit a bit):

\code
int binpow (int a, int n) {
int res = 1;
while (n)
if (n & 1) {
res *= a;
--n;
}
else {
a *= a;
n >>= 1;
}
return res;
}
\endcode

Esta aplicación se puede simplificar aún más, al darse cuenta de que la construcción del $a$ el cuadrado se realiza siempre, independientemente de que funcione la condición de par $n$ o no:

\code
int binpow (int a, int n) {
int res = 1;
while (n) {
if (n & 1)
res *= a;
a *= a;
n >>= 1;
}
return res;
}
\endcode

Finalmente, vale la pena señalar que el binario exponenciación ya se ha implementado en el lenguaje Java, pero sólo para la clase de la larga aritmética BigInteger (función pow esta clase funciona es por el algoritmo binario de la edificación).



\h2{ Ejemplos de la solución de tareas }


\h3{ Eficiente el cálculo de los números de fibonacci }

\bf{Condición}. Dado el número de $n$. Se desea calcular el $F_n$, donde $F_i$ --- \algohref=fibonacci_numbers{secuencia de números de fibonacci}.

\bf{para Decisión}. Con más detalle esta solución se describe en \algohref=fibonacci_numbers{artículo acerca de la secuencia de fibonacci}. Aquí sólo nos presentamos brevemente la esencia de esta decisión.

La idea básica es la siguiente. Calcular el siguiente número de fibonacci se basa en el conocimiento de las dos anteriores números de fibonacci, es decir, cada número de fibonacci se obtiene como la suma de los dos anteriores. Esto significa que podemos construir la matriz de $2 \times 2$, que se ajuste a esta transformación: como de dos números de fibonacci $F_i$ y $F_{i+1}$ calcular el siguiente número, es decir, ir a la par de $F_{i+1}$, $F_{i+2}$. Por ejemplo, la aplicación de esta conversión $n$ vez a un par de $F_0$ y $F_1$, obtenemos un par de $F_n$ y $F_{n+1}$. Por lo tanto, alzando la matriz de esta conversión en $n$-ésimo grado, por lo tanto, encontraremos la búsqueda de $F_n$ por hora $O (\log n)$, que nosotros y era necesario.


\h3{ Construcción de una permutación en $k$-ésimo grado }

\bf{Condición}. Dada una permutación de $p$ de longitud $n$. Es necesario construir en $k$-ésimo grado, es decir, qué pasa si a тождественной cambiar de lugar de $k$ en vez de aplicar una permutación de $p$.

\bf{para Decisión}. Sólo se aplica a cambiar de lugar de $p$ descrito anteriormente, el algoritmo de exponenciación binaria. No hay diferencia en comparación con la construcción de los números en el grado - - - - no. La solución se obtiene con la асимптотикой $O (n \log k)$.

(Nota. Esta tarea se puede resolver de manera más efectiva, \bf{por lineal del tiempo}. Para ello, basta con seleccionar en el intercambio de todos los ciclos, después de lo cual a examinar por separado cada uno de los bucles, y tomando $k$ por el módulo de la longitud de ciclo actual, encontrar la respuesta para este ciclo.)


\h3{ Rápida aplicación de un conjunto de operaciones geométricas a los puntos }

\bf{Condición}. Dado $n$ puntos $p_i$, y se les da $m$ transformaciones que se deben aplicar a cada uno de estos puntos. Cada transformación --- es el desplazamiento en un vector o un escalar (multiplicación de coordenadas en la configuración de los factores), o la rotación alrededor del eje dado en un ángulo determinado. Además, hay un componente de operación para el ciclo: tiene el aspecto de un "repetir el número de veces especificado lista de transformaciones" (la operación de ciclo se pueden anidar unos dentro de otros).

Se desea calcular el resultado de la aplicación de las operaciones a todos los puntos de (efectivamente, es decir, por el tiempo menor que $O(n \cdot length)$, donde $length --- el número total de operaciones que hay que hacer).

\bf{para Decisión}. Veamos los diferentes tipos de transformaciones desde el punto de vista de cómo se modifican las coordenadas:

\ul{

\li Operación de desplazamiento --- simplemente se añade a todas las coordenadas de la unidad, домноженную en algunas constantes.

\li Operación de zoom --- se multiplica cada coordenada en cierta constante.

\li Operación de rotación alrededor del eje --- también se puede presentar de la siguiente manera: nuevos derivados de coordenadas se puede escribir como combinación lineal de los antiguos.

(No vamos aquí precisar la manera en que ésta se realiza. Por ejemplo, para la facilidad de presentar, en la forma de combinación de cinco bidimensionales de curvas: primero en los planos de $OXY$ y $OXZ$ de manera que el eje de rotación coincide con la dirección positiva del eje de $OX$ y, a continuación, necesario girar alrededor de un eje en el plano de $YZ$ y, a continuación, las devoluciones de los giros en los planos de $OXZ$ y $OXY$ de manera que el eje de rotación de vuelta en su posición original.)

}

Como es fácil de ver, cada una de estas transformaciones --- es el cálculo de coordenadas lineal de las ecuaciones. Por lo tanto, toda esta transformación se puede escribir en forma de matriz $4 \times 4$:

$$ \begin{pmatrix}
a_11 & a_{12} & a_{13} & a_{14} \\
a_21 & a_{22} & a_{23} & a_{24} \\
a_31 & a_{32} & a_{33} & a_{34} \\
a_41 & a_{42} & a_{43} & a_{44} \\
\end{pmatrix}, $$

que al multiplicar (a la izquierda) en la línea de las antiguas coordenadas constantes y unidades da la línea de las nuevas coordenadas y constantes-unidades:

$$ \begin{pmatrix} x & y & z & 1 \end{pmatrix} \cdot \begin{pmatrix}
a_{11} & a_{12} & a_{13} & a_{14} \\
a_{21} & a_{22} & a_{23} & a_{24} \\
a_{31} & a_{32} & a_{33} & a_{34} \\
a_{41} & a_{42} & a_{43} & a_{44} \\
\end{pmatrix} = \begin{pmatrix} x' y y' y z' & 1 \end{pmatrix}. $$

(¿Por qué necesitó introducir ficticia a la cuarta coordenada, siempre igual a la unidad? Sin esto no sería implementar la operación de cambio: ya que el desplazamiento es-es justamente la suma de las coordenadas de la unidad, домноженной en algunos factores. Sin ficticia de unidades podríamos sólo realizar combinaciones lineales de las propias coordenadas, y sumar a ellos especificados constantes --- no podríamos hacer.)

Ahora, la tarea se vuelve casi trivial. Una vez cada operación elemental se describe la matriz, la secuencia de operaciones se describe una obra de estas matrices, y la operación para el ciclo --- la construcción de esta matriz en el grado. Por lo tanto, estamos a tiempo $O (m \cdot \log repetition)$ podemos предпосчитать la matriz de $4 \times 4$, describe la conversión de todo, y entonces hay que multiplicar cada punto de $p_i$ en esta matriz --- por tanto, vamos a responder a todas las consultas por hora $O (n)$.


\h3{ el Número de caminos de longitud fija en la columna }

\bf{Condición}. Dan неориентированный el conde de $G$ c $n$ vértices, y dado el número de $k$. Se requiere para cada par de vértices $i$ y $j$ encontrar el número de rutas entre ellos, contienen exactamente $k$ de las costillas.

\bf{para Decisión}. Más detalles de esta tarea se aborda en \algohref=fixed_length_paths{un artículo separado}. Aquí sólo recordaremos la esencia de esta solución: simplemente estamos construyendo en $k$-ésimo grado de la matriz de adyacencia de este conde, y los elementos de esta matriz es y será la búsqueda de soluciones. Total asíntotas --- $O (n^3 \log k)$.

(Nota. En \algohref=fixed_length_paths{artículo} y la otra variación de esta tarea: cuando el conde de asma, y es necesario encontrar el camino mínimo de peso, contiene exactamente $k$ de las costillas. Como se muestra en este artículo, esta tarea también se resuelve mediante el binario de exponenciación de la matriz de adyacencia de un conde, sin embargo, en lugar de una operación normal перемножения de las dos matrices deben utilizar modificada: en lugar de умножений es la cantidad, sino que en lugar de sumar --- toma el mínimo.)


\h3{ Variación binario de exponenciación: multiplicar dos números de módulo }

Aquí una interesante variación binario de exponenciación.

Supongamos que nos enfrentamos a esta \bf{meta}: multiplicar dos números de $a$ y $b$ por módulo $m$:

$$ a \cdot b \pmod m $$

Supongamos que los números pueden ser lo suficientemente grandes: tanto, que los números se colocan en los tipos de datos integrados, y aquí está su obra directa $a \cdot b$ --- ya no (tenga en cuenta que también necesitaremos para la suma de los números se situaba en el tipo de datos). En consecuencia, el reto consiste en contar lo que busca el valor de $a \cdot b) \pmod m$, sin tener que recurrir a \algohref=big_integer{larga de la aritmética}.

\bf{para Decisión} es tal. Nosotros simplemente aplicamos el algoritmo binario de la edificación, descrito anteriormente, sólo que en lugar de la multiplicación vamos a producir suma. En otras palabras, multiplicar dos números nos han reducido a $O (\log m)$ de las operaciones de la suma y la multiplicación por dos (que también, de hecho, hay sumar).

(Nota. Esta tarea se puede resolver y \bf{otro}, con la ayuda de las operaciones con números de coma flotante. Es decir, contaremos en los números de punto flotante, la expresión $a \cdot b / m$, y округлим su más cercano con el todo. Así nos encontraremos \bf{aproximada} privada. Ruina de la obra $a \cdot b$ (ignorar el desbordamiento), lo más probable es que obtengamos un número relativamente reducido, que puede tomar por módulo $m$ --- y traerlo de vuelta como respuesta. Esta solución parece bastante fiable, pero es muy rápido, y muy brevemente, se ha implementado).



\h2{ Tareas en línea judges }

La lista de tareas que se pueden resolver utilizando binario exponenciación:

\ul{

\li \href=http://acm.sgu.es/problem.php?contest=0&problem=265{SGU #265 \bf{"Wizards"} ~~~~ [dificultad media]}

}

