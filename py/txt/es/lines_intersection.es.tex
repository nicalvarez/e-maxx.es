\h1{el Punto de intersección de rectas}

Que nos son dadas dos rectas definidas sus cuotas de $A_1, B_1, C_1$ y $A_2, B_2, C_2$. Es necesario encontrar el punto de intersección, o averiguar lo que rectas son paralelas.

\h2{para Decisión}

Si dos rectas no son paralelas, sino que se cruzan. Para encontrar el punto de intersección, basta de hacer de las dos ecuaciones directas de un sistema y resolver:
$$ \cases{ A_1 x + B_1 y + C_1 = 0, \cr
A_2 x + B_2 y + C_2 = 0. } $$

Usando la fórmula de kramer, en seguida encontramos la solución del sistema, la cual será considerada \bf{el punto de intersección}:

$$ x = - \frac{ \left|\matrix{C_1&B_1 \cr C_2&B_2}\right| }{ \left|\matrix{A_1&B_1 \cr A_2&B_2}\right| } = - \frac{ C_1 B_2 - C_2 B_1 }{ A_1 B_2 - A_2 B_1 }, $$
$$ y = - \frac{ \left|\matrix{A_1&C_1 \cr A_2&C_2}\right| }{ \left|\matrix{A_1&B_1 \cr A_2&B_2}\right| } = - \frac{ A_1 C_2 - A_2 C_1 }{ A_1 B_2 - A_2 B_1 }. $$

Si el denominador es cero, es decir,
$$ \left|\matrix{A_1&B_1 \cr A_2&B_2}\right| = A_1 B_2 - A_2 B_1 = 0 $$
el sistema de decisiones no tiene directas (\bf{paralelas} y no es la misma) o tiene un número infinito directas (\bf{coinciden}). Si es necesario distinguir entre los dos casos, es necesario comprobar que los factores de $C$ rectas son proporcionales con el mismo coeficiente de proporcionalidad, que los factores de $A$ y $B$, para lo cual es suficiente contar dos identificador, si ambos son iguales a cero, entonces rectas coinciden:

$$ \left|\matrix{ A_1 & C_1 \cr A_2 & C_2 }\right|, \left|\matrix{ B_1 & C_1 \cr B_2 & C_2 }\right| $$

\h2{Aplicación}

\code
struct pt {
double x, y;
};

struct line {
double a, b, c;
};

const double EPS = 1e-9;

double det (double a, double b, double c, double d) {
return a * d - b * c;
}

bool intersect (line m, line n, pt & res) {
double zn = det (m.a, m.b, n.a, n.b);
if (abs (zn) < EPS)
return false;
res.x = - det (m.c, m.b, n.c, n.b) / zn;
res.y = - det (m.a, m.c, n.a, n.c) / zn;
return true;
}

bool parallel line m, line n) {
return abs (det (m.a, m.b, n.a, n.b)) < EPS;
}

bool equivalent (line m, line n) {
return abs (det (m.a, m.b, n.a, n.b)) < EPS
&& abs (det (m.a, m.c, n.a, n.c)) < EPS
&& abs (det (m.b, m.c, n.b, n.c)) < EPS;
}
\endcode