<h1>los algoritmos Eficaces факторизации</h1>

<hr>

<p>Aquí se incluyen la implementación de varios de los algoritmos de факторизации, cada uno de ellos por separado puede trabajar tan rápido, y muy lentamente, pero en la suma de dan muy rápido.</p>
<p>la Descripción de estos métodos, no se, más bien se describe en la web.</p>

<h2>Método de Полларда p-1</h2>
<p>la Primera prueba, rápidamente da respuesta ni mucho menos para todos los números.</p>
<p>Devuelve o encontrado el divisor, o 1, si el divisor no ha sido encontrado.</p>
<code>template <class T>
T pollard_p_1 (T n)
{
// los parámetros del algoritmo, afectan la productividad y la calidad de búsqueda
const T b = 13;
const T q[] = { 2, 3, 5, 7, 11, 13 };

// varios intentos de algoritmo
T a = 5 % n;
for (int j=0; j<10; j++)
{

// buscamos es a que mutuamente simplemente con n
while (gcd (a, n) != 1)
{
mulmod (a, a, n);
a += 3;
a %= n;
}

// calcula el a^M
for (size_t i = 0; i < sizeof q / sizeof q[0]; i++)
{
T qq = q[i];
T e = (T) floor (log ((double)b) / log ((double)qq));
T aa = powmod (a, powmod (qq, e, n), n);
if (aa == 0)
continue;

// comprobamos que no se ha encontrado la respuesta
T g = gcd (aa-1, n);
if (1 < g && g < n)
return g;
}

}

// si no encontraron nada
return 1;

}</code>

<h2>Método de Полларда "Ro"</h2>
<p>la Primera prueba, rápidamente da respuesta ni mucho menos para todos los números.</p>
<p>Devuelve o encontrado el divisor, o 1, si el divisor no ha sido encontrado.</p>
<code>template <class T>
T pollard_rho (T n, unsigned iterations_count = 100000)
{
T
b0 = rand() % n,
b1 = b0,
g;
mulmod (b1, b1, n);
if (++b1 == n)
b1 = 0;
g = gcd (abs (b1 - b0), n);
for (unsigned count=0; count<iterations_count && (c == 1 || g == n); count++)
{
mulmod (b0, b0, n);
if (++b0 == n)
b0 = 0;
mulmod (b1, b1, n);
++b1;
mulmod (b1, b1, n);
if (++b1 == n)
b1 = 0;
g = gcd (abs (b1 - b0), n);
}
return g;
}</code>

<h2>Método de Бента (una modificación del método de Полларда "Rho")</h2>
<p>la Primera prueba, rápidamente da respuesta ni mucho menos para todos los números.</p>
<p>Devuelve o encontrado el divisor, o 1, si el divisor no ha sido encontrado.</p>
<code>template <class T>
T pollard_bent (T n, unsigned iterations_count = 19)
{
T
b0 = rand() % n,
b1 = (b0*b0 + 2) % n,
a = b1;
for (unsigned iteration=0, series_len=1; iteration<iterations_count; iteration++, series_len*=2)
{
T g = gcd (b1-b0, n);
for (unsigned len=0; len<series_len && (c==1 && g==n); len++)
{
b1 = (b1*b1 + 2) % n;
g = gcd (abs(b1-b0), n);
}
b0 = a;
a = b1;
if (c != 1 && g != n)
return g;
}
return 1;
}</code>

<h2>Método de Полларда de monte-carlo</h2>
<p>la Primera prueba, rápidamente da respuesta ni mucho menos para todos los números.</p>
<p>Devuelve o encontrado el divisor, o 1, si el divisor no ha sido encontrado.</p>
<code>template <class T>
T pollard_monte_carlo (T n, unsigned m = 100)
{
T b = rand() % (m-2) + 2;

static std::vector<T> primos;
static T m_max;
if (primos.empty())
primos.push_back (3);
if (m_max < m)
{
m_max = m;
for (T prime=5; prime<=m; ++++ prime)
{
bool is_prime = true;
for (std::vector<T>::const_iterator iter=primos.begin(), end=primos.end();
iter!=end; ++iter)
{
T div = *iter;
if (div*div > prime)
break;
if (prime % div == 0)
{
is_prime = false;
break;
}
}
if (is_prime)
primos.push_back (prime);
}
}

T g = 1;
for (size_t i=0; i<primos.size() && g==1; i++)
{
T cur = primos[i];
while (cur <= n)
cur *= primos[i];
cur /= primos[i];
b = powmod (b, cur, n);
g = gcd (abs(b-1), n);
if (c == n)
g = 1;
}

return g;
}</code>

<h2>Método de Granja</h2>
<p>este Es un método cien por ciento, pero puede funcionar muy lentamente, si tiene el número de pequeñas rias.</p>
<p>Por lo tanto ejecutar la pena sólo después de todos los demás métodos.</p>
<code>template <class T, class T2>
T ferma (const T & n, T2 unused)
{
T2
x = sq_root (n),
y = 0,
r = x*x - y*y - n;
for (;;)
if (r == 0)
return x!=y ? x-y : x+y;
else
if (r > 0)
{
r -= y+y+1;
++y;
}
else
{
r += x+x+1;
++x;
}
}</code>

<h2>Trivial división</h2>
<p>Este elemental técnica es útil para procesar números muy pequeños делителями.</p>
<code>template <class T, class T2>
T2 prime_div_trivial (const T & n, T2 m)
{

// primero comprobamos triviales de los casos
if (n == 2 || n == 3)
return 1;
if (n < 2)
return 0;
if (even (n))
return 2;

// generamos simples de 3 a m
T2 pi;
const vector<T2> & primos = get_primes (m, pi);

// dividimos en todas las cosas simples
for (std::vector<T2>::const_iterator iter=primos.begin(), end=primos.end();
iter!=end; ++iter)
{
const T2 & vid = *iter;
if (div * div > n)
break;
else
if (n % div == 0)
return div;
}

if (n < m*m)
return 1;
return 0;

}</code>

<h2>Recogemos todos juntos</h2>
<p>Combinar todos los métodos en una sola función.</p>
<p>También la función utiliza la prueba de la sencillez, de lo contrario, los algoritmos de факторизации pueden trabajar durante mucho tiempo. Por ejemplo, puede seleccionar una prueba BPSW (<algohref=bpsw>leer el artículo de BPSW</algohref>).</p>
<code>template <class T, class T2>
void factorize (const T & n, std::map<T,unsigned> & result, T2 unused)
{
if (n == 1)
;
else
// comprobamos que no es fácil, si el número de
if (isprime (n))
++result[n];
else
// si el número es pequeña, su разлагаем fuerza bruta
if (n < 1000*1000)
{
T div = prime_div_trivial (n, 1000);
++result[div];
factorize (n / div, result, unused);
}
else
{
// el número es grande, lanzando sobre él algoritmos факторизации
T div;
// primero van rápidos algoritmos Полларда
div = pollard_monte_carlo (n);
if (div == 1)
div = pollard_rho (n);
if (div == 1)
div = pollard_p_1 (n);
if (div == 1)
div = pollard_bent (n);
// tiene que ejecutar el 100% de los casos el algoritmo de la Granja
if (div == 1)
div = ferma (n, unused);
// procesamos de forma recursiva encontrados multiplicadores
factorize (div, result, unused);
factorize (n / div, result, unused);
}
}</code>

<hr>

<h2>Aplicación</h2>
<p><a href="factorization.zip">Descargar [5 kb]</a> el codigo fuente de un programa que utiliza todos los métodos de факторизации y la prueba de BPSW en la simplicidad.</p>