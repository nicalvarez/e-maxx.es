\h2{Número de marcado de los condes}

Dado el número de $N$ vértices. Es necesario calcular la cantidad de $G_N$ diferentes marcados de gráficos con $N$ vértices (es decir, los vértices del grafo se marcan los diferentes números de $1 a$ a $N$, y los gráficos se comparan con la función de la pintura de los vértices). La aleta del conde de неориентированы, bisagras y múltiples costillas están prohibidos.

Consideremos el conjunto de todas las posibles aristas de la gráfica. Para cualquier aleta $(i,j)$ supongamos que $i<j$ (basado en el неориентированности conde y la ausencia de bucles). Entonces el conjunto de todas las posibles aristas conde tiene la potencia de $C_N^2$, es decir $\frac{ N (N-1) }{ 2 }$.

Porque cualquier marcado el conde se identifica por sus costillas, el número de grafos marcados con $N$ vértices es igual a:
$$ G_N = 2^{ \frac{ N (N-1) }{ 2 } } $$

\h2{Número de enlaces marcados de los condes}

En comparación con el anterior objetivo, además, ponemos la limitación de que el gráfico debe ser coherente..

Se denota la búsqueda de número a través de $Conn_N$.

Aprendamos, por el contrario, contar la cantidad de \bf{несвязных} de los condes; entonces el número de enlaces de los condes consigue $G_N$ menos se halló un número. Además, aprenderemos a contar la cantidad de \bf{raíz} (es decir, con el resaltado de la cima de la raíz) \bf{несвязных de los condes}; entonces el número de несвязных de los condes salga de él, la división en $N$. Tenga en cuenta que, así como el conde de incoherente complaciente, en él hay un componente de conectividad, dentro de la cual está la raíz y el resto del conde constituiría aún más (como mínimo una) el componente de conectividad.

Переберем la cantidad de $K$ vértices en este componente de la conectividad, contiene la raíz (obviamente, $K = 1 \ldots, N-1$), y encontraremos el número de estos grafos. En primer lugar, tenemos que elegir el $K$ vértices de $N$, es decir, la respuesta se multiplica por $C_N^K$. En segundo lugar, el componente de conectividad con la raíz da un multiplicador de $Conn_K$. En tercer lugar, el resto de los conde de $N-K$ vértices es arbitrario por el conde, y por eso se da un multiplicador de $G_{N-K}$. Por último, el número de maneras de seleccionar una raíz en el componente de conectividad de $K$ vértices es igual a $K$. Total, si se fija $K$ el número de \bf{raíz несвязных} condes es igual a:
$$ K\ C_N^K\ Conn_K\ G_{N-K} $$

Significa la cantidad de \bf{несвязных} gráficos con $N$ vértices es igual a:
$$ \frac{1}{N} \sum_{K=1}^{N-1} K\ C_N^K\ Conn_K\ G_{N-K} $$

Por último, buscando \bf{nexos} condes es igual a:
$$ Conn_N = G_N - \frac{1}{N} \sum_{K=1}^{N-1} K\ C_N^K\ Conn_K\ G_{N-K} $$

\h2{Número de grafos marcados con $K$ componentes de la conectividad}

Basándose en la fórmula anterior, aprenderemos a contar el número de grafos marcados con $N$ vértices y $K$ componentes de la conectividad.

Esto se puede hacer a través de la programación dinámica. Aprenderemos a contar \bf{$D[N][K]$} --- cantidad de grafos marcados con $N$ vértices y $K$ componentes de la conectividad.

Aprenderemos a calcular el siguiente elemento $D[N][K]$, sabiendo los valores anteriores. Usaremos el estándar de tomar la decisión de tales tareas: tomaremos a la cima con el número 1, que pertenece a algún tipo de componente este componente es lo que vamos a tocar. Переберем el tamaño de la $S$ de este componente, entonces el número de maneras de elegir este tipo de conjunto de vértices es igual a $C_{N-1}^{S-1}$ (una cima -de-la - cima 1 --- la enumeración no es necesario). El número de maneras de construir el componente de conectividad de $S$ vértices de nosotros ya sabemos leer - - - $Conn_S$. Después de la eliminación de este componente del gráfico nos queda el conde con $N S$ vértices y $K-1$ los componentes de la conectividad, es decir, hemos recibido рекуррентную la dependencia, por la que se puede calcular un valor de $D[][]$:

$$ D[N][K] = \sum_{S=1}^{N} C_{N-1}^{S-1}\ Conn_S\ D[N-S][K-1] $$

Total obtenemos un código parecido a este:
\code
int d[n+1][k+1]; // inicialmente está lleno de ceros
d[0][0][0] = 1;
for (int i=1; i<=n; ++i)
for (int j=1; j<=i && j<=k; j++)
for (int s=1; s<=i; ++s)
d[i][j] += C[i-1][s-1] * conn[s] * d[i-s][j-1];
cout << d[n][k][n];
\endcode

Por supuesto, en la práctica, probablemente, se necesita \algohref=big_integer{larga aritmética}.