<h1>Нахождение наибольшего по весу вершинно-взвешенного паросочетания</h1>
<p>Дан двудольный граф G. Для каждой вершины первой доли указан её вес. Требуется найти паросочетание наибольшего веса, т.е. с наибольшей суммой весов насыщенных вершин.</p>
<p>Ниже мы опишем и докажем алгоритм, основанный на <algohref=kuhn_matching>алгоритме Куна</algohref>, который будет находить оптимальное решение.</p>
<h2>Алгоритм</h2>
<p>Сам алгоритм чрезвычайно прост. <b>Отсортируем</b> вершины первой доли в порядке убывания (точнее говоря, невозрастания) весов, и применим к полученному графу <b><algohref=kuhn_matching>алгоритм Куна</algohref></b>.</p>
<p>Утверждается, что полученное при этом максимальное (с точки зрения количества рёбер) паросочетание будет и оптимальным с точки зрения суммы весов насыщенных вершин (несмотря на то, что после сортировки мы фактически больше не используем эти веса).</p>
<p>Таким образом, реализация будет примерно такой:</p>
<code>int n;
vector < vector<int> > g (n);
vector<char> used (n);
vector<int> order (n); // список вершин, отсортированный по весу
... чтение ...

for (int i=0; i<n; ++i) {
	int v = order[i];
	used.assign (n, false);
	try_kuhn (v);
}</code>
<p>Функция try_kuhn() берётся безо всяких изменений из алгоритма Куна.</p>
<h2>Доказательство</h2>
<p>Напомним основные положения <b>теории матроидов</b>.</p>
<p>Матроид M - это упорядоченная пара (S,I), где S - некоторое множество, I - непустое семейство подмножеств множества S, которые удовлетворяют следующим условиям:</p>
<ol>
<li>Множество S конечное.</li>
<li>Семейство I является наследственным, т.е. если какое-то множество принадлежит I, то все его подмножества также принадлежат I.</li>
<li>Структура M обладает свойством замены, т.е. если A∈I, и B∈I, и |A|<|B|, то найдётся такой элемент x∈A-B, что A∪x∈I.</li>
</ol>
<p>Элементы семейства I называются независимыми подмножествами.</p>
<p>Матроид называется взвешенным, если для каждого элемента x∈S определён некоторый вес. Весом подмножества называется сумма весов его элементов.</p>
<p>Наконец, важнейшая теорема в теории взвешенных матроидов: чтобы получить оптимальный ответ, т.е. независимое подмножество с наибольшим весом, нужно действовать жадно: начиная с пустого подмножества, будем добавлять (если, конечно, текущий элемент можно добавить без нарушения независимости) все элементы по одному в порядке уменьшения (точнее, невозрастания) их весов:</p>
<code>отсортировать множество S по невозрастанию веса;
ans = [];
foreach (x in S)
	if (ans ∪ x ∈ I)
		ans = ans ∪ x;</code>
<p>Утверждается, что по окончании этого процесса мы получим подмножество с наибольшим весом.</p>
<p>Теперь <b>докажем</b>, что <b>наша задача -</b> не что иное, как взвешенный <b>матроид</b>.</p>
<p>Пусть S - множество всех вершин первой доли. Чтобы свести нашу задачу в двудольном графе к матроиду относительно вершин первой доли, поставим в соответствие каждому паросочетанию такое подмножество S, которое равно множеству насыщенных вершин первой доли. Можно также определить и обратное соответствие (из множества насыщенных вершин - в паросочетание), которое, хотя и не будет однозначным, однако вполне нас будет устраивать.</p>
<p>Тогда определим семейство I как семейство таких подмножеств множества S, для которых найдётся хотя бы одно соответствующее паросочетание.</p>
<p>Далее, для каждого элемента S, т.е. для каждой вершины первой доли, по условию определён некоторый вес. Причём вес подмножества, как нам и требуется в рамках теории матроидов, определяется как сумма весов элементов в нём.</p>
<p>Тогда задача о нахождении паросочетания наибольшего веса теперь переформулируется как задача нахождения независимого подмножества наибольшего веса.</p>
<p>Осталось проверить, что выполнены 3 вышеописанных условия, наложенных на матроид. Во-первых, очевидно, что S является конечным. Во-вторых, очевидно, что удаление ребра из паросочетания эквивалентно удалению вершины из множества насыщенных вершин, а потому свойство наследственности выполняется. В-третьих, как следует из корректности алгоритма Куна, если текущее паросочетание не максимально, то всегда найдётся такая вершина, которую можно будет насытить, не удаляя из множества насыщенных вершин другие вершины.</p>
<p>Итак, мы показали, что наша задача является взвешенным матроидом относительно множества насыщенных вершин первой доли, а потому к ней применим жадный алгоритм.</p>
<p>Осталось показать, что <b>алгоритм Куна является этим жадным алгоритмом</b>.</p>
<p>Однако это довольно очевидный факт. Алгоритм Куна на каждом шаге пытается насытить текущую вершину - либо просто проводя ребро в ненасыщенную вершину второй доли, либо находя удлиняющую цепь и чередуя паросочетание вдоль неё. И в том, и в другом случае никакие уже насыщенные вершины не перестают быть ненасыщенными, а ненасыщенные на предыдущих шагах вершины первой доли не насыщаются и на этом шаге. Таким образом, алгоритм Куна является жадным алгоритмом, строящим оптимальное независимое подмножества матроида, что и завершает наше доказательство.</p>