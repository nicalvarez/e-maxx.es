\h1{ Модульное линейное уравнение первого порядка }

\h2{ Постановка задачи }

Это уравнение вида:

$$a \cdot x = b \pmod n,$$

где $a, b, n$ --- заданные целые числа, $x$ --- неизвестное целое число.

Требуется найти искомое значение $x$, лежащее в отрезке $[0; n-1]$ (поскольку на всей числовой прямой, ясно, может существовать бесконечно много решений, которые будут отличаться друг друга на $n \cdot k$, где $k$ --- любое целое число). Если решение не единственно, то мы рассмотрим, как получить все решения.


\h2{ Решение с помощью нахождения Обратного элемента }

Рассмотрим сначала более простой случай --- когда $a$ и $n$ \bf{взаимно просты}. Тогда можно найти \algohref=reverse_element{обратный элемент} к числу $a$, и, домножив на него обе части уравнения, получить решение (и оно будет \bf{единственным}):

$$x = b \cdot a^{-1} \pmod n$$

Теперь рассмотрим случай, когда $a$ и $n$ \bf{не взаимно просты}. Тогда, очевидно, решение будет существовать не всегда (например, $2 \cdot x = 1 \pmod 4$).

Пусть $g = {\rm gcd(a,n)}$, т.е. их \algohref=euclid_algorithm{наибольший общий делитель} (который в данном случае больше единицы).

Тогда, если $b$ не делится на $g$, то решения не существует. В самом деле, при любом $x$ левая часть уравнения, т.е. $(a \cdot x) \pmod n$, всегда делится на $g$, в то время как правая часть на него не делится, откуда и следует, что решений нет.

Если же $b$ делится на $g$, то, разделив обе части уравнения на это $g$ (т.е. разделив $a$, $b$ и $n$ на $g$), мы придём к новому уравнению:

$$a^\prime \cdot x = b^\prime \pmod {n^\prime}$$

в котором $a^\prime$ и $n^\prime$ уже будут взаимно просты, а такое уравнение мы уже научились решать. Обозначим его решение через $x^\prime$.

Понятно, что это $x^\prime$ будет также являться и решением исходного уравнения. Однако если $g > 1$, то оно будет \bf{не единственным} решением. Можно показать, что исходное уравнение будет иметь ровно $g$ решений, и они будут иметь вид:

$$x_i = (x^\prime + i \cdot n^\prime) \pmod n,$$
$$i = 0 \ldots (g-1).$$

Подводя итог, можно сказать, что \bf{количество решений} линейного модульного уравнения равно либо $g = {\rm gcd(a,n)}$, либо нулю.

\h2{ Решение с помощью Расширенного алгоритма Евклида }

Приведём наше модулярное уравнение к диофантову уравнению следующим образом:

$$a \cdot x + n \cdot k = b,$$

где $x$ и $k$ --- неизвестные целые числа.

Способ решения этого уравнения описан в соответствующей статье \algohref=diofant_2_equation{Линейные диофантовы уравнения второго порядка}, и заключается он в применении \algohref=extended_euclid_algorithm{Расширенного алгоритма Евклида}.

Там же описан и способ получения всех решений этого уравнения по одному найденному решению, и, кстати говоря, этот способ при внимательном рассмотрении абсолютно эквивалентен способу, описанному в предыдущем пункте.
